% Prepared by Calvin Kent
\documentclass[12pt,oneside]{book} %
\usepackage{CKpreamble}
\usepackage{CKlecture}
\usepackage{mdframed}
\usepackage{import}
\usepackage{pdfpages}
\usepackage{euscript}
\usepackage{transparent}
\usepackage{xcolor}

\newcommand{\incfig}[2][1]{%
    \def\svgwidth{#1\columnwidth}
    \import{./figures/}{#2.pdf_tex}
}

\pdfsuppresswarningpagegroup=1

%
\renewcommand*{\doctitle}{Class Based Lecture Notes}
\makeatletter\patchcmd{\chapter}{\if@openright\cleardoublepage\else\clearpage\fi}{}{}{}\makeatother % only used in class based
\begin{document}
	% Start of Class settings
	\renewcommand*{\term}{Term 2} % Term
	\renewcommand*{\coursecode}{MCR3U} % Course code
	\renewcommand*{\coursename}{Course Name} % Full course name
	\renewcommand*{\thelecnum}{4} % Lecture number
	\renewcommand*{\profname}{Prof Name} % Prof Name
	\renewcommand*{\colink}{http://www.student.math.uwaterloo.ca/~c2kent} % Course outline link
	% End of Class settings
	\clearpage
	\pagenumbering{arabic}
	\pagestyle{classlecture}
	%%% Note to user: CTRL + F <CHANGE ME:> (without the angular brackets) in CKpreamble to specify graphics paths accordingly.
	%%% If a new chapter was started in the middle of a lecture, \fix chap{Second Chapter} must be used immediately above the next lecture.
	% Course notes start
\setchap{2}{Protozoa (\textit{Additional notes})}
\begin{lec}{March 20222}
	\chapter{\chapname\chaplec}

  \section*{Protozoa}
  \begin{itemize}
    \item Species within the Protist Kingdom.
    \item Are heterotrophic (Eat other things).
    \item Possess a membrane-bound nucleus.
    \item Confined to aquatic habitats.
  \end{itemize}

  
  Protozoa's are divided into four categories segregated based on movement.

  \section*{Amoeboids}
  \begin{itemize}
    \item Protists that change body shape in order to move.
    \item Expand and contract their \textbf{pseudopods} to grip and move, as a consequence body composition changes
       temporarily. 
    \item \textbf{Pseudopods:} Groups of cytoplasm temporarily used as ``feet'' for movement.
    \item \textbf{Phagocytosis:} Process in which certain cells consume energy (food).
       \begin{enumerate}
         \item \textbf{Recognition:} Detection of microbes using receptors on the cell membrane.
         \item \textbf{Engulfment:} Cell morphs around the microbe and traps it within the vacoule for digesting food.
         \item \textbf{Intraceullar Killing:} Lysosmes within the cells release antibacterial molecules to kill and digest
            the microbe, filtering any waste in the process.
         \item \textbf{Exocytosis:} Process of waste elimination.
       \end{enumerate}
    \item Can eat from any part of their body.
  \end{itemize}

  \section*{Flagellates}

  \begin{itemize}
    \item \textbf{Flagella:} Long hairlike organ, used similar to how fish use their tail to swim.
    \item Have either a thin \textbf{Pellicle} outer covering or a coating of a jellylike substance . 
    \item Reproduce either \textbf{asexually} (longitudinal splitting) or sexually.
    \item Can be either heterotrophic or autotrophic;
    \item \textbf{Zooflagellates:} Heterotrophic , eat use phagocytosis like amoebas.
    \item \textbf{Phytoflagellates:} Autotrophic (produces its own food), contain chlorophyll which allows them to obtain
       nutrients through \textit{photosynthesis}.
  \end{itemize}

  \section*{Ciliates}

  \begin{itemize}
    \item \textbf{Cilia:} Short hairs found around the body of celibates that enable them to move.
       (Movement similar to a boat)
    \item Contain two types of nuclei;
    \item \textbf{Macronucleus:} Larger nuclei, handles \textbf{all} non-reproductive cell functions.
    \item \textbf{Micronucleus:} Handles reproduction, genes are passed to offspring during reproduction, macronucleus
       degenerates and a new one is formed from the genes of the micronucleus.
  \end{itemize}

  \section*{Apicomplexa}

  \begin{itemize}
    \item A.K.A \textbf{Sporozoa}, because of their ability to form sporelike cells.
    \item Are \textbf{immobile}.
    \item They are \textbf{parasitic}, they rely on a host to latch onto not only for movement but for nutrients as well.
    \item \textbf{Apical Complex:} Organelle used to enter host cells.
    \begin{itemize}
      \item \textbf{Apical Cap:} Tip of the sporozoa.
      \item \textbf{Rhoptries:} Produce enzymes to ease entry into host cells.
    \end{itemize}
    \item Similar to viruses, trick host cells using apical complex.
  \end{itemize}


  







  







  




  





  

  
  



  


  



  

  

  





  




















	\end{lec}

\end{document}

