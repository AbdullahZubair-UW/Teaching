% Prepared by Calvin Kent
\documentclass[12pt,oneside]{book} %
\usepackage{CKpreamble}
\usepackage{CKlecture}
\usepackage{mdframed}
\usepackage{import}
\usepackage{pdfpages}
\usepackage{euscript}
\usepackage{transparent}
\usepackage{xcolor}

\newcommand{\incfig}[2][1]{%
    \def\svgwidth{#1\columnwidth}
    \import{./figures/}{#2.pdf_tex}
}

\pdfsuppresswarningpagegroup=1

%
\renewcommand*{\doctitle}{Class Based Lecture Notes}
\makeatletter\patchcmd{\chapter}{\if@openright\cleardoublepage\else\clearpage\fi}{}{}{}\makeatother % only used in class based
\begin{document}
	% Start of Class settings
	\renewcommand*{\term}{Term 2} % Term
	\renewcommand*{\coursecode}{MCR3U} % Course code
	\renewcommand*{\coursename}{Course Name} % Full course name
	\renewcommand*{\thelecnum}{4} % Lecture number
	\renewcommand*{\profname}{Prof Name} % Prof Name
	\renewcommand*{\colink}{http://www.student.math.uwaterloo.ca/~c2kent} % Course outline link
	% End of Class settings
	\clearpage
	\pagenumbering{arabic}
	\pagestyle{classlecture}
	%%% Note to user: CTRL + F <CHANGE ME:> (without the angular brackets) in CKpreamble to specify graphics paths accordingly.
	%%% If a new chapter was started in the middle of a lecture, \fix chap{Second Chapter} must be used immediately above the next lecture.
	% Course notes start
\setchap{3}{Algae (\textit{Additional notes})}
\begin{lec}{March 20222}
	\chapter{\chapname\chaplec}

  \subsection*{Characteristics of Algae}

  \begin{itemize}
    \item Only live in aquatic areas.
    \item Classified based on water they grow; its salt content, temp, quality, flagella content.
    \item Can be both autotrophic or heterotrophic.
  \end{itemize}

  \subsection*{Classifications of Algae}
  \begin{itemize}
    \item \textbf{Chlorophyta:} Are green algae.
    \item \textbf{Phaeophyta:} Are brown algae.
    \item \textbf{Rhodophyta:} Are red algae.
  \end{itemize}

  \section*{Chlorophyta}
  \begin{itemize}
    \item Found mostly in marine habitats; seawater, swamps, lakes and ponds.
    \item Single celled, prefer to form colonies.
    \item \textbf{Chlorophyll:} 
       \begin{itemize}
         \item The pigments within cells where photosynthesis occurs.
         \item Reflect high-energy green light which gives them their green color.
       \end{itemize}
  \end{itemize}


  \section*{Phaeophyta}

  \begin{itemize}
    \item Multicellular.
    \item \textbf{Blades:} Help algae collect as much light from the surface.
    \item \textbf{Holdfasts:} Rootlike appendages which attach themselves to rocks.
    \item \textbf{Fucoxanthin:} Works together with chlorophyll to enable photosynthesis. Reflect brown light which gives the
       plant its brown color.
  \end{itemize}

  \newpage

  \section*{Rhodophyta}
  \begin{itemize}
    \item \textbf{Phycoerythrin:} Pigments that reflect red light which give the plant its red color.
    \item \textbf{Filaments:} Long chains of single-celled algae.
    \item \textbf{Coralline algae:} Type of rhodophyta, important for coral reefs due to their ability to produce CaCO$_3$ which
       is a component of reefs. 
  \end{itemize}

  \section*{Diatoms}
  \begin{itemize}
    \item \textbf{Silica cell walls:} Cell walls are silica based, chemical properties including solubility
       electropositivity and brittility. 
    \item Has more than 2 Million species.
    \item Release oxygen into the environment. (1/3 of Earth oxygen from diatoms) making them a vital component to any
       ecosystem whose organism need a source of oxygen.
    \item \textbf{Environment:} Artic/Marine ; tropical reefs, sea water, forests.
  \end{itemize}

  \section*{Dino flagellates}
  \begin{itemize}
    \item Single-celled, \textbf{both} autotrophic \& heterotrophic.
    \item \textbf{Heterotrophic dinoflagellates:} Hunt and eat other protists or protozoa.
    \item \textbf{Autotrophic dinoflagellates:} Use their chlorophyll for photosynthesis.
    \item Have two flagella to assist in motion.
    \item Some are \textbf{Bioluminescent}; living organisms that are able to produce light.
    \item If they receive large quantities of nutrients they may \textbf{bloom}, which may produce toxins that can kill
       thousands of fish and also infect those who eat them.
    \item \textbf{Bloom:} A rapid reproduction of microorganisms.
  \end{itemize}





  
  


  
















  





  

  
  



  


  



  

  

  





  




















	\end{lec}

\end{document}

