% Prepared by Calvin Kent
\documentclass[12pt,oneside]{book} %
\usepackage{CKpreamble}
\usepackage{CKlecture}
\usepackage{mdframed}
\usepackage{import}
\usepackage{pdfpages}
\usepackage{euscript}
\usepackage{transparent}
\usepackage{xcolor}

\newcommand{\incfig}[2][1]{%
    \def\svgwidth{#1\columnwidth}
    \import{./figures/}{#2.pdf_tex}
}

\pdfsuppresswarningpagegroup=1

%
\renewcommand*{\doctitle}{Class Based Lecture Notes}
\makeatletter\patchcmd{\chapter}{\if@openright\cleardoublepage\else\clearpage\fi}{}{}{}\makeatother % only used in class based
\begin{document}
	% Start of Class settings
	\renewcommand*{\term}{Term 2} % Term
	\renewcommand*{\coursecode}{MCR3U} % Course code
	\renewcommand*{\coursename}{Course Name} % Full course name
	\renewcommand*{\thelecnum}{4} % Lecture number
	\renewcommand*{\profname}{Prof Name} % Prof Name
	\renewcommand*{\colink}{http://www.student.math.uwaterloo.ca/~c2kent} % Course outline link
	% End of Class settings
	\clearpage
	\pagenumbering{arabic}
	\pagestyle{classlecture}
	%%% Note to user: CTRL + F <CHANGE ME:> (without the angular brackets) in CKpreamble to specify graphics paths accordingly.
	%%% If a new chapter was started in the middle of a lecture, \fix chap{Second Chapter} must be used immediately above the next lecture.
	% Course notes start
\setchap{3}{Arthropods}
\begin{lec}{March 20222}
	\chapter{\chapname\chaplec}
  \textbf{Arthropods:} \textit{Animals with segmented bodies, allowing them to
    turn and manevour in very precise ways.}


  \textbf{Exoskeleton:} \textit{A hard outer body covering}.

  \textbf{Molting:} \textit{The name of the process in which arthropods shed and rebuild their outer shells}

  \textbf{Spermatophores:} Capsules that contain sperm cells.

  \subsection*{4 Different Classes  of Arthropods}

  \begin{itemize}
    \item \textbf{Crustaceans:} Crabs, shrimp and lobsters.
    \item \textbf{Chelicerates:} Scorpions, spiders and ticks.
    \item \textbf{Myriapods:} Centipedes and millipedes.
    \item \textbf{Insects:} Flies and Bees
  \end{itemize}
  
  \subsection*{Crustaceans}
  Arthropods which contain a head, a thorax and an abdomen.\newline

  \textbf{Abdomen:} \textit{The part of animal's body that contains the stomach and
  intestines.}
  
  \textbf{Thorax:} \textit{The part of the animal which usually contains the heart and
    lungs. (Often found between the neck and the abdomen).}
  
  \subsection*{Chelicerates}
  Arthropods which have a cephalothorax, an abdomen and a jawline structure called the chelicera.

  \textbf{Cephalothorax:} \textit{The fused head and throax of arthropods.}
  
  \textbf{Chelicera:} \textit{Jawlike structure often shaped as articulated fangs. (Head of a spider)}

  \textbf{Pedipalps:} \textit{Appendages that are close to the chelicera, help with taste and smell as well as working
    as external weapons. (Think spiders again)}

  \newpage

  \subsection*{Myriapods}
  Arthropods with very long segmented bodies, and a pair of antennae. Each segment of their bodies have a pair of legs,
  and, depending on the type of body, myriads can have anywhere between 10 to 750 legs.

  Examples: Millipedes, centipedes.


  \subsection*{Insects}

  Of the most common arthropods. Bodies are divided into three segments;

  \begin{enumerate}
    \item \textbf{Head:} Containing a pair of antennae.
    \item \textbf{Thorax:} Containing six legs and sometimes a pair of wings.
    \item \textbf{abdomen} 
  \end{enumerate}


  \textbf{Metamorphosis:} \textit{A process in which an organism passes through three or four distinct life phases.}

  \underline{Processes of development (Difference of opinion):}
  \begin{enumerate}
    \item \textbf{Egg:} Beginning stage of life 
    \item \textbf{Larva:} Secondary stage, primary goals are to consume as much energy as possible while undergoing processes
    such as molting and separating of cells.
    \item \textbf{Pupa:} Inactive immature state prior to adult stage, larval cells die off to provide energy for the
    processes involved in the development of the final stage.
    \item \textbf{Adult:} Final stage of development.
  \end{enumerate}
  


  


  



  

  

  





  




















	\end{lec}

\end{document}

