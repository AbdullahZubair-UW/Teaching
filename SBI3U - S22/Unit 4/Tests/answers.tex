% Prepared by Calvin Kent
%
% Assignment Template v19.02
%
%%% 20xx0x/MATHxxx/Crowdmark/Ax
%
\documentclass[12pt]{article} %
\usepackage{amsthm}
\usepackage{CKpreamble}
\usepackage{CKassignment}
\usepackage{dirtytalk}
\usepackage{afterpage}
\usepackage{csquotes}
\usepackage{mdframed}
\usepackage{euscript}
\usepackage{tikz}
\usepackage{pgfplots}

\newcommand\blankpage{%
  \null
  \thispagestyle{empty}%
  \addtocounter{page}{-1}%
  \newpage}

%
\begin{document}
	\pagenumbering{arabic}
	% Start of class settings ...
	\renewcommand*{\coursecode}{MATH 235} % renew course code
	\renewcommand*{\assgnnumber}{Assignment 1} % renew assignment number
	\renewcommand*{\submdate}{September 14, 2021} % renew the date
	\renewcommand*{\studentfname}{Abdullah} % Student first name
	\renewcommand*{\studentlname}{Zubair} % Student last name
    \renewcommand*{\proofname}{Proof:}
	% \renewcommand*{\studentnum}{20836288} % Student number

	\renewcommand\qedsymbol{$\blacksquare$}
	\setfigpath
	% End of class settings	
	% \pagestyle{crowdmark}
	\newgeometry{left=18mm, right=18mm, top=22mm, bottom=22mm} % page is set to default values
	\fancyhfoffset[L,O]{0pt} % header orientation fixed
	% End of class settings
	%%% Note to user:
	% CTRL + F <CHANGE ME:> (without the angular brackets) in CKpreamble to specify graphics paths accordingly.
	% The command \circled[]{} accepts one optional and one mandatory argument.
	% Optional argument is for the size of the circle and mandatory argument is for its contents.
	% \circled{A} produces circled A, with size drawn for letter A. \circled[TT]{A} produces circled A with size drawn for TT.
	% https://github.com/CalvinKent/My-LaTeX
	%%%

	%%%%%%%%%%%%%%%%%%%%%%%%%%%%%%%%%%%%%%%%%%%%%%%%%%%%%%%%%%%%%%%%%%%%%%%%%%%%%%%%%%%%%%%%%%%%%%%%%%%%%%%%%%%%%%%%%%
	%%%                        CUSTOM MACRO VIM-TEX                                                      (Word Wrap->
	%%       call IMAP('NOM', '\nomenclature{}', 'tex')               

	%%%%%%%%%%%%%%%%%%%%%%%%%%%%%%%%%%%%%%%%%%%%%%%%%%%%%%%%%%%%%%%%%%%%%%%%%%%%%%%%%%%%%%%%%%%%%%%%%%%%%%%%%%%%%%%%%%

	% Crowdmark assignment start
	% qnumber, name, points

\begin{center}
	\textbf{\underline{\Huge{Unit 4 Test - Animals - SOLUTION}}}
\end{center}

\subsection*{Preamble:}
Answer the following questions in the \textbf{Answer} section. Give detailed answers and use proper style and scientific terminology.
\section*{Questions}
  \begin{qstn}
    What is an exoskeleton?
  \end{qstn}
  \begin{soln}
    A hard outershell covering.
  \end{soln}

  \begin{qstn}
    What is the purpose of excretion?
  \end{qstn}
  \begin{soln}
    Excretion helps an organism stay healthy by removing waste product's.
  \end{soln}

  \begin{qstn}
    What class of modern mammals do humans fall under and why?
  \end{qstn}
  \begin{soln}
    Eutherians, due to development within a placenta and the existence of an umbilical cord.
  \end{soln}

  \begin{qstn}
    What do annelids eat?
  \end{qstn}
  \begin{soln}
    Organic matter within soil.
  \end{soln}

  \begin{qstn}
    What is the purpose of the umbilical cord?
  \end{qstn}
  \begin{soln}
    A cord which carries food, water and oxygen to the embryo and returns waste back to the mother. (Belly button was the
        connecting point).
  \end{soln}


  \begin{qstn}
    Explain the process of metamorphosis. 
  \end{qstn}
  \begin{soln}
    Process in which an organsim goes through three - four stages of development. Such stages include the Egg, Larva, Puppa, Adult.
  \end{soln}

  \begin{qstn}
    What are crustaceans?
  \end{qstn}
  \begin{soln}
  Arthropods which contain a head, a thorax and an abdomen.\newline
  \end{soln}

  \begin{qstn}
    What is one way that a species can adapt to an external environment's change?
  \end{qstn}
  \begin{soln}
    Through a process of evolution.
  \end{soln}

  \begin{qstn}
    How do chordates filter water?
  \end{qstn}
  \begin{soln}
    Two answers are valid here;\\
    \textbf{Tunicate}: Intake through oral siphon, followed by Branchial Basket, remains exit through Atrial siphon.\\
    \textbf{Cephalochordates:} Use pharyngeal slits to filter, intake through oral cirri, food particles caught by mucus and remaining proceeds through the atripore.
  \end{soln}

  \begin{qstn}
    What are the two main groups of nonvertebrate chordates?
  \end{qstn}
  \begin{soln}
    Tunicates, Cephalochordates.
  \end{soln}

  \begin{qstn}
    How do caecilians ``see''?
  \end{qstn}
  \begin{soln}
    Caecilians use their skin, feeling for vibrations in the ground.
  \end{soln}

  \begin{qstn}
    How do animals get their nutrients?
  \end{qstn}
  \begin{soln}
    By eating/consuming other organisms.
  \end{soln}

  \begin{qstn}
    What is an amnion?
  \end{qstn}
  \begin{soln}
    A thin transparent fluid in which the embryo is enclosed and suspended whilst being cushioned by its
      surroundings
  \end{soln}

  \begin{qstn}
    What is the function of the air bladder found in fish?
  \end{qstn}
  \begin{soln}
    Helps them remain afloat and maintain a stable buoyancy
  \end{soln}

  \begin{qstn}
    Where do starfish hold their vital organs?
  \end{qstn}
  \begin{soln}
    Starfish house an equal share of vital organs in each arm.
  \end{soln}

  \begin{qstn}
    Why must amphibians undergo metamorphosis? What do they gain?
  \end{qstn}
  \begin{soln}
    Must undergo a process of \textbf{metamorphosis} in order to develop organs to enable the transition to a
    terrestrial environment. Necessary organs include,
    \begin{itemize}
      \item Limbs, to allow manoeuvrability.
      \item Eyes to enable vision on land.
      \item Lungs to enable breathing methods on terrestrial land. (Gas exchange with air not water) (They also breathe
      through their skin)
    \end{itemize}
  \end{soln}

  \begin{qstn}
    What are the scales of reptiles made of?
  \end{qstn}
  \begin{soln}
    Keratin.
  \end{soln}

  \begin{qstn}
    What are the cells that contain the toxin that cnidarians secrete called?
  \end{qstn}
  \begin{soln}
    Cnidocytes.
  \end{soln}




\end{document}
