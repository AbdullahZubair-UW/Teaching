% Prepared by Calvin Kent
\documentclass[12pt,oneside]{book} %
\usepackage{CKpreamble}
\usepackage{CKlecture}
\usepackage{mdframed}
\usepackage{import}
\usepackage{pdfpages}
\usepackage{euscript}
\usepackage{transparent}
\usepackage{xcolor}

\newcommand{\incfig}[2][1]{%
    \def\svgwidth{#1\columnwidth}
    \import{./figures/}{#2.pdf_tex}
}

\pdfsuppresswarningpagegroup=1

%
\renewcommand*{\doctitle}{Class Based Lecture Notes}
\makeatletter\patchcmd{\chapter}{\if@openright\cleardoublepage\else\clearpage\fi}{}{}{}\makeatother % only used in class based
\begin{document}
	% Start of Class settings
	\renewcommand*{\term}{Term 2} % Term
	\renewcommand*{\coursecode}{MCR3U} % Course code
	\renewcommand*{\coursename}{Course Name} % Full course name
	\renewcommand*{\thelecnum}{4} % Lecture number
	\renewcommand*{\profname}{Prof Name} % Prof Name
	\renewcommand*{\colink}{http://www.student.math.uwaterloo.ca/~c2kent} % Course outline link
	% End of Class settings
	\clearpage
	\pagenumbering{arabic}
	\pagestyle{classlecture}
	%%% Note to user: CTRL + F <CHANGE ME:> (without the angular brackets) in CKpreamble to specify graphics paths accordingly.
	%%% If a new chapter was started in the middle of a lecture, \fix chap{Second Chapter} must be used immediately above the next lecture.
	% Course notes start
\setchap{6}{Amniotes (\textit{Additional notes})}
\begin{lec}{March 20222}
	\chapter{\chapname\chaplec}

  \textbf{Embryo:} (\textbf{Ask Class})\\
  
  \textbf{Amnion:} (\textbf{Ask Class})\\

  \textbf{Amniotes} Are made up of reptiles, birds and mammals. These are species whose embryos develop within an amnion.


  \section*{Structure of Amniote Egg}
  \begin{itemize}
    \item \textbf{Amnion} is the outer layer surrounding and protecting the embryo.
    \item Within the egg exists a liquid known as \textbf{amniotic fluid} to provide a stable fluid environment.
    \item The \textbf{allantois} serves two roles;
      \begin{enumerate}
        \item Assists in gas exchange.
        \item Waste Removal.
      \end{enumerate}
    \item Food intake chamber is known as the \textbf{yolk sac}.
    \item The \textbf{chorion} provides the overall enclosed structure.
  \end{itemize}

  \section*{Body Temperature}
  \subsection*{Ectothermic}
  \begin{itemize}
    \item Require an external source of heat to maintain body temperatures.
    \item Most reptiles are ectothermic.
  \end{itemize}

  \subsection*{Endothermic}
  \begin{itemize}
    \item Can self-regulate body temperatures by changing \textit{basal metabolic rate}.
    \item Most mammals and birds are endothermic.
  \end{itemize}
  
  \newpage
 
  \section*{Reptiles}
  
  \subsubsection*{Herbivorous}
  \begin{itemize}
    \item Species that only eat plants.
    \item Structure of certain organs adapt to such dietary behaviours.
    \item \textbf{Not} limited to mammals!
    \item \underline{Examples:} Cow, Goat, Deer, Horse.
  \end{itemize}

  \subsubsection*{Carnivorous}
  \begin{itemize}
    \item Species that only eat meat.
    \item Structure of certain organs adapt to such dietary behaviours.
    \item \textbf{Not} limited to mammals!
    \item \underline{Examples:} Bears, Great white Sharks, Hyenas, Seals.
  \end{itemize}
  
  \subsubsection*{Omnivorous}
  \begin{itemize}
    \item Species that eat both meat and plants.
    \item Organs enable them to digest/consume both types of food.
    \item \textbf{Not} limited to mammals!
    \item \underline{Examples:} Lobsters, Dogs, Humans, Bears.
  \end{itemize}
  
  \subsubsection*{Body Structure}
  \begin{itemize}
    \item Reptiles have scales made of \textbf{keratin} to help retain and regulate their bodies' moisture and heat.
    \item Undergo molting whereby they adapt to shade in order to do so.
    \item \textbf{Keratin:} Structural proteins that serve important structural and protective roles, as well as cell development in
    some cases.
  \end{itemize}


  \section*{Birds}
  \begin{itemize}
    \item The wings are shaped precisely weighted to one side to help generate lift and also help with CG balancing.
    \item The main bone known as the \textbf{humerus} is hollow instead of solid to reduce their
    overall weight while preserving their skeleton structure.
    \item Digestive system allows them to eat on the fly and digest later. 
    \item Contrasting to other mammals, birds lay eggs. This makes as nesting eggs is far easier than carrying their
    offspring in their bodies.
  \end{itemize}

  \section*{Mammals}
  \begin{itemize}
    \item Bodies wrapped in fur, produce milk to feed their offspring.
    \item Have a four-chambered heart, limbs/fins.
    \item \textbf{Diaphragm:}
    \begin{itemize}
      \item Contracts during inhalation to create a vacuum effect to enable air intake.
      \item Expands during exhalation to release air.
    \end{itemize}
    \item \textbf{Mammary Glands:}
    \begin{itemize}
      \item Ducts that female mammals use to feed milk to their offspring.
    \end{itemize}
  \end{itemize}

  \subsection*{Modern Mammals}
  
   \subsubsection*{Monotremes}
   \begin{itemize}
     \item Lay eggs instead giving birth.
     \item \underline{Examples:} Platypus, Echidnas.
   \end{itemize}
   
   \subsubsection*{Marsupials}
   \begin{itemize}
     \item Give birth to underdeveloped embryos that continue to mature and develop inside the pouch of their mother.
     \item \underline{Examples:} Koala, giraffe, wombat.
   \end{itemize}

   \subsubsection*{Eutherians}
   \begin{itemize}
     \item AKA \textbf{Placentals}.
     \item Develop their offspring within the \textbf{placenta}.
     \item \textbf{Placenta:} An organ that connects the mother to her embryo.
     \item \textbf{Umbilical Cord:} A cord which carries food, water and oxygen to the embryo and returns waste back to the
     mother. (Belly button was the connecting point).
   \end{itemize}


















  





  




















	\end{lec}

\end{document}

