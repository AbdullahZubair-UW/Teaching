% Prepared by Calvin Kent
\documentclass[12pt,oneside]{book} %
\usepackage{CKpreamble}
\usepackage{CKlecture}
\usepackage{mdframed}
\usepackage{import}
\usepackage{pdfpages}
\usepackage{euscript}
\usepackage{transparent}
\usepackage{xcolor}

\newcommand{\incfig}[2][1]{%
    \def\svgwidth{#1\columnwidth}
    \import{./figures/}{#2.pdf_tex}
}

\pdfsuppresswarningpagegroup=1

%
\renewcommand*{\doctitle}{Class Based Lecture Notes}
\makeatletter\patchcmd{\chapter}{\if@openright\cleardoublepage\else\clearpage\fi}{}{}{}\makeatother % only used in class based
\begin{document}
	% Start of Class settings
	\renewcommand*{\term}{Term 2} % Term
	\renewcommand*{\coursecode}{MCR3U} % Course code
	\renewcommand*{\coursename}{Course Name} % Full course name
	\renewcommand*{\thelecnum}{4} % Lecture number
	\renewcommand*{\profname}{Prof Name} % Prof Name
	\renewcommand*{\colink}{http://www.student.math.uwaterloo.ca/~c2kent} % Course outline link
	% End of Class settings
	\clearpage
	\pagenumbering{arabic}
	\pagestyle{classlecture}
	%%% Note to user: CTRL + F <CHANGE ME:> (without the angular brackets) in CKpreamble to specify graphics paths accordingly.
	%%% If a new chapter was started in the middle of a lecture, \fix chap{Second Chapter} must be used immediately above the next lecture.
	% Course notes start
\setchap{4}{Chordates (\textit{Additional notes})}
\begin{lec}{March 20222}
	\chapter{\chapname\chaplec}

  
  Chordates are animals that exhibit certain characteristics. 

  \begin{enumerate}
    \item \textbf{Notochord:} 
      \begin{itemize}
        \item Supporting rod that runs through the length of the body. 
        \item Plays a fundamental role in the development of the overall vertebrate structure of the animal. 
        \item Supports the growth and development of the bone and cartilage structure.
      \end{itemize}    
    \item \textbf{Nerve Cord:}
      \begin{itemize}
        \item A hollow supporting structure that runs along the dorsal (top side) of the animal.
        \item Framework for the central nervous system, 
          wherein \textit{neurons} (Cells that transmit electrical messages) are transmitted.
      \end{itemize}
    \item \textbf{Pharyngeal slits:}
      \begin{itemize}
        \item Opening slits between the pharynx (near to throat) and the exterior of the chordate.   
        \item For certain chordates, this is used for gas exchange, while others use it to filter food particles from water.
      \end{itemize}
    \item \textbf{Post-Aanal Tail:} 
     \begin{itemize}
       \item Extension of the body that runs through the anal passage. 
     \end{itemize}
  \end{enumerate}

  \section{Nonvertebrate Chordates}
  
  \section*{Tunicates}
  \begin{itemize}
    \item \textbf{Tunic:} Stable, flexible body covering.
    \item They attach themselves to rigid objects likes rocks or coral.
    \item \textbf{Oral Siphon:} The siphon by which they draw seawater through their bodies.
    \item \textbf{Branchial Basket:} The filtering system within their bodies used to filter oxygen and food particles
    contained in the water that they consume.
    \item \textbf{Atrial Siphon:} Remains of the filtering process are expelled through this siphon.
  \end{itemize}

  \textbf{Examples:} Larvae, Ascidians.

  \subsection*{Cephalochordates}

  \begin{itemize}
    \item Also known as \textbf{lancelets}.
    \item Occupy themselves mostly under sand.
    \item Have almost 100 \textbf{pharyngeal slits} used to filter food particles from water intake.
    \begin{itemize}
      \item The process begins with water entering the \textit{oral cirri}.
      \item Then the water proceeds to the pharyngeal slits.
      \item The food particles are caught by the mucus and the water proceeds out of the atrium through the \textit{atriopore}
    \end{itemize}
    \item Reproduce sexually by means of shedding sperm cells and eggs directly in the water. 

  \end{itemize}

  \subsection*{Reproduction}

  \begin{itemize}
    \item All vertebrate chordates are sexual reproducers.
    \item \textit{Aquatic} species fertilize eggs externally by means of passing sperm cells into the water for females to
    latch onto.
    \item \textit{Terrestrial} species fertilize eggs internally, by means of direct contact between males and females. 
  \end{itemize}







  




  





  

  
  



  


  



  

  

  





  




















	\end{lec}

\end{document}

