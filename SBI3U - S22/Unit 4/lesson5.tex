% Prepared by Calvin Kent
\documentclass[12pt,oneside]{book} %
\usepackage{CKpreamble}
\usepackage{CKlecture}
\usepackage{mdframed}
\usepackage{import}
\usepackage{pdfpages}
\usepackage{euscript}
\usepackage{transparent}
\usepackage{xcolor}

\newcommand{\incfig}[2][1]{%
    \def\svgwidth{#1\columnwidth}
    \import{./figures/}{#2.pdf_tex}
}

\pdfsuppresswarningpagegroup=1

%
\renewcommand*{\doctitle}{Class Based Lecture Notes}
\makeatletter\patchcmd{\chapter}{\if@openright\cleardoublepage\else\clearpage\fi}{}{}{}\makeatother % only used in class based
\begin{document}
	% Start of Class settings
	\renewcommand*{\term}{Term 2} % Term
	\renewcommand*{\coursecode}{MCR3U} % Course code
	\renewcommand*{\coursename}{Course Name} % Full course name
	\renewcommand*{\thelecnum}{4} % Lecture number
	\renewcommand*{\profname}{Prof Name} % Prof Name
	\renewcommand*{\colink}{http://www.student.math.uwaterloo.ca/~c2kent} % Course outline link
	% End of Class settings
	\clearpage
	\pagenumbering{arabic}
	\pagestyle{classlecture}
	%%% Note to user: CTRL + F <CHANGE ME:> (without the angular brackets) in CKpreamble to specify graphics paths accordingly.
	%%% If a new chapter was started in the middle of a lecture, \fix chap{Second Chapter} must be used immediately above the next lecture.
	% Course notes start
\setchap{5}{Anamniotes (\textit{Additional notes})}
\begin{lec}{March 20222}
	\chapter{\chapname\chaplec}

  \textbf{Embryo:} An organism in its early development stage.\\
  
  \textbf{Amnion:} A thin transparent fluid in which the embryo is enclosed and suspended whilst being cushioned by its
  surroundings.\\

  \textbf{Anamniotes} are a group composed of fish and amphibians whose embryos are \textbf{not} enclosed by an amnion.

  As a consequence, anamniotes must lay their eggs in water, otherwise their eggs would dry out.\\

  \section{Fish}
  Characteristics of fish,
  \begin{itemize}
    \item Water Habitat.
    \item Breathe underwater using their \textbf{gills} by a method of gas exchange with water to extract oxygen atoms.
    \item Fins at the \textbf{top and sides} for assistance in motion and stability.
    \item Some fish are equipped with a \textbf{air bladder} which helps them remain afloat and maintain a stable buoyancy 
  \end{itemize}
  The three main classes of fish are,
  \begin{itemize}
    \item Agnatha.
    \item Chondrichthyes
    \item Osteopathy's.
  \end{itemize}

  \newpage

  \section{Amphibians}
  Characteristics of amphibians,
  \begin{itemize}
    \item Young/larval stages in water, adult stages in land.
    \item \textit{Mostly} fertilize externally (Eggs fertilized by sperm cells).
    \item Must undergo a process of \textbf{metamorphosis} in order to develop organs to enable the transition to a
    terrestrial environment. Necessary organs include,
    \begin{itemize}
      \item Limbs, to allow manoeuvrability.
      \item Eyes to enable vision on land.
      \item Lungs to enable breathing methods on terrestrial land. (Gas exchange with air not water) (They also breathe
      through their skin)
    \end{itemize}
      \item Must keep their bodies moist/wet and hence occupy near to a water source.
      \item Hibernate in colder months.
  \end{itemize}

  Main groups of amphibians,
  \begin{itemize}
    \item Ochthyostegalia.
    \item Anura.
    \item Urodela.
    \item Apoda.
  \end{itemize}













  




  





  

  
  



  


  



  

  

  





  




















	\end{lec}

\end{document}

