% Prepared by Calvin Kent
%
% Assignment Template v19.02
%
%%% 20xx0x/MATHxxx/Crowdmark/Ax
%
\documentclass[12pt]{article} %
\usepackage{amsthm}
\usepackage{CKpreamble}
\usepackage{CKassignment}
\usepackage{dirtytalk}
\usepackage{csquotes}
\usepackage{mdframed}
\usepackage{euscript}
\usepackage{tikz}
\usepackage{pgfplots}

%
\begin{document}
	\pagenumbering{arabic}
	% Start of class settings ...
	\renewcommand*{\coursecode}{MATH 235} % renew course code
	\renewcommand*{\assgnnumber}{Assignment 1} % renew assignment number
	\renewcommand*{\submdate}{September 14, 2021} % renew the date
	\renewcommand*{\studentfname}{Abdullah} % Student first name
	\renewcommand*{\studentlname}{Zubair} % Student last name
    \renewcommand*{\proofname}{Proof:}
	% \renewcommand*{\studentnum}{20836288} % Student number

	\renewcommand\qedsymbol{$\blacksquare$}
	\setfigpath
	% End of class settings	
	% \pagestyle{crowdmark}
	\newgeometry{left=18mm, right=18mm, top=22mm, bottom=22mm} % page is set to default values
	\fancyhfoffset[L,O]{0pt} % header orientation fixed
	% End of class settings
	%%% Note to user:
	% CTRL + F <CHANGE ME:> (without the angular brackets) in CKpreamble to specify graphics paths accordingly.
	% The command \circled[]{} accepts one optional and one mandatory argument.
	% Optional argument is for the size of the circle and mandatory argument is for its contents.
	% \circled{A} produces circled A, with size drawn for letter A. \circled[TT]{A} produces circled A with size drawn for TT.
	% https://github.com/CalvinKent/My-LaTeX
	%%%

	%%%%%%%%%%%%%%%%%%%%%%%%%%%%%%%%%%%%%%%%%%%%%%%%%%%%%%%%%%%%%%%%%%%%%%%%%%%%%%%%%%%%%%%%%%%%%%%%%%%%%%%%%%%%%%%%%%
	%%%                        CUSTOM MACRO VIM-TEX                                                      (Word Wrap->
	%%       call IMAP('NOM', '\nomenclature{}', 'tex')               

	%%%%%%%%%%%%%%%%%%%%%%%%%%%%%%%%%%%%%%%%%%%%%%%%%%%%%%%%%%%%%%%%%%%%%%%%%%%%%%%%%%%%%%%%%%%%%%%%%%%%%%%%%%%%%%%%%%

	% Crowdmark assignment start
	% qnumber, qname, points

\begin{center}
	\textbf{\underline{\Huge{Post Reading Discussion - Section 3}}}
\end{center}

\section*{Due Date: Wednesday, March 9}
\large{\textbf{Page Range :} 60 - 90}



\section*{Discussion questions}
  \begin{qstn}
    Santiago was finally able to see the fish. Describe what he saw.
  \end{qstn}
  \begin{qstn}
    Santiago considers the fish more noble and more able. What quality do humans have that fish don't?
  \end{qstn}
  \begin{qstn}
    Why do you think that Santiago dreams about lions?
  \end{qstn}
  \begin{qstn}
    How did Santiago know that the fish was getting tired?
  \end{qstn}
  \begin{qstn}
    What do you think Santiago means by his comment,
    \say{I'm glad we do not have to kill the stars"}?
  \end{qstn}
  \begin{qstn}
    What was Santiago's plan for harpooning the fish?
  \end{qstn}
  \begin{qstn}
    Why was Santiago uncomfortable in the boat?
  \end{qstn}

\section*{Written Assignment}
\begin{enumerate}
  \item What major conflicts have you noticed throughout the novel so far?
  \item As Santiago dozes in the boat, he has three dreams. What are they, and what do you think they mean to Santiago and/or symbolize in this novel?
\end{enumerate}







\end{document}































