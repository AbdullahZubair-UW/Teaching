\documentclass[12pt,oneside]{book} %
\usepackage{CKpreamble}
\usepackage{CKlecture}
\usepackage{mdframed}
\usepackage{import}
\usepackage{pdfpages}
\usepackage{transparent}
\usepackage{xcolor}

\newcommand{\incfig}[2][1]{%
    \def\svgwidth{#1\columnwidth}
    \import{./figures/}{#2.pdf_tex}
}

\pdfsuppresswarningpagegroup=1

%
\renewcommand*{\doctitle}{Class Based Lecture Notes}
\makeatletter\patchcmd{\chapter}{\if@openright\cleardoublepage\else\clearpage\fi}{}{}{}\makeatother % only used in class based
\begin{document}
	% Start of Class settings
	\renewcommand*{\term}{Term 2} % Term
	\renewcommand*{\coursecode}{MCR3U} % Course code
	\renewcommand*{\coursename}{Course Name} % Full course name
	\renewcommand*{\thelecnum}{2} % Lecture number
	\renewcommand*{\profname}{Prof Name} % Prof Name
	\renewcommand*{\colink}{http://www.student.math.uwaterloo.ca/~c2kent} % Course outline link
	% End of Class settings
	\clearpage
	\pagenumbering{arabic}
	\pagestyle{classlecture}
	%%% Note to user: CTRL + F <CHANGE ME:> (without the angular brackets) in CKpreamble to specify graphics paths accordingly.
	%%% If a new chapter was started in the middle of a lecture, \fixchap{Second Chapter} must be used immediately above the next lecture.
	% Course notes start
\setchap{2}{Introduction to Functions}
\begin{lec}{November 2021}
	\chapter{\chapname\chaplec}
	\begin{mdframed}
		\begin{defn}
			Let $\mathcal{A}$ and $\mathcal{B}$  be sets. A  \textbf{function}, $f$, is rule that acts on each element in $\mathcal{A}$,
			and assigns it to some \textbf{unique} element in the set $\mathcal{B}$. We say that,
      \begin{itemize}
        \item $\mathcal{A}$ is the \textbf{domain} of $f$.
        \item $\mathcal{B}$ is the \textbf{co-domain} of $f$.
      \end{itemize}
		\end{defn}
    (I will define the word 'domain' in more detail later on)
	\end{mdframed}
	Its best to visualize this using the concept of diagrams I introduced to you about sets form last  lecture \textbf{(IN class)}.

	\begin{notn}
		Since a function $f$ assigns elements from the set $\mathcal{A}$ to elements in the set $\mathcal{B}$, we write this
		symbolically as,
		\[
					f \colon \mathcal{A} \to \mathcal{B}
		.\] 
		You can read this as, ``$f$ goes from the set $\mathcal{A}$ to $\mathcal{B}$ ''.
	\end{notn}

	\begin{notn}
			When $f$ will act on some element $a\in \mathcal{A}$, then symbolically we write
			\[
							f(a)
			.\] You can also read this as ``$f$ at $a$ '' or  ``$f$ evaluated at $a$ '', etc. \textbf{This is NOT multiplication, just
			notation.}
	\end{notn}

	\begin{ex}
		Lets take a look at our first function. Let $\mathcal{A} = \{2,3,4\}, \mathcal{B} = \{3,4,5\} $ be sets, lets define a
		function $f$,
		\begin{itemize}
			\item $f \colon \mathcal{A} \to \mathcal{B}$. (Must define this)
			\item \textbf{rule of $f$:} Take each element in $\mathcal{A}$, and add $1$.
		\end{itemize}
	\end{ex}

	\begin{rem}
			\textbf{Note!} We do not require a function that goes from $\mathcal{A}$ to $\mathcal{B}$ to map to \emph{every} element in
			$B$.
	\end{rem}

	\begin{ex}
		Lets take another function. Let $\mathcal{H} = \{2,3,4\}, \mathcal{T} = \{3,4,5\} $ be sets, lets define a
		function $g$,
		\begin{itemize}
			\item $g \colon \mathcal{H} \to \mathcal{T}$.
			\item \textbf{rule of $f$:} Take each element in $\mathcal{H}$, and subtract $1$.
		\end{itemize}
	\end{ex}

	\begin{rem}
			\textbf{Notice} that the definition of a function requries that each element in $\mathcal{A}$ is mapped to a
			\emph{unique} element in $\mathcal{B}$. \textbf{(Explanation in class)}
	\end{rem}

	\newpage

	\begin{notn}
			How to write the rule of a function symbolically.\\
			When rules of functions begin to get more complex, we need to switch over to writing rules of functions symbolically.
			So when we say something like ``Take each element of $\mathcal{H}$ and subtract one'', how can we write this symbolically?
			We can say ``Take each element $h\in \mathcal{H}$ and preform,
						\[
						f(h) = h - 1
						.\] 
			This translates to, take each element in $\mathcal{H}$ and subtract that element by one. Now we have a much better may of defining the
			rules of functions.
	\end{notn}

	\begin{ex}
		Translate the following symbolical rules to plain English.
		\begin{enumerate}[label=(\alph*)]
			\item \textbf{rule of $Q$:} For every element $a\in \mathcal{A}$, evaluate $Q(a) = a + 1$.
			\item \textbf{rule of $M$:} For every element $x\in \mathcal{X}$, evaluate $M(x) = \frac{x}{2}$.
			\item \textbf{rule of $Z$:} For every element $b\in \mathcal{B}$, evaluate  $Z(b) = 2b$.
			\item \textbf{rule of $R$:} For every element $c\in \mathcal{C}$, evaluate $R(c) = \sqrt{c} $.
			\item \textbf{rule of $S$:} For every element $y\in \mathcal{Y}$, evaluate $S(y) = y^2 - 2$.
		\end{enumerate}
	\end{ex}
	
	\begin{ex}
		Let $\mathcal{A} = \R$, and $\mathcal{B} = \R$, lets define a function $f$,
		\begin{itemize}
			\item $f \colon \mathcal{A} \to \mathcal{B}$.
			\item \textbf{rule of $f$:} For every element $a\in \mathcal{A}$, evaluate $f(a) = a^2 - 5a + 6$.
		\end{itemize}
		With this definition, determine the following outputs
		\begin{enumerate}[label=(\alph*)]
			\item $f(-1)$
			\item $f(\frac{2}{3})$
			\item $f(f(3))$
		\end{enumerate}
	\end{ex}

	\begin{ex}
		\textbf{(Class Example)} Let $\mathcal{X} = \R$, and $\mathcal{Y} = \R$, lets define \emph{two} functions $f$,$g$,
		\begin{itemize}
		\item $f \colon \mathcal{X} \to \mathcal{Y}$, $g \colon \mathcal{X} \to \mathcal{Y}$.
		\item \textbf{rule of $f$:} For every element $x\in \mathcal{X}$, evaluate $f(x) = x^2 - 4 $.
		\item \textbf{rule of $g$:} For every element $x\in \mathcal{X}$, evaluate $g(x) = x - 1$.
		\end{itemize}
		With this definition, determine the following outputs
		\begin{enumerate}[label=(\alph*)]
			\item $f(-2)$
			\item $g(0)$
			\item $f(g(3))$
		\end{enumerate}
	\end{ex}

	\section{Graphing/Sketching Functions}
	Lets take both functions from our previous example, $g(x) = x - 1$ and $f(x) = x^2 -  4$. Notice that the right hand side of
	$g$ is a simple linear function. Lets graph it \textbf{(In class)}.

	Notice that the right hand side of $f$ is a quadratic function, lets sketch it as well \textbf{(In class)}.

  \begin{rem}
      There is something called the \textbf{vertical line test} that helps you determine wether or not a graph is a function or
      not. You use it by drawing a straight line through your graph (until it touches it everywhere possilbe). If your vertical
      line intersects the graph at two or more \emph{distinct} points, then the graph is \emph{not} a function. Else, it is a
      function.
  \end{rem}

  \begin{ex}
    Use the vertical line test to show that $x^2 + y^2 = 4$ is \emph{not} a function.
  \end{ex}

  \begin{ex}
    Use the vertical line test to show that $f(x) = x^2 - 4$ is a function.
  \end{ex}

  \section{Domain and Range of Functions}
  \begin{mdframed}
    \begin{defn}
      The \textbf{domain} of a function is the \emph{set} of all values that the function is allowed to take as input. (We usually label
      this set with the capital letter $\mathcal{D}$).
    \end{defn}
  \end{mdframed}
  \begin{ex}
    Let $f(x) = \sqrt{x}$ be a function. Determine the domain of $f$. \textbf{(In class)}.
  \end{ex}
  \begin{ex}
    Let $T(x) = x^2 - 4$ be a function. Determine the domain of $T$. \textbf{(In class)}.
  \end{ex}

  \begin{mdframed}
    \begin{defn}
      The \textbf{range} of a function is the \emph{set} that contains ALL of the output values of the function. (We usually label
      this set with the capital letter $\mathcal{R}$).
    \end{defn}
  \end{mdframed}
  \begin{ex}
    Let $g(x) = x^2$ be a function. Determine the range of $g$. \textbf{(In class)}
  \end{ex}
  \begin{ex}
    Let $r(x) = 2x + 1$ be a function. Determine the range of $r$. \textbf{(In class)}
  \end{ex}

  \begin{ex}
    We will define a function as a class, then we will determine the range of our function.
    \begin{itemize}
      \item Define the domain $\mathcal{D} = \cdots$ 
      \item \textbf{rule of $g$:} For every element $x\in \mathcal{D}$, \ldots 
      \item The range of our function is $\mathcal{R} = \cdots$
    \end{itemize}
  \end{ex}

 \section{Maximum's and Minimums}
 Lets say we have some function,
    \[
          f(x) = a(x - h) ^2 + k
    \] 
 that outputs some quadratic equation in vertex form. Then,
  \begin{itemize}
		\item \textbf{IF $a < 0$ ($a$ is negative)} then we say that the vertex represents a \emph{maximum}.
			\begin{itemize}
				\item This makes sense because if $a < 0$, the parabola points down, and the highest point of the parabola must have been
					the vertex.
			\end{itemize}
		\item \textbf{ELSE IF $a > 0$ ($a$ is positive)} then we say that the vertex represents a \emph{minimum}.
			\begin{itemize}
				\item This makes sense because if $a > 0$, the parabola points up, and the lowest point of the parabola must have been
					the vertex.
			\end{itemize}
\end{itemize}






	









	\end{lec}

\end{document}
%\fixchap{Second Chapter}
%	\begin{figure}[H]
%	\centering
%	\includegraphics[width=0.75\linewidth]{p}
%	\caption{caption.\label{fig:}}
%	\end{figure}
















