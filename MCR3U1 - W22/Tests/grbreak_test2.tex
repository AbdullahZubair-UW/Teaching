% Prepared by Calvin Kent
%
% Assignment Template v19.02
%
%%% 20xx0x/MATHxxx/Crowdmark/Ax
%
\documentclass[12pt]{article} %
\usepackage{amsthm}
\usepackage{CKpreamble}
\usepackage{CKassignment}
\usepackage{mdframed}
\usepackage{euscript}
\usepackage{tikz}
\usepackage{pgfplots}

%
\begin{document}
	\pagenumbering{arabic}
	% Start of class settings ...
	\renewcommand*{\coursecode}{MATH 235} % renew course code
	\renewcommand*{\assgnnumber}{Assignment 1} % renew assignment number
	\renewcommand*{\submdate}{September 14, 2021} % renew the date
	\renewcommand*{\studentfname}{Abdullah} % Student first name
	\renewcommand*{\studentlname}{Zubair} % Student last name
    \renewcommand*{\proofname}{Proof:}
	% \renewcommand*{\studentnum}{20836288} % Student number

	\renewcommand\qedsymbol{$\blacksquare$}
	\setfigpath
	% End of class settings	
	% \pagestyle{crowdmark}
	\newgeometry{left=18mm, right=18mm, top=22mm, bottom=22mm} % page is set to default values
	\fancyhfoffset[L,O]{0pt} % header orientation fixed
	% End of class settings
	%%% Note to user:
	% CTRL + F <CHANGE ME:> (without the angular brackets) in CKpreamble to specify graphics paths accordingly.
	% The command \circled[]{} accepts one optional and one mandatory argument.
	% Optional argument is for the size of the circle and mandatory argument is for its contents.
	% \circled{A} produces circled A, with size drawn for letter A. \circled[TT]{A} produces circled A with size drawn for TT.
	% https://github.com/CalvinKent/My-LaTeX
	%%%

	%%%%%%%%%%%%%%%%%%%%%%%%%%%%%%%%%%%%%%%%%%%%%%%%%%%%%%%%%%%%%%%%%%%%%%%%%%%%%%%
	%%%                        CUSTOM MACRO VIM-TEX                             %%%
	%%       call IMAP('NOM', '\nomenclature{}', 'tex')               

	%%%%%%%%%%%%%%%%%%%%%%%%%%%%%%%%%%%%%%%%%%%%%%%%%%%%%%%%%%%%%%%%%%%%%%%%%%%%%%%

	% Crowdmark assignment start
	% qnumber, qname, points

\begin{center}
	\textbf{\underline{\Huge{Functions Test 2 - Grade Breakdown}}}
\end{center}

\begin{qstn} \text{(10 marks)}
  \begin{itemize}
    \item $1$ Mark for each true or false
  \end{itemize}
\end{qstn}

\begin{qstn} \text{(8 marks)}
  \begin{itemize}
    \item $4$ marks for part (a),
      \begin{itemize}
        \item  $1$ mark for each of mapping diagram, not surjective claim, not invertible b/c not surjective claim.
        \item $1$ mark for range.
      \end{itemize}
    \item $4$ marks for part (b),
      \begin{itemize}
        \item  $1$ mark for each of mapping diagram, not injective claim, not invertible b/c not injective claim.
        \item $1$ mark for range.
      \end{itemize}
  \end{itemize}
\end{qstn}

\begin{qstn} \text{(13 marks)}
  \begin{itemize}
    \item $5$ marks for part (a)
      \begin{itemize}
        \item  $1$ mark for each correct output.
        \item $1$ mark for domain formalization.
      \end{itemize}
    \item $3$ marks for part (b) for each step in determining original function.
    \item $5$ marks for part (c),
      \begin{itemize}
        \item $4$ marks in correct outputs using formula from part (b).
        \item $1$ mark for final concluding statement.
      \end{itemize}
  \end{itemize}
\end{qstn}

\begin{qstn} \text{(7 marks)}
  \begin{itemize}
    \item $6$ marks for standard proof procedure, outputs in table and final statement.
      \begin{itemize}
        \item Watch out for correctness errors in table and labelling.
      \end{itemize}
    \item $1$ for final concluding statement.
  \end{itemize}
\end{qstn}

\begin{qstn} $\text{(10 marks)}$
   \begin{itemize}
     \item $5$ marks for part (a) for each step that is necessary.
     \item $5$ marks for part (b) for each step that is necessary.
   \end{itemize}
  
\end{qstn}

\begin{qstn} $\text{(18 marks)}$
   \begin{itemize}
     \item $5$ marks for part (a) for each of the five transformations described.
     \item $2$ marks for part (b) for each x,y coordinate simplification.
     \item $5$ marks for part (c) for each of the five coordinate transformations.
     \item $6$ marks for part (d)
      \begin{itemize}
        \item $5$ marks for each labelled coordinate.
        \item $1$ marks for overall correctness.
      \end{itemize}
   \end{itemize}
  
\end{qstn}

\begin{qstn}
  asf
\end{qstn}

\begin{qstn} $\text{(18 marks)}$
   \begin{itemize}
     \item $6$ marks for part (a) for each correct path string.
     \item $3$ marks for part (b) for overall correctness,
      \begin{itemize}
        \item Essential part of connecting the positions of $U$ or $R$ is key.
      \end{itemize}
     \item $6$ marks for part (c) for correct mapping diagram for each input.
     \item $3$ marks for part (d) for overall correctness,
   \end{itemize}
  
\end{qstn}



\end{document}



























