% Prepared by Calvin Kent
%
% Assignment Template v19.02
%
%%% 20xx0x/MATHxxx/Crowdmark/Ax
%
\documentclass[12pt]{article} %
\usepackage{CKpreamble}
\usepackage{CKassignment}
\usepackage{tkz-euclide}
\usepackage{physunits}
\usepackage{physics}
\usepackage{mathtools}
\usepackage{multicol}
\usepackage{lmodern}
\usepackage{tikz}
\usepackage{pgfplots}
\usepackage{pdfpages}
\usepackage{euscript}
\usepackage{transparent}
\usepackage{xcolor}
\usepackage{tasks}
\usepackage{tkz-euclide}

\usepackage{pgfplots}
\usepgfplotslibrary{polar}
\usepgflibrary{shapes.geometric}
\usetikzlibrary{calc}


\usepackage{euscript}
\usepackage{microtype}
\usepackage{upgreek}
\usepackage[misc]{ifsym}


%%Title
\title{\textbf{Assignment 3 Functions} \\ \textbf{Due Date: } Tuesday, Feburary 1}
\date{}

%%% Maths and science packages

\usepackage{amsmath,amsthm,amssymb}
\usepackage{pgfplots}
	\usetikzlibrary{
		calc,
		patterns,
		positioning
	}
	\pgfplotsset{
		compat=1.16,
		samples=200,
		clip=false,
		my axis style/.style={
			axis x line=middle,
			axis y line=middle,
			legend pos=outer north east,
			axis line style={
				->,
			},
			legend style={
				font=\footnotesize
			},
			label style={
				font=\footnotesize
			},
			tick label style={
				font=\footnotesize
			},
			xlabel style={
				at={
					(ticklabel* cs:1)
				},
				anchor=west,
				font=\footnotesize,
			},
			ylabel style={
				at={
					(ticklabel* cs:1)
				},
				anchor=west,
				font=\footnotesize,
			},
			xlabel= $x$,
			ylabel=$\vec d (\m \tx{[East]})$
		},
	}
	\tikzset{
		>=stealth
	}

\pgfplotsset{my style/.append style={axis x line=middle, axis y line=
middle, xlabel={$t$}, ylabel={$y[\text{m}]$}, axis equal }}

%%% Tables and figures packages

\usepackage{float}
\usepackage{caption}
	\captionsetup{
		format=plain,
		labelfont=bf,
		font=small,
		justification=centering
	}
	
%%% Numbers and sets

\newcommand{\E}{\mathrm{e}}

\newcommand{\tx}[1]{\text{#1}}
\newcommand{\rem}[1]{\operatorname{rem}{(#1)}}


%
\begin{document}
	\pagenumbering{arabic}
	% Start of class settings ...
	\renewcommand*{\coursecode}{MATH 235} % renew course code
	\renewcommand*{\assgnnumber}{Assignment 1} % renew assignment number
	\renewcommand*{\submdate}{September 14, 2021} % renew the date
	\renewcommand*{\studentfname}{Abdullah} % Student first name
	\renewcommand*{\studentlname}{Zubair} % Student last name
    \renewcommand*{\proofname}{Proof:}
	% \renewcommand*{\studentnum}{20836288} % Student number

	\renewcommand\qedsymbol{$\blacksquare$}
	\setfigpath
	% End of class settings	
	% \pagestyle{crowdmark}
	\newgeometry{left=18mm, right=18mm, top=22mm, bottom=22mm} % page is set to default values
	\fancyhfoffset[L,O]{0pt} % header orientation fixed
	% End of class settings
	%%% Note to user:
	% CTRL + F <CHANGE ME:> (without the angular brackets) in CKpreamble to specify graphics paths accordingly.
	% The command \circled[]{} accepts one optional and one mandatory argument.
	% Optional argument is for the size of the circle and mandatory argument is for its contents.
	% \circled{A} produces circled A, with size drawn for letter A. \circled[TT]{A} produces circled A with size drawn for TT.
	% https://github.com/CalvinKent/My-LaTeX
	%%%

	%%%%%%%%%%%%%%%%%%%%%%%%%%%%%%%%%%%%%%%%%%%%%%%%%%%%%%%%%%%%%%%%%%%%%%%%%%%%%%%
	%%%                        CUSTOM MACRO VIM-TEX                             %%%
	%%       call IMAP('NOM', '\nomenclature{}', 'tex')               

	%%%%%%%%%%%%%%%%%%%%%%%%%%%%%%%%%%%%%%%%%%%%%%%%%%%%%%%%%%%%%%%%%%%%%%%%%%%%%%%

	% Crowdmark assignment start
	% qnumber, qname, qpoints
\maketitle
	\section{Preamble}
  This assignment covers everything taught so far. The solutions that you hand in should be \textbf{neat} and \textbf{legible},
  this is an assignment, not a quiz, so I expect you to take your time and present thorough and detailed solutions.

  This assignment has $\textbf{(REPLACE)}$ questions. \textbf{Start early}.
  \section{Replacement} 

  If your mark on this assignment is better than the last assignment, then I will replace the old assignment with
  this one. This means that your final assignment mark will consist of two assignment marks, either A1, A3 or A1,
  A2.

  \section{Bonus}
  If you type this assignment in \LaTeX, I will give you a bonus 7\%.
  
\section{Name and Date:}
	Print your name and todays date below;\\


	\begin{center}
	\noindent\begin{tabular}{ll}
		\makebox[3in]{\hrulefill} & \makebox[3in]{\hrulefill}\\
		Name & Date\\[8ex]% adds space between the two sets of signatures
	\end{tabular}
	\end{center}
	\newpage

  \begin{qstn}
    State in your own words, what does it mean for a number to be prime. What types of numbers can be primes? Why
    do we care about primes?
  \end{qstn}

  \begin{qstn}
    Let $x \in \Z$ and let $p$ be a prime.
    \begin{enumerate}[label=(\alph*)]
      \item Determine all possible values of $\gcd (x,p)$.
      \item Determine all possible values of $\gcd (x^2,p)$.
      \item Determine all possible values of $\gcd (x,p^2)$.
      \item Determine all possible values of $\gcd (x^2,p^2)$.
      \item Determine all possible values of $\gcd (p^2,p)$.
    \end{enumerate}
  \end{qstn}

  \begin{qstn}
    Let $p$ be a prime number. Determine the value of $ \operatorname{rem}(p,2)$ \textbf{and} explain how you got
    your answer.
  \end{qstn}
  
  \begin{qstn}
    This question is meant for review as it will definitely appear on the Final. Let $f(x) = \sqrt{x} $, 
    and let $R(x) = -2f(\frac{1}{2}x + 1) + 2$ be a transformation of $f$. 
    \begin{enumerate}[label=(\alph*)]
      \item Describe the transformation. \textbf{(Remember to Factor First)}

      \item Determine the expression for the coordinate transformation,
        \[
            \left( \frac{x- H}{B}, Af(x) + K \right)
        \] 

      \item Complete a coordinate table to determine the corresponding transformed coordinates using the following
        base coordinates,
        \[
            (0,0) \hspace*{2cm} (1,1) \hspace*{2cm} (4,2) \hspace*{2cm} (9,3) \hspace*{2cm} (16,4)
        .\] 

     \item Using your results from the coordinate table, sketch the transformation $R(x)$ \textbf{on your axis
       sheet}. Be sure to \textbf{label} the transformed coordinates as well as the function.
    \end{enumerate}

  \end{qstn}


  \begin{qstn}
    Preform a prime factorization of the following natural numbers.\\
    \textbf{Note:} If the number is a prime itself, then state that it is.
    \begin{enumerate}[label=(\alph*)]
      \item 3634.
      \item 555.
      \item 663.
      \item 991.
    \end{enumerate}
  \end{qstn}


  \begin{qstn}
    \textbf{Fully }simplify the following exponential expressions.\\ \textbf{(Leave answers with positive exponents)}
    \begin{enumerate}[label=(\alph*)]
      \item $-x^2(-x)^2x^{-3}$
                      \newpage
      \item $\left( x^{-4} \right) / \left( y^{2} \right)^{-3}  $.
      \item $\left( 4^{-1}y^2z^{-3}x^{8}x^{-3} \right)^{-3}y^{-6}z^{9}$.
      \item $\left( 81a^{3}b^{2}z^{-6} \right)^{-2}  / \left( 3a^{9}bz^{-4} \right)^{-3}$.
      \item $\left( 16 \right)^{\frac{3}{2}}\left( 9 \right)^{\frac{3}{2}} \left( 4\right) ^{\frac{-5}{2}} $.
    \end{enumerate}
  \end{qstn}
    
  \begin{qstn}
    \textbf{Textbook, Pg. 39\,\,\, Q1. a),c),e)}
  \end{qstn}

  \begin{qstn}
    \textbf{Textbook, Pg. 39\,\,\, Q3. a),c),e)}
  \end{qstn}

  \begin{qstn}
    \textbf{Textbook, Pg. 39\,\,\, Q4. a),c),e)}
  \end{qstn}

  \begin{qstn}
    \textbf{Textbook, Pg. 39\,\,\, Q5. a),c),e)}
  \end{qstn}
    
  \begin{qstn}
    \textbf{Textbook, Pg. 39\,\,\, Q7. a),c),e)}
  \end{qstn}

  \begin{qstn}
    \textbf{Textbook, Pg. 94\,\,\, Q4. a),c)}
  \end{qstn}

  \begin{qstn}
    \textbf{Textbook, Pg. 94\,\,\, Q6. c),d)}
  \end{qstn}

  \begin{qstn}
    \textbf{Textbook, Pg. 94\,\,\, Q6. b),c)}
  \end{qstn}

  \begin{qstn}
    You are given $\triangle SPQ$ where $SP = 13$,  $\angle PQS = 45^{\circ}$ and $\cos \beta = \sqrt{3} / 2$. Determine the
        \textbf{exact }area of  $\triangle SPQ$.
        \begin{center}
          \begin{tikzpicture}[thick]
          \coordinate (S) at (0,0);
          \coordinate (P) at (6,4);
          \coordinate (Q) at (15,0);
          \coordinate (R) at (6,0);

          \draw[black] (S) -- (Q) node[circle,fill,inner sep=2pt]{} node[above]{$Q$};
          \draw[black] (S) -- (P) node[circle,black,fill,inner sep=2pt]{} node[above]{$P$} node[midway, above]{$13$};
          \draw[black] (P) -- (Q) node[circle,black,fill,inner sep=2pt]{};
          \draw[dashed] (P) -- (R) node[circle,black,fill,inner sep=2pt]{};
          \draw[black] (S) node[circle,black,fill,inner sep=2pt]{} node[above]{$S$};
          \draw[black] (R) node[circle,black,fill,inner sep=2pt]{} node[below]{$R$};

          \tkzMarkAngle[fill= orange,size=1cm,%
          opacity=1](S,P,R)
          \tkzLabelAngle[pos = 1.4](S,P,R){$\beta$}

          \tkzMarkAngle[fill= orange,size=1.4cm,%
          opacity=1](P,Q,S)
          \tkzLabelAngle[pos = 1.8](P,Q,S){$45^{\circ}$}

          \tkzMarkRightAngle[size=0.4,opacity=0.9](S,R,P)% square angle here

          \end{tikzpicture}
      \end{center}
  \end{qstn}

  \begin{qstn}
    Describe in your own words what polar coordinates are? Why are they useful and what advantages do they provide
    as a metric?
  \end{qstn}

  \begin{qstn}
    Convert (a), (b) to polar coordinates and (c), (d) to standard coordinates.
    \begin{multicols}{4}
      \begin{enumerate}[label=(\alph*)]
        \item $\vb P(-3,4)$.
          \columnbreak
        \item $\vb R(-1,-3)$.
          \columnbreak
        \item $\vb Q(4,320 ^{\circ})$.
          \columnbreak
        \item $\vb T(9,30^{\circ})$.
      \end{enumerate}
      
    \end{multicols}

  \end{qstn}

  \begin{qstn}
    Determine the six trigonometric ratios for the following angles,
    \begin{multicols}{4}
      \begin{enumerate}[label=(\alph*)]
        \item $\theta_1 = 60^{\circ}$
          \columnbreak
        \item $\theta_2 = 220^{\circ}$
          \columnbreak
        \item $\theta_3 = -240^{\circ}$
          \columnbreak
        \item $\theta_4 = 330^{\circ}$
      \end{enumerate}
      
    \end{multicols}

  \end{qstn}













\end{document}



































