% Prepared by Calvin Kent
%
% Assignment Template v19.02
%
%%% 20xx0x/MATHxxx/Crowdmark/Ax
%
\documentclass[12pt]{article} %
\usepackage{amsthm}
\usepackage{CKpreamble}
\usepackage{CKassignment}
\usepackage{mdframed}
\usepackage{euscript}
\usepackage{tikz}
\usepackage{pgfplots}

%
\begin{document}
	\pagenumbering{arabic}
	% Start of class settings ...
	\renewcommand*{\coursecode}{MATH 235} % renew course code
	\renewcommand*{\assgnnumber}{Assignment 1} % renew assignment number
	\renewcommand*{\submdate}{September 14, 2021} % renew the date
	\renewcommand*{\studentfname}{Abdullah} % Student first name
	\renewcommand*{\studentlname}{Zubair} % Student last name
    \renewcommand*{\proofname}{Proof:}
	% \renewcommand*{\studentnum}{20836288} % Student number

	\renewcommand\qedsymbol{$\blacksquare$}
	\setfigpath
	% End of class settings	
	% \pagestyle{crowdmark}
	\newgeometry{left=18mm, right=18mm, top=22mm, bottom=22mm} % page is set to default values
	\fancyhfoffset[L,O]{0pt} % header orientation fixed
	% End of class settings
	%%% Note to user:
	% CTRL + F <CHANGE ME:> (without the angular brackets) in CKpreamble to specify graphics paths accordingly.
	% The command \circled[]{} accepts one optional and one mandatory argument.
	% Optional argument is for the size of the circle and mandatory argument is for its contents.
	% \circled{A} produces circled A, with size drawn for letter A. \circled[TT]{A} produces circled A with size drawn for TT.
	% https://github.com/CalvinKent/My-LaTeX
	%%%

	%%%%%%%%%%%%%%%%%%%%%%%%%%%%%%%%%%%%%%%%%%%%%%%%%%%%%%%%%%%%%%%%%%%%%%%%%%%%%%%
	%%%                        CUSTOM MACRO VIM-TEX                             %%%
	%%       call IMAP('NOM', '\nomenclature{}', 'tex')               

	%%%%%%%%%%%%%%%%%%%%%%%%%%%%%%%%%%%%%%%%%%%%%%%%%%%%%%%%%%%%%%%%%%%%%%%%%%%%%%%

	% Crowdmark assignment start
	% qnumber, qname, points

\begin{center}
	\textbf{\underline{\Huge{Functions Exam - Grade Breakdown}}}
\end{center}

\section*{Part A - (16 Marks)}

\begin{qstn}[16 Marks] \texttt{  }
  \begin{itemize}
    \item $1$ Mark for each true or false
  \end{itemize}
\end{qstn}

\section*{Part B - (8 Marks)}

\textbf{Question 2 - 5} (8 Marks)
\begin{itemize}
  \item 2 Marks per Question, look for generally a good explanation.
\end{itemize}

\section*{Part C - (64 Marks)}
\textbf{Question 6} (6 Marks)
\begin{multicols}{4}
  \begin{enumerate}[label=(\alph*)]
    \item 1 Mark
      \columnbreak
    \item 1 Mark
      \columnbreak
    \item 2 Marks
      \columnbreak
    \item 2 Marks
  \end{enumerate}
\end{multicols}

\textbf{Question 7} (8 Marks)
\begin{multicols}{3}
  \begin{enumerate}[label=(\alph*)]
    \item 2 Mark
      \columnbreak
    \item 4 Marks, an extra mark for overall correctness.
      \columnbreak
    \item 2 Marks
  \end{enumerate}
\end{multicols}

\textbf{Question 8} (10 Marks)
\begin{multicols}{2}
  \begin{enumerate}[label=(\alph*)]
    \item 5 Marks
      \columnbreak
    \item 5 Marks
  \end{enumerate}
\end{multicols}

\textbf{Question 9} (4 Marks)

\textbf{Question 10} (4 Marks)

\textbf{Question 11} (10 Marks)
\begin{multicols}{2}
  \begin{enumerate}[label=(\alph*)]
    \item 5 Marks
      \columnbreak
    \item 5 Marks, extra mark for FOIL.
  \end{enumerate}
\end{multicols}


\textbf{Question 12} (14 Marks)
\begin{enumerate}[label=(\alph*)]
  \item 6 Marks.
    \begin{itemize}
      \item 4 Marks for each of the correct factoring.
      \item 2 Marks for cancelling and Final answer.
    \end{itemize}

    \newpage

  \item 8 Marks.
    \begin{itemize}
      \item 2 Marks for Factoring Each Denominator.
      \item 2 Marks for LCD.
      \item 2 Marks for Missing Factor.
      \item 2 Marks for overall correctness and Final answer.
    \end{itemize}
\end{enumerate}


\textbf{Question 13} (8 Marks)
\begin{multicols}{2}
  \begin{enumerate}[label=(\alph*)]
    \item 4 Marks.
      \columnbreak
    \item 4 Marks.
  \end{enumerate}
\end{multicols}

\section*{Part D - (12 Marks)}

\textbf{Question 14} (12 Marks)
\begin{enumerate}[label=(\alph*)]
  \item 12 Marks.
    \begin{itemize}
      \item 4 Marks for each correct coordinate for $\vb R$.\\ (Check for correct reference angle, sign and trig
        ratio, and then final coordinate).
      \item 4 Marks for each correct coordinate for $\vb T$.\\ (Check for correct reference angle, sign and trig
        ratio, and then final coordinate).
      \item 4 Marks for formula, simplification and final answer.
    \end{itemize}

  \item 12 Marks.
    \begin{itemize}
      \item Just look for overall correct procedure.
    \end{itemize}
\end{enumerate}

\end{document}



























