% Prepared by Calvin Kent
%
% Lecture Template v19.02
%
%%% 201901/MATHxxx/Notes
%
\documentclass[12pt,oneside]{book} %
\usepackage{CKpreamble}
\usepackage{CKlecture}
\usepackage{mdframed}
\usepackage{import}
\usepackage{pdfpages}
\usepackage{transparent}
\usepackage{xcolor}

\newcommand{\incfig}[2][1]{%
    \def\svgwidth{#1\columnwidth}
    \import{./figures/}{#2.pdf_tex}
}

\pdfsuppresswarningpagegroup=1

%
\renewcommand*{\doctitle}{Class Based Lecture Notes}
\makeatletter\patchcmd{\chapter}{\if@openright\cleardoublepage\else\clearpage\fi}{}{}{}\makeatother % only used in class based
\begin{document}
	% Start of Class settings
	\renewcommand*{\term}{Term 2} % Term
	\renewcommand*{\coursecode}{MCR3U} % Course code
	\renewcommand*{\coursename}{Course Name} % Full course name
	\renewcommand*{\profname}{Prof Name} % Prof Name
	\renewcommand*{\colink}{http://www.student.math.uwaterloo.ca/~c2kent} % Course outline link
	% End of Class settings
	\clearpage
	\pagenumbering{arabic}
	\pagestyle{classlecture}
	%%% Note to user: CTRL + F <CHANGE ME:> (without the angular brackets) in CKpreamble to specify graphics paths accordingly.
	%%% If a new chapter was started in the middle of a lecture, \fixchap{Second Chapter} must be used immediately above the next lecture.
	% Course notes start
\setchap{1}{Introduction to Sets}
\begin{lec}{November 2021}
	\chapter{\chapname\chaplec}
	\begin{mdframed}
		\begin{defn}
		\emph{Sets} are defined to be a collection of objects in a pair of curly braces $\{\}$.
		\end{defn}
	\end{mdframed}
	When we say objects, we are usually referring to numbers, however objects can be any other piece of information as well
		(Letters, symbols, etc).

		We usually assign a captial letter to represent a set.
		\begin{ex} Here are some examples of sets.
			\begin{itemize}
			    \item $A = \{1,2,3,4\}$
			    \item $B = \{\clubsuit, \heartsuit, \triangle\}$
			    \item $S =  \{A,B,C,D,E\}$
			    \item $\emptyset$
	    \end{itemize}
    \end{ex}
		(Class examples)

		\begin{notn} Lets say we have a set $S = \{2,3,4\}$.
			\begin{itemize}
				\item  Then we say that $2,3,4$ are \emph{elements} of the set $S$, and we may write this
			symbolically as $2 \in S$, where '$\in$' means 'element of'.
		\item Is $5$ an element of $S$? It isn't, and hence we say that $5$ is \emph{not} an element of $S$, and we may write this
			symbolically as $5\not\in S$, where '$\not\in$' means 'not an element of'.
		\end{itemize}
			
		(Class examples)
			\end{notn}

		\begin{rem}
			When we are dealing with sets, there are two things to keep in mind.
				\begin{itemize}
					\item Order does not matter! (Class Example)
					\item Duplicate elements are \emph{always} deleted. (Class Example)
					\end{itemize}
			\end{rem}

			\begin{mdframed}
				\begin{defn}
			The \emph{empty set} is the set that contains nothing. We label it $\emptyset$.
				\end{defn}
			\end{mdframed}

			\vspace*{0.2cm}

			\begin{mdframed}
				\begin{defn} Important Sets.\\
					There a few important sets that we will see throughout the course.
					\begin{enumerate}
					\item $\mathbb Z$ denotes the set of all integers $\mathbb Z = \{\dots,-2,-1,0,1,2,\dots \}$.
					\item $\mathbb Q$ denotes the set of all rational numbers.\\ (A rational number is any number that can be written as a fraction
						of two integers).
					\item $\mathbb R$ denotes the set of all real numbers (rational or irrational).
					\end{enumerate}
								
					\end{defn}
			\end{mdframed}
			
			\newpage
	\begin{notn}
			Dots.

			We often use dots in sets to indicate that the obvious pattern continues on forever. For example, what to do you think that
			the set $S = \{1,2,3,4,5,\dots\}$ will look like?
			\end{notn}

		\begin{notn} Describing sets.

			The most common way we will describe a set is to give a rule which must be obeyed by \emph{all} elements of the set. For
			example, we may say that all elements must be even, or that all elements must be greater than four, or we may say that all
			elements must be between one and thirteen, etc. This way we can come up with more unique sets.

			For example, lets write down the set of all integers between one and four, label the set $S$.
																$$S = \{1,2,3,4\}$$
			For this set, the rule is that all integers are allowed such that they are between one and four. We write the previous
			sentence symbolically as,
										$$S = \{x\in \Z \mid 1 \leq x \leq 4\}$$
			The bar '$\mid$' means 'such that'. Here '$x$' acts as a placeholder that represents all integers that satisfy the condition
			that they are between one and four, or in other words, $1\leq x\leq 4$.
			\end{notn}

		\begin{ex}
			Describe the following sets symbolically,
			\begin{enumerate}
				\item All the real numbers that are greater than or equal to $4$
				\item All the real numbers that are not zero.
				\item All the integers that are less than $-1$.
				\item All the real numbers that are between $1$ and $6$.
				\item \emph{(Bonus)}All of the positive integers (0,1,2,3,$\dots$).
					\end{enumerate}
			\end{ex}

			\begin{ex}
				Write down the elements of the following sets.
				\begin{enumerate}
					\item $A = \{y \in \Z \mid y \geq 1\}$.
					\item $B = \{x \in \R \mid x = 1\}$.
					\item $T = \{y \in \Z \mid -1 \leq y \leq  2\}$.
					\item $S = \{z \in \Z \mid \text{$z$ is even}\}$.
						\end{enumerate}
				\end{ex}
				\newpage
				\section{Visualizing Sets}
				Suppose that we have some set $S = \{2,4,6,7\}$. Sometimes we can think of sets using the following diagram,
				\begin{figure}[ht]
			    \centering
					\incfig[0.3]{riemmans-theorem}
					\label{fig:riemmans-theorem}
				\end{figure}\\
				This will be a useful representation of sets when we begin to study functions.
	\end{lec}

\end{document}
%\fixchap{Second Chapter}
%	\begin{figure}[H]
%	\centering
%	\includegraphics[width=0.75\linewidth]{p}
%	\caption{caption.\label{fig:}}
%	\end{figure}


































