\documentclass[12pt,oneside]{book} %
\usepackage{CKpreamble}
\usepackage{CKlecture}
\usepackage{mdframed}
\usepackage{import}
\usepackage{pdfpages}
\usepackage{transparent}
\usepackage{xcolor}

\newcommand{\incfig}[2][1]{%
    \def\svgwidth{#1\columnwidth}
    \import{./figures/}{#2.pdf_tex}
}

\pdfsuppresswarningpagegroup=1

%
\renewcommand*{\doctitle}{Class Based Lecture Notes}
\makeatletter\patchcmd{\chapter}{\if@openright\cleardoublepage\else\clearpage\fi}{}{}{}\makeatother % only used in class based
\begin{document}
	% Start of Class settings
	\renewcommand*{\term}{Term 2} % Term
	\renewcommand*{\coursecode}{MCR3U} % Course code
	\renewcommand*{\coursename}{Course Name} % Full course name
	\renewcommand*{\thelecnum}{3} % Lecture number
	\renewcommand*{\profname}{Prof Name} % Prof Name
	\renewcommand*{\colink}{http://www.student.math.uwaterloo.ca/~c2kent} % Course outline link
	% End of Class settings
	\clearpage
	\pagenumbering{arabic}
	\pagestyle{classlecture}
	%%% Note to user: CTRL + F <CHANGE ME:> (without the angular brackets) in CKpreamble to specify graphics paths accordingly.
	%%% If a new chapter was started in the middle of a lecture, \fixchap{Second Chapter} must be used immediately above the next lecture.
	% Course notes start
\setchap{3}{Inverse Functions Part 1}
\begin{lec}{December 2021}
	\chapter{\chapname\chaplec}
In this lesson, we will explore the notion of an invertible function. This is of particular importance since invertible functions
tell us more about the domain, the co-domain and correspondences between the two. In particular, we can reconstruct 
the entire domain based strictly on knowledge of the co-domain.

\begin{defn}
    Let $\mathcal{A}$ and $\mathcal{B}$ be sets. The \textbf{identity function} on the set $\mathcal{A}$ is the function defined as,
    \begin{align*}
      \id_\mathcal{A} \colon \mathcal{A} &\rightarrow \mathcal{B}\\
      \id_\mathcal{A}(a) &= a
    \end{align*}
    This is essentially the function that takes each element in $\mathcal{A}$ and returns it.
\end{defn}
The identity function is essentially the most trivial function, lets see how it works in the following example.

\begin{ex}
  Let $\mathcal{H} = \{4,6,7,10,12\}$ and $\mathcal{T} = \{1,4,5,7,8,10,12,6\}$ be sets. Draw the mapping diagram for
  the identity function on $\mathcal{H}$, i.e $\id_\mathcal{H} \colon \mathcal{H} \to \mathcal{T}$. \textbf{(In class)}
\end{ex}

\begin{defn}
    Let $\mathcal{A}$ and $\mathcal{B}$ be sets. Let $f \colon \mathcal{A} \to \mathcal{B}$ be some function. We say that $f$ is
    \textbf{surjective} if every element in $\mathcal{B}$ is mapped to.
\end{defn}

\begin{ex}
  Let $ \mathcal{A} = \{2,3,4,5,6\} $ and $ \mathcal{B} = \{0,1\} $ be sets, lets define the
  following function, 
  \begin{itemize}
    \item $\mathcal{R} \colon \mathcal{A} \to \mathcal{B}$.
    \item $\mathcal{R}(a) = \operatorname{rem}(a,2)$.
  \end{itemize}
  I claim that $\mathcal{R}$ is surjective. \textbf{(Explanation in Class)}
\end{ex}

\begin{ex}
  Notice that the function defined in Example 3.1 is \textbf{not} surjective. \textbf{(Explanation in Class)}
\end{ex}

\begin{defn}
    Let $\mathcal{A}$ and $\mathcal{B}$ be sets. Let $f \colon \mathcal{A} \to \mathcal{B}$ be some function. We say that $f$ is
    \textbf{injective} if no two elements in $\mathcal{A}$ map to a single element in $\mathcal{B}$.
\end{defn}

\begin{ex}
  Notice that the function defined in Example 3.2 is \textbf{not} injective. \textbf{(Explanation in Class)}
\end{ex}

\newpage

\begin{ex}
  Let $ \mathcal{A} = \{0,1,2\} $ and $ \mathcal{B} = \{1,2,3,25,36\} $ be sets, lets define the
  following function, 
  \begin{itemize}
    \item $f \colon \mathcal{A} \to \mathcal{B}$.
    \item $f(a) = a + 1$.
  \end{itemize}
  I claim that the function $f$ is injective . \textbf{(In class explanation)}
\end{ex}

We are now ready to define invertible functions

\begin{defn}
    Let $\mathcal{A}$ and $\mathcal{B}$ be sets. Let $f \colon \mathcal{A} \to \mathcal{B}$ be some function. If $f$ is both
    injective and surjective, then we say $f$ is \textbf{invertible} and an inverse function for $f$ exists.
\end{defn}


\begin{ex}
  Let $ \mathcal{A} = \{1,2,3,5,18\} $ and $ \mathcal{B} = \{2,4,6,10,36\} $ be sets, lets define the
  following function, 
  \begin{itemize}
    \item $\mathcal{L} \colon \mathcal{A} \to \mathcal{B}$.
    \item $\mathcal{L}(a) = 2a$.
  \end{itemize}
  I claim that $\mathcal{L}$ is invertible. \textbf{(Explanation in Class)}
\end{ex}

\begin{ex}
  Let $ \mathcal{A} = \{-2,-1,0,1,2,3\} $ and $ \mathcal{B} = \{0,1,4,9,25,36\} $ be sets, lets define the
  following function, 
  \begin{itemize}
    \item $f \colon \mathcal{A} \to \mathcal{B}$.
    \item $f(a) = a^2$.
  \end{itemize}
  I claim that the function $f$ is \textbf{not} invertible. \textbf{(In class explanation)}
\end{ex}

\begin{question}
  Let $\mathcal{A}$, $\mathcal{B}$ be sets, and let $f \colon  \mathcal{A} \to \mathcal{B}$ be a function. Suppose that $f$ is
  \textbf{surjective}, then is it true that the range is equal to $\mathcal{B}$? In other words, is $\mathcal{R}_f = \mathcal{B}$?
\end{question}

\begin{answer}
  Yes ! \textbf{(In class)}
\end{answer}


\begin{question}
  Let $\mathcal{A}$, $\mathcal{B}$ be sets, and let $f \colon  \mathcal{A} \to \mathcal{B}$ be an \textbf{invertible} function.
  Then is it true that $ \left|\mathcal{A}\right| = \left|\mathcal{B}\right|$? \textbf{(Class Question)}
\end{question}

\begin{answer}
  Yes ! \textbf{(In class)}
\end{answer}























	\end{lec}

\end{document}
%\fixchap{Second Chapter}
%	\begin{figure}[H]
%	\centering
%	\includegraphics[width=0.75\linewidth]{p}
%	\caption{caption.\label{fig:}}
%	\end{figure}
