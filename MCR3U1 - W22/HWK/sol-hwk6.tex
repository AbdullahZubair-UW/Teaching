% Prepared by Calvin Kent
%
% Assignment Template v19.02
%
%%% 20xx0x/MATHxxx/Crowdmark/Ax
%
\documentclass[12pt]{article} %
\usepackage{amsthm}
\usepackage{CKpreamble}
\usepackage{CKassignment}
\usepackage{mdframed}
\usepackage{import}
\usepackage{pdfpages}
\usepackage{transparent}
\usepackage{xcolor}
\usepackage{tkz-euclide}
\usepackage{physunits}
\usepackage{physics}
\usepackage{lmodern}
\usepackage{microtype}
\usepackage{euscript}
\usepackage{upgreek}
\usepackage[misc]{ifsym}

%%% Maths and science packages

\usepackage{amsmath,amsthm,amssymb}
\usepackage{pgfplots}
	\usetikzlibrary{
		calc,
		patterns,
		positioning
	}
	\pgfplotsset{
		compat=1.16,
		samples=200,
		clip=false,
		my axis style/.style={
			axis x line=middle,
			axis y line=middle,
			legend pos=outer north east,
			axis line style={
				->,
			},
			legend style={
				font=\footnotesize
			},
			label style={
				font=\footnotesize
			},
			tick label style={
				font=\footnotesize
			},
			xlabel style={
				at={
					(ticklabel* cs:1)
				},
				anchor=west,
				font=\footnotesize,
			},
			ylabel style={
				at={
					(ticklabel* cs:1)
				},
				anchor=west,
				font=\footnotesize,
			},
			xlabel= $x$,
			ylabel=$\vec d (\m \tx{[East]})$
		},
	}
	\tikzset{
		>=stealth
	}

%%% Tables and figures packages

\usepackage{float}
\usepackage{caption}
	\captionsetup{
		format=plain,
		labelfont=bf,
		font=small,
		justification=centering
	}
	


\theoremstyle{ex}
\newtheorem*{ex}{Example}

\newcommand{\incfig}[2][1]{%
    \def\svgwidth{#1\columnwidth}
    \import{./figures/}{#2.pdf_tex}
}

\pdfsuppresswarningpagegroup=1

\newcounter{step}[section]
\newenvironment{step}[1][]
{\refstepcounter{step} \textbf{Step #1.}}


%
\begin{document}
	\pagenumbering{arabic}
	% Start of class settings ...
	\renewcommand*{\coursecode}{MATH 235} % renew course code
	\renewcommand*{\assgnnumber}{Assignment 1} % renew assignment number
	\renewcommand*{\submdate}{September 14, 2021} % renew the date
	\renewcommand*{\studentfname}{Abdullah} % Student first name
	\renewcommand*{\studentlname}{Zubair} % Student last name
    \renewcommand*{\proofname}{Proof:}
	% \renewcommand*{\studentnum}{20836288} % Student number

	\renewcommand\qedsymbol{$\blacksquare$}
	\setfigpath
	% End of class settings	
	% \pagestyle{crowdmark}
	\newgeometry{left=18mm, right=18mm, top=22mm, bottom=22mm} % page is set to default values
	\fancyhfoffset[L,O]{0pt} % header orientation fixed
	% End of class settings
	%%% Note to user:
	% CTRL + F <CHANGE ME:> (without the angular brackets) in CKpreamble to specify graphics paths accordingly.
	% The command \circled[]{} accepts one optional and one mandatory argument.
	% Optional argument is for the size of the circle and mandatory argument is for its contents.
	% \circled{A} produces circled A, with size drawn for letter A. \circled[TT]{A} produces circled A with size drawn for TT.
	% https://github.com/CalvinKent/My-LaTeX
	%%%

	%%%%%%%%%%%%%%%%%%%%%%%%%%%%%%%%%%%%%%%%%%%%%%%%%%%%%%%%%%%%%%%%%%%%%%%%%%%%%%%
	%%%                        CUSTOM MACRO VIM-TEX                             %%%
	%%       call IMAP('NOM', '\nomenclature{}', 'tex')               

	%%%%%%%%%%%%%%%%%%%%%%%%%%%%%%%%%%%%%%%%%%%%%%%%%%%%%%%%%%%%%%%%%%%%%%%%%%%%%%%

	% Crowdmark assignment start
	% qnumber, qname, qpoints

\begin{center}
		\Huge{\underline{\textbf{Solutions - Lecture 6 - Homework}}}
\end{center}
\begin{qstn}
  \begin{solution}
    The only divisors of $p$ are $1$ and $p$, since its prime. Hence if $p$ divides $x$, then the greatest common
    divisor in $p$, else the greatest common divisor is $1$. Thus the only possibilities are $1,p$.
  \end{solution}
\end{qstn}

\begin{qstn}
  \begin{solution}
    433, 79.
  \end{solution}
\end{qstn}

\begin{qstn}
  \begin{solution} \texttt{  }
    \begin{enumerate}[label=(\alph*)]
      \item $363 = 3\cdot 11\cdot 11 = 3\cdot 11^2$.
      \item $237 = 3 \cdot 79$.
      \item $688 = 2\cdot 2\cdot 2\cdot 2\cdot 43 = 2^{4}\cdot 43$.
      \item $732 = 2\cdot 2\cdot \cdot 3\cdot 61 = 2^2\cdot 3\cdot 61$.
    \end{enumerate}
  \end{solution}
\end{qstn}

\begin{qstn}
  \begin{solution} \texttt{  }
    \begin{enumerate}[label=(\alph*)]
      \item  $1 / (x^{4}y^{10})$.
      \item $(x^{18}y^{21}) / z^{6}$.
      \item $(a^{12}b^2) / (4z^{12})$.
      \item $16^{\frac{3}{2}} = (16^{\frac{1}{2}})^{3} = \left( \sqrt{16}  \right) ^{3} = \left( 4 \right) ^3 =
        64$.
    \end{enumerate}
  \end{solution}
\end{qstn}

\textbf{Question 5 - 12. \,\, Refer to Textbook Solutions, ask me if you need clarification.}



















\end{document}




























