% Prepared by Calvin Kent
%
% Assignment Template v19.02
%
%%% 20xx0x/MATHxxx/Crowdmark/Ax
%
\documentclass[12pt]{article} %
\usepackage{amsthm}
\usepackage{CKpreamble}
\usepackage{CKassignment}
\usepackage{mdframed}
\usepackage{euscript}
\usepackage{tikz}
\usepackage{pgfplots}
\usepackage{pdfpages}
\usepackage{euscript}
\usepackage{transparent}
\usepackage{xcolor}
\usepackage{tasks}
\usepackage{tkz-euclide}

%
\begin{document}
	\pagenumbering{arabic}
	% Start of class settings ...
	\renewcommand*{\coursecode}{MATH 235} % renew course code
	\renewcommand*{\assgnnumber}{Assignment 1} % renew assignment number
	\renewcommand*{\submdate}{September 14, 2021} % renew the date
	\renewcommand*{\studentfname}{Abdullah} % Student first name
	\renewcommand*{\studentlname}{Zubair} % Student last name
    \renewcommand*{\proofname}{Proof:}
	% \renewcommand*{\studentnum}{20836288} % Student number

	\renewcommand\qedsymbol{$\blacksquare$}
	\setfigpath
	% End of class settings	
	% \pagestyle{crowdmark}
	\newgeometry{left=18mm, right=18mm, top=22mm, bottom=22mm} % page is set to default values
	\fancyhfoffset[L,O]{0pt} % header orientation fixed
	% End of class settings
	%%% Note to user:
	% CTRL + F <CHANGE ME:> (without the angular brackets) in CKpreamble to specify graphics paths accordingly.
	% The command \circled[]{} accepts one optional and one mandatory argument.
	% Optional argument is for the size of the circle and mandatory argument is for its contents.
	% \circled{A} produces circled A, with size drawn for letter A. \circled[TT]{A} produces circled A with size drawn for TT.
	% https://github.com/CalvinKent/My-LaTeX
	%%%

	%%%%%%%%%%%%%%%%%%%%%%%%%%%%%%%%%%%%%%%%%%%%%%%%%%%%%%%%%%%%%%%%%%%%%%%%%%%%%%%%%%%%%%%%%%%%%%%%%%%%%%%%%%%%%%%%%%
	%%%                        CUSTOM MACRO VIM-TEX                                                      (Word Wrap->
	%%       call IMAP('NOM', '\nomenclature{}', 'tex')               

	%%%%%%%%%%%%%%%%%%%%%%%%%%%%%%%%%%%%%%%%%%%%%%%%%%%%%%%%%%%%%%%%%%%%%%%%%%%%%%%%%%%%%%%%%%%%%%%%%%%%%%%%%%%%%%%%%%

	% Crowdmark assignment start
	% qnumber, qname, points

\begin{center}
	\textbf{\underline{\Huge{Lecture 8 - Homework}}}
\end{center}

\begin{qstn}
  \textbf{Homework sheet \,\, Q3, Q6, Q8}. 
\end{qstn}

\begin{qstn}
  \textbf{Homework sheet \,\, Q9a), Q10a)}. 
\end{qstn}

\begin{qstn}
  \textbf{Homework sheet \,\, Q11, Q12, Q13, Q14, Q16}. 
\end{qstn}

\begin{qstn}
  Determine the value of the following,
  \begin{enumerate}[label=(\alph*)]
    \item $ \cos 60^{\circ}\sin 30^{\circ} + \tan 30^{\circ}\csc 60^{\circ}$
    \item $ \left( \cos 45^{\circ}\right)^2 + \left( \sin 45^{\circ}\right)^2 $
    \item $ \cos 30^{\circ}\sec 30^{\circ} + \tan 30^{\circ}\sin 60^{\circ}$
    \item $ \sec 30^{\circ}\csc 60^{\circ} + 2\left(\cot 30^{\circ}\right)^2$.
  \end{enumerate}
\end{qstn}

\begin{qstn}
  \textbf{Homework sheet \,\, Q31, Q33, Q36}. \\
  \textbf{Note:} When they say solve the triangle, they mean determine \textbf{all} unknown angles and sides.\\
  \textbf{Note:} $\pi / 6 = 30^{\circ}$.
\end{qstn}

\begin{qstn}
  \textbf{Homework sheet \,\, Q41, Q42, Q44}.
\end{qstn}

\begin{qstn}
  (The following problem is from the updated Nelson Functions 11).
    \begin{enumerate}[label=(\alph*)]
      \item $\triangle ABC$, $BM = 6$,  $\angle MBC = 60^{\circ}$ and $\tan \alpha = 1$. Determine the
        \textbf{exact }area of  $\triangle ABC$. 
        \begin{center}
          \begin{tikzpicture}[thick]
          \coordinate (A) at (0,0);
          \coordinate (B) at (4,4);
          \coordinate (C) at (15,0);
          \coordinate (M) at (4,0);

          \draw[black] (A) -- (C) node[circle,fill,inner sep=2pt]{} node[above]{$C$};
          \draw[black] (A) -- (B) node[circle,black,fill,inner sep=2pt]{} node[above]{$B$};
          \draw[black] (B) -- (C) node[circle,black,fill,inner sep=2pt]{};
          \draw[dashed] (B) -- (M) node[circle,black,fill,inner sep=2pt]{} node[midway, left]{$6$};
          \draw[black] (A) node[circle,black,fill,inner sep=2pt]{} node[above]{$A$};
          \draw[black] (M) node[circle,black,fill,inner sep=2pt]{} node[below]{$M$};

          \tkzMarkAngle[fill= orange,size=1cm,%
          opacity=1](C,A,B)
          \tkzLabelAngle[pos = 1.3](C,A,B){$\alpha$}

          \tkzMarkAngle[fill= orange,size=0.9cm,%
          opacity=1](M,B,C)
          \tkzLabelAngle[pos = 1.3](M,B,C){$60^{\circ}$}

          \tkzMarkRightAngle[size=0.4,opacity=0.9](B,M,C)% square angle here

          \end{tikzpicture}
      \end{center}
      \item $\triangle SPQ$, $SP = 13$,  $\angle PQS = 45^{\circ}$ and $\cos \beta = \sqrt{3} / 2$. Determine the
        \textbf{exact }area of  $\triangle SPQ$.
        \begin{center}
          \begin{tikzpicture}[thick]
          \coordinate (S) at (0,0);
          \coordinate (P) at (6,4);
          \coordinate (Q) at (15,0);
          \coordinate (R) at (6,0);

          \draw[black] (S) -- (Q) node[circle,fill,inner sep=2pt]{} node[above]{$Q$};
          \draw[black] (S) -- (P) node[circle,black,fill,inner sep=2pt]{} node[above]{$P$} node[midway, above]{$13$};
          \draw[black] (P) -- (Q) node[circle,black,fill,inner sep=2pt]{};
          \draw[dashed] (P) -- (R) node[circle,black,fill,inner sep=2pt]{};
          \draw[black] (S) node[circle,black,fill,inner sep=2pt]{} node[above]{$S$};
          \draw[black] (R) node[circle,black,fill,inner sep=2pt]{} node[below]{$R$};

          \tkzMarkAngle[fill= orange,size=1cm,%
          opacity=1](S,P,R)
          \tkzLabelAngle[pos = 1.4](S,P,R){$\beta$}

          \tkzMarkAngle[fill= orange,size=1.4cm,%
          opacity=1](P,Q,S)
          \tkzLabelAngle[pos = 1.8](P,Q,S){$45^{\circ}$}

          \tkzMarkRightAngle[size=0.4,opacity=0.9](S,R,P)% square angle here

          \end{tikzpicture}
      \end{center}
    \end{enumerate}


\end{qstn}









\end{document}































