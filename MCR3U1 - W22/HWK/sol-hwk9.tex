% Prepared by Calvin Kent
%
% Assignment Template v19.02
%
%%% 20xx0x/MATHxxx/Crowdmark/Ax
%
\documentclass[12pt]{article} %
\usepackage{amsthm}
\usepackage{CKpreamble}
\usepackage{CKassignment}
\usepackage{mdframed}
\usepackage{euscript}
\usepackage{tikz}
\usepackage{pgfplots}
\usepackage{pdfpages}
\usepackage{euscript}
\usepackage{transparent}
\usepackage{xcolor}
\usepackage{tasks}
\usepackage{tkz-euclide}

%
\begin{document}
	\pagenumbering{arabic}
	% Start of class settings ...
	\renewcommand*{\coursecode}{MATH 235} % renew course code
	\renewcommand*{\assgnnumber}{Assignment 1} % renew assignment number
	\renewcommand*{\submdate}{September 14, 2021} % renew the date
	\renewcommand*{\studentfname}{Abdullah} % Student first name
	\renewcommand*{\studentlname}{Zubair} % Student last name
    \renewcommand*{\proofname}{Proof:}
	% \renewcommand*{\studentnum}{20836288} % Student number

	\renewcommand\qedsymbol{$\blacksquare$}
	\setfigpath
	% End of class settings	
	% \pagestyle{crowdmark}
	\newgeometry{left=18mm, right=18mm, top=22mm, bottom=22mm} % page is set to default values
	\fancyhfoffset[L,O]{0pt} % header orientation fixed
	% End of class settings
	%%% Note to user:
	% CTRL + F <CHANGE ME:> (without the angular brackets) in CKpreamble to specify graphics paths accordingly.
	% The command \circled[]{} accepts one optional and one mandatory argument.
	% Optional argument is for the size of the circle and mandatory argument is for its contents.
	% \circled{A} produces circled A, with size drawn for letter A. \circled[TT]{A} produces circled A with size drawn for TT.
	% https://github.com/CalvinKent/My-LaTeX
	%%%

	%%%%%%%%%%%%%%%%%%%%%%%%%%%%%%%%%%%%%%%%%%%%%%%%%%%%%%%%%%%%%%%%%%%%%%%%%%%%%%%%%%%%%%%%%%%%%%%%%%%%%%%%%%%%%%%%%%
	%%%                        CUSTOM MACRO VIM-TEX                                                      (Word Wrap->
	%%       call IMAP('NOM', '\nomenclature{}', 'tex')               

	%%%%%%%%%%%%%%%%%%%%%%%%%%%%%%%%%%%%%%%%%%%%%%%%%%%%%%%%%%%%%%%%%%%%%%%%%%%%%%%%%%%%%%%%%%%%%%%%%%%%%%%%%%%%%%%%%%

	% Crowdmark assignment start
	% qnumber, qname, points

\begin{center}
	\textbf{\underline{\Huge{Solutions - Lecture 9 - Homework}}}
\end{center}

\begin{qstn} \texttt{  }
    \begin{enumerate}[label=(\alph*)]
      \item
        \[
            \sin 30^{\circ} = \frac{1}{2} \hspace{0.7cm} \cos 30^{\circ} = \frac{\sqrt{3}}{2} \hspace{0.7cm} 
            \tan 30^{\circ} = \frac{1}{\sqrt{3}}
        \] 

        \[
            \csc 30^{\circ} = 2 \hspace{0.7cm} \sec 30^{\circ} = \frac{2}{\sqrt{3}} \hspace{0.7cm} 
            \cot 30^{\circ} = \sqrt{3} 
        \] 

      \item
        \[
            \sin 225^{\circ} = -\frac{1}{\sqrt{2}} \hspace{0.7cm} \cos 225^{\circ} = -\frac{1}{\sqrt{2}} \hspace{0.7cm} 
            \tan 225^{\circ} = 1
        \] 

        \[
            \csc 225^{\circ} = -\sqrt{2}  \hspace{0.7cm} \sec 225^{\circ} = -\sqrt{2}  \hspace{0.7cm} 
            \cot 225^{\circ} = 1
        \] 

      \item
        \[
            \sin -240^{\circ} = \frac{\sqrt{3}}{2} \hspace{0.7cm} \cos -240^{\circ} = -\frac{1}{2} \hspace{0.7cm} 
            \tan -240^{\circ} = -\sqrt{3}. 
        \] 

        \[
            \csc -240^{\circ} = \frac{2}{\sqrt{3}} \hspace{0.7cm} \sec -240^{\circ} = -2 \hspace{0.7cm} 
            \cot -240^{\circ} = -\frac{1}{\sqrt{3}}. 
        \] 

      \item
        \[
            \sin 330^{\circ} = -\frac{1}{2} \hspace{0.7cm} \cos 330^{\circ} = \frac{\sqrt{3}}{2} \hspace{0.7cm} 
            \tan 330^{\circ} = -\frac{1}{\sqrt{3}}
        \] 

        \[
            \csc 330^{\circ} = -2 \hspace{0.7cm} \sec 330^{\circ} = \frac{2}{\sqrt{3}} \hspace{0.7cm} 
            \cot 330^{\circ} = -\sqrt{3} 
        \] 
      
    \end{enumerate}
\end{qstn}

\begin{qstn}\texttt{  }
  \begin{enumerate}[label=(\alph*)]
    \item $ \vb P\left( \frac{1}{\sqrt{2}}, \frac{1}{\sqrt{2}} \right)$
    \item $ \vb Q\left( -1, \sqrt{3}  \right) $
    \item $ \vb T\left( -2\sqrt{3} , -2 \right)$
    \item $ \vb M\left( 0,-3  \right) $
    \item $ \vb G\left( \frac{3\sqrt{3}}{2} ,-\frac{3}{2}  \right) $
  \end{enumerate}
\end{qstn}

\newpage

\begin{qstn} \texttt{  }
    \begin{tasks}(4)
       \task $\theta_1 = 330^{\circ}$
       \task $\theta_2 = 330^{\circ}$
       \task $\theta_3 = 300^{\circ}$
       \task $\theta_4 = 345.96^{\circ}$
    \end{tasks}
\end{qstn}

\begin{qstn} \texttt{  }
    \begin{tasks}(4)
       \task $\lambda_1 = 135^{\circ}$
       \task $\lambda_2 = 120^{\circ}$
       \task $\lambda_3 = 150^{\circ}$
       \task $\lambda_4 = 123.69^{\circ}$
    \end{tasks}
\end{qstn}

\begin{qstn}\texttt{  }
  \begin{enumerate}[label=(\alph*)]
    \item $ \vb P\left( 8, 330^{\circ}\right)$
    \item $ \vb Q\left( \sqrt{2} , 135^{\circ} \right) $
    \item $ \vb T\left( 3\sqrt{5} , 206.56^{\circ} \right)$
    \item $ \vb M\left( 6, 300^{\circ} \right) $
  \end{enumerate}
\end{qstn}

\newpage





\end{document}































