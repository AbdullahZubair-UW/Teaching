
% Prepared by Calvin Kent
%
% Assignment Template v19.02
%
%%% 20xx0x/MATHxxx/Crowdmark/Ax
%
\documentclass[12pt]{article} %
\usepackage{CKpreamble}
\usepackage{CKassignment}
\usepackage{tkz-euclide}
\usepackage{physunits}
\usepackage{physics}
\usepackage{lmodern}
\usepackage{microtype}
\usepackage{upgreek}
\usepackage[misc]{ifsym}


%%Title
\title{\textbf{Assignment 1 Functions} \\ \textbf{Due Date: } Thursday, December 16}
\date{November, 2021}

%%% Maths and science packages

\usepackage{amsmath,amsthm,amssymb}
\usepackage{pgfplots}
	\usetikzlibrary{
		calc,
		patterns,
		positioning
	}
	\pgfplotsset{
		compat=1.16,
		samples=200,
		clip=false,
		my axis style/.style={
			axis x line=middle,
			axis y line=middle,
			legend pos=outer north east,
			axis line style={
				->,
			},
			legend style={
				font=\footnotesize
			},
			label style={
				font=\footnotesize
			},
			tick label style={
				font=\footnotesize
			},
			xlabel style={
				at={
					(ticklabel* cs:1)
				},
				anchor=west,
				font=\footnotesize,
			},
			ylabel style={
				at={
					(ticklabel* cs:1)
				},
				anchor=west,
				font=\footnotesize,
			},
			xlabel= $x$,
			ylabel=$\vec d (\m \tx{[East]})$
		},
	}
	\tikzset{
		>=stealth
	}

%%% Tables and figures packages

\usepackage{float}
\usepackage{caption}
	\captionsetup{
		format=plain,
		labelfont=bf,
		font=small,
		justification=centering
	}
	
%%% Numbers and sets

\newcommand{\E}{\mathrm{e}}

\newcommand{\tx}[1]{\text{#1}}
\newcommand{\rem}[1]{\operatorname{rem}{(#1)}}


%
\begin{document}
	\pagenumbering{arabic}
	% Start of class settings ...
	\renewcommand*{\coursecode}{MATH 235} % renew course code
	\renewcommand*{\assgnnumber}{Assignment 1} % renew assignment number
	\renewcommand*{\submdate}{September 14, 2021} % renew the date
	\renewcommand*{\studentfname}{Abdullah} % Student first name
	\renewcommand*{\studentlname}{Zubair} % Student last name
    \renewcommand*{\proofname}{Proof:}
	% \renewcommand*{\studentnum}{20836288} % Student number

	\renewcommand\qedsymbol{$\blacksquare$}
	\setfigpath
	% End of class settings	
	% \pagestyle{crowdmark}
	\newgeometry{left=18mm, right=18mm, top=22mm, bottom=22mm} % page is set to default values
	\fancyhfoffset[L,O]{0pt} % header orientation fixed
	% End of class settings
	%%% Note to user:
	% CTRL + F <CHANGE ME:> (without the angular brackets) in CKpreamble to specify graphics paths accordingly.
	% The command \circled[]{} accepts one optional and one mandatory argument.
	% Optional argument is for the size of the circle and mandatory argument is for its contents.
	% \circled{A} produces circled A, with size drawn for letter A. \circled[TT]{A} produces circled A with size drawn for TT.
	% https://github.com/CalvinKent/My-LaTeX
	%%%

	%%%%%%%%%%%%%%%%%%%%%%%%%%%%%%%%%%%%%%%%%%%%%%%%%%%%%%%%%%%%%%%%%%%%%%%%%%%%%%%
	%%%                        CUSTOM MACRO VIM-TEX                             %%%
	%%       call IMAP('NOM', '\nomenclature{}', 'tex')               

	%%%%%%%%%%%%%%%%%%%%%%%%%%%%%%%%%%%%%%%%%%%%%%%%%%%%%%%%%%%%%%%%%%%%%%%%%%%%%%%

	% Crowdmark assignment start
	% qnumber, qname, qpoints
\maketitle
	\section{Preamble}
  This assignment covers everything most of Unit 1. The solutions that you hand in should be \textbf{neat} and \textbf{legible},
  this is an assignment, not a quiz, so I expect you to take your time and present thorough and detailed solutions.
\section{Name and Date:}
	Print your name and todays date below;\\


	\begin{center}
	\noindent\begin{tabular}{ll}
		\makebox[3in]{\hrulefill} & \makebox[3in]{\hrulefill}\\
		Name & Date\\[8ex]% adds space between the two sets of signatures
	\end{tabular}
	\end{center}
	\newpage



\begin{qstn}
  We define the \textbf{cardinality} of sets to be the size of the set, in other words the cardinality of a set is the number of
  elements in the set. For example if $S = \{3,2,1,\triangle \} $, then we say the that cardinality of $S$ is $4$, because  $S$ 
  consists of four elements. The notation we use to describe the cardinality of a set is a pair of bars $(\left|\right|)$ similar to absolute
  values. So going back to our example with  $S$, instead of saying the sentence; The cardinality of  $S$ is $4$, we could
  instead simply write $\left|S\right| = 4$. In this question we will discover that for sets $A,B,C$, it is \textbf{not} always
  true that $\left|A + B\right| = \left|A\right| + \left|B\right|$.

  Lets say we have the following sets,
  \begin{itemize}
    \item $\mathcal{H} = \{x\in \Z \mid -1 \leq x \leq 5\} $
    \item $\mathcal{T} = \{y\in \Z \mid  2 \leq y < 7\} $
  \end{itemize}
  

  \begin{enumerate}[label=(\alph*)]
    \item Write down the elements of both sets.
    \item Determine $|\mathcal{H}|$ and $|\mathcal{T}|$.
    \item Determine $|\mathcal{H + T}|$. (Go back to Homework-1 if you forgot how we preform set addition).
    \item Explain why $|\mathcal{H + T}| \neq |\mathcal{H}| + |\mathcal{T}|$.
  \end{enumerate}
\end{qstn}


\begin{qstn}
  Let 
  \begin{align*}
    f(x) &= -2x^2 - 4x + 5\\
    g(x) &= \frac{3}{4}x - 4
  \end{align*}
  \begin{enumerate}[label=(\alph*)]
    \item Compute $f(-2)$ \textbf{and} $g(3)$.
    \item Compute $f(g(f(0)))$.
    \item Provide a \textbf{sketch} of $f$ on the \emph{axis sheet}.
    \item Provide a \textbf{graph} of $g$ on the \emph{axis sheet}.
    \item Does the vertex of $f$ represent a minimum or maximum? Explain your answer.
  \end{enumerate}

\end{qstn}


\begin{qstn} Recall that when you divide two numbers, there will always be a remainder, sometimes the remainder is zero, other
  times it may not be zero, lets take a look at the following examples;
  \begin{itemize}
    \item $\frac{5}{3}$, the remainder is $2$.
    \item $\frac{12}{2}$, the remainder is $0$.
    \item $\frac{4}{7}$, the remainder is $4$. 
  \end{itemize}
  There is a better way to describe the remainder when dividing two numbers, that is to use the following notation,
  \[
        \text{rem}(a,b)
  .\] The output of this is the remainder when dividing $a$ by $b$, so if we repeat our previous examples again using this new
  notation, we would have,
  \begin{itemize}
    \item $\operatorname{rem}(5,3) = 2$
    \item $\operatorname{rem}(12,2) = 0$
    \item $\operatorname{rem}(4,7) = 4$
  \end{itemize}
  The set of all 'positive numbers' is written as,
  \[
        \mathbb N = \{1,2,3,4,5,6,7,\dots\} 
  .\] Now lets define the following function,
  \begin{align*}
    f &\colon \mathbb N \to \mathbb N\\
    f(n) &= \rem{n,2} + 1
  \end{align*}
  Lets take a look at some examples to see how the function behaves,
  \begin{itemize}
    \item $f(4) = \rem{4,2} + 1 = 0 + 1 = 0$
    \item $f(11) = \rem{11,2} + 1 = 1 + 1 = 2$
  \end{itemize}
  Determine the following,
  \begin{enumerate}[label=(\alph*)]
    \item $f(6)$
    \item $f(15)$
    \item $f(f(1))$
    \item The range of $f$.\\
      (Think carefully about this one, have you seen a pattern with the outputs so far?)
  \end{enumerate}

\end{qstn}


\begin{qstn}
  Determine the Domain and Range of the following functions,
  \begin{enumerate}[label=(\alph*)]
    \item $L(x) = -4x^2 + 8x + 4$
    \item $R(x) = -50\left|x - 3\right| - 7$
    \item $Q(x) = \frac{-3}{2x - 4} + \frac{4}{5}$
    \item $g(x) = 11\sqrt{-3x + 12} + 1$ 
    \item $x^2 + (y + 11)^2 = 4$.
  \end{enumerate}
\end{qstn}

\begin{qstn}
  Lets say we have the following functions,
  \begin{itemize}
    \item $f(x) = x^2 - 2x - 3$
    \item $g(x) = x - 4$
  \end{itemize}
  Lets define a new function,
  \[
          h(x) = f(g(x))
  .\] 
  \begin{enumerate}[label=(\alph*)]
    \item How many solutions will $h(x)$ have? (\textbf{Hint:} Use the discriminant formula)
    \item Determine the solutions to,
      \[
            h(x) = 0
      .\] 
    \item Determine the solutions to,
       \[
            f(x) = g(x)
      .\]
      (\textbf{Note!:} Leave answers in \textbf{exact} form, NO decimals).

  \end{enumerate}
  
\end{qstn}


\end{document}



































