% Prepared by Calvin Kent
%
% Assignment Template v19.02
%
%%% 20xx0x/MATHxxx/Crowdmark/Ax
%
\documentclass[12pt]{article} %
\usepackage{amsthm}
\usepackage{CKpreamble}
\usepackage{CKassignment}
\usepackage{mdframed}
\usepackage{euscript}
\usepackage{tikz}
\usepackage{pgfplots}

%
\begin{document}
	\pagenumbering{arabic}
	% Start of class settings ...
	\renewcommand*{\coursecode}{MATH 235} % renew course code
	\renewcommand*{\assgnnumber}{Assignment 1} % renew assignment number
	\renewcommand*{\submdate}{September 14, 2021} % renew the date
	\renewcommand*{\studentfname}{Abdullah} % Student first name
	\renewcommand*{\studentlname}{Zubair} % Student last name
    \renewcommand*{\proofname}{Proof:}
	% \renewcommand*{\studentnum}{20836288} % Student number

	\renewcommand\qedsymbol{$\blacksquare$}
	\setfigpath
	% End of class settings	
	% \pagestyle{crowdmark}
	\newgeometry{left=18mm, right=18mm, top=22mm, bottom=22mm} % page is set to default values
	\fancyhfoffset[L,O]{0pt} % header orientation fixed
	% End of class settings
	%%% Note to user:
	% CTRL + F <CHANGE ME:> (without the angular brackets) in CKpreamble to specify graphics paths accordingly.
	% The command \circled[]{} accepts one optional and one mandatory argument.
	% Optional argument is for the size of the circle and mandatory argument is for its contents.
	% \circled{A} produces circled A, with size drawn for letter A. \circled[TT]{A} produces circled A with size drawn for TT.
	% https://github.com/CalvinKent/My-LaTeX
	%%%

	%%%%%%%%%%%%%%%%%%%%%%%%%%%%%%%%%%%%%%%%%%%%%%%%%%%%%%%%%%%%%%%%%%%%%%%%%%%%%%%
	%%%                        CUSTOM MACRO VIM-TEX                             %%%
	%%       call IMAP('NOM', '\nomenclature{}', 'tex')               

	%%%%%%%%%%%%%%%%%%%%%%%%%%%%%%%%%%%%%%%%%%%%%%%%%%%%%%%%%%%%%%%%%%%%%%%%%%%%%%%

	% Crowdmark assignment start
	% qnumber, qname, points

\begin{center}
	\textbf{\underline{\Huge{Lecture 5 - Homework (FIXED **)}}}
\end{center}

\textbf{NOTE*:} For Question 1 and Question 2, in each function you are given a set of coordinates that you \textbf{must}
  transform, and you must \textbf{label} them on your final sketch. Of course, you can use more coordinates as you wish for 
  more accuracy in your sketch.


\begin{qstn}
  Let $f(x) = x^2$. Sketch the following functions,
  \begin{enumerate}[label=(\alph*)]
    \item $\EuScript{Z}(x) = -f(-x)$,
      \begin{enumerate}
        \item [(i)] $(0,0), (2,4), (-2,4)$.
      \end{enumerate}
    \item $L(x) = -2f(\frac{1}{4}x) + 3$,
      \begin{enumerate}
        \item [(i)] $(0,0), (-1,1), (-2,4)$.
      \end{enumerate}
    \item $R(x) = \frac{1}{2}f(-2x + 2) - 1$, 
      \begin{enumerate}
        \item [(i)] $(0,0), (-2,4), (4,16)$.
      \end{enumerate}
  \end{enumerate}
\end{qstn}

\begin{qstn}
  Let $f(x) = \left|x\right|$. Sketch the following functions,
  \begin{enumerate}[label=(\alph*)]
    \item $r(x) = -2f(x + 1)$, 
      \begin{enumerate}
        \item [(i)] $(0,0), (2,2), (-3,3)$.
      \end{enumerate}
    \item $g(x) = -f(x - 3) + 1$, 
      \begin{enumerate}
        \item [(i)] $(0,0), (-5,5), (3,3)$.
      \end{enumerate}
    \item $h(x) = \frac{3}{2}f(-2x + 4) - 2$,
      \begin{enumerate}
        \item [(i)] $(0,0), (4,4), (-4,4)$.
      \end{enumerate}
  \end{enumerate}
\end{qstn}

\begin{qstn}
  Let $f(x) = \sqrt{x} $. Suppose we apply the following transformations to $f$,
  \begin{itemize}
    \item Reflection across the x-axis.
    \item Vertical compression by a factor of $2$.
    \item Horizontal compression by a factor of $4$.
    \item Horizontal shift, left by $4$ units.
    \item Vertical shift, up by $1$ units.
  \end{itemize}
  Sketch the transformed function and label it $g(x)$. 
  \\ \textbf{Label} the coordinates $(-4,g(-4))$, $(0,g(0))$, $(12,g(12))$.
  
\end{qstn}

\begin{qstn}
  For each of the following, state the transformations preformed to the \textit{parent function}.
  \begin{enumerate}[label=(\alph*)]
    \item $f(x) = -\frac{2}{3}(3x - 6)^2 - 1$.
    \item $g(x) = 3\sqrt{-4x + 12} + 9$.
    \item $h(x) = -4\left|\frac{1}{2}x + 1\right|$.
  \end{enumerate}
\end{qstn}

\begin{qstn}
  \textbf{Homework sheet, Q4, Q61, Q62.}
\end{qstn}




\end{document}































