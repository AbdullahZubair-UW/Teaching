% Prepared by Calvin Kent
%
% Assignment Template v19.02
%
%%% 20xx0x/MATHxxx/Crowdmark/Ax
%
\documentclass[12pt]{article} %
\usepackage{amsthm}
\usepackage{CKpreamble}
\usepackage{CKassignment}
\usepackage{mdframed}
\usepackage{euscript}
\usepackage{tikz}
\usepackage{pgfplots}
\usepackage{pdfpages}
\usepackage{euscript}
\usepackage{transparent}
\usepackage{xcolor}
\usepackage{tasks}
\usepackage{tkz-euclide}

%
\begin{document}
	\pagenumbering{arabic}
	% Start of class settings ...
	\renewcommand*{\coursecode}{MATH 235} % renew course code
	\renewcommand*{\assgnnumber}{Assignment 1} % renew assignment number
	\renewcommand*{\submdate}{September 14, 2021} % renew the date
	\renewcommand*{\studentfname}{Abdullah} % Student first name
	\renewcommand*{\studentlname}{Zubair} % Student last name
    \renewcommand*{\proofname}{Proof:}
	% \renewcommand*{\studentnum}{20836288} % Student number

	\renewcommand\qedsymbol{$\blacksquare$}
	\setfigpath
	% End of class settings	
	% \pagestyle{crowdmark}
	\newgeometry{left=18mm, right=18mm, top=22mm, bottom=22mm} % page is set to default values
	\fancyhfoffset[L,O]{0pt} % header orientation fixed
	% End of class settings
	%%% Note to user:
	% CTRL + F <CHANGE ME:> (without the angular brackets) in CKpreamble to specify graphics paths accordingly.
	% The command \circled[]{} accepts one optional and one mandatory argument.
	% Optional argument is for the size of the circle and mandatory argument is for its contents.
	% \circled{A} produces circled A, with size drawn for letter A. \circled[TT]{A} produces circled A with size drawn for TT.
	% https://github.com/CalvinKent/My-LaTeX
	%%%

	%%%%%%%%%%%%%%%%%%%%%%%%%%%%%%%%%%%%%%%%%%%%%%%%%%%%%%%%%%%%%%%%%%%%%%%%%%%%%%%%%%%%%%%%%%%%%%%%%%%%%%%%%%%%%%%%%%
	%%%                        CUSTOM MACRO VIM-TEX                                                      (Word Wrap->
	%%       call IMAP('NOM', '\nomenclature{}', 'tex')               

	%%%%%%%%%%%%%%%%%%%%%%%%%%%%%%%%%%%%%%%%%%%%%%%%%%%%%%%%%%%%%%%%%%%%%%%%%%%%%%%%%%%%%%%%%%%%%%%%%%%%%%%%%%%%%%%%%%

	% Crowdmark assignment start
	% qnumber, qname, points

\begin{center}
	\textbf{\underline{\Huge{Lecture 9 - Homework}}}
\end{center}

\begin{qstn}
  Determine the six trigonometric ratios for the following angles,
  \begin{multicols}{4}
    \begin{enumerate}[label=(\alph*)]
      \item $\theta_1 = 30^{\circ}$
        \columnbreak
      \item $\theta_2 = 225^{\circ}$
        \columnbreak
      \item $\theta_3 = -240^{\circ}$
        \columnbreak
      \item $\theta_4 = 330^{\circ}$
    \end{enumerate}
    
  \end{multicols}

\end{qstn}

\begin{qstn}
     Convert the following polar coordinates to standard coordinates,
    \begin{tasks}(5)
     \task $\vb P(1, 45^{\circ})$
     \task $\vb Q(2, -240^{\circ})$ 
     \task $\vb T(4, 210^{\circ})$
     \task $\vb M(3, -90^{\circ})$ 
     \task $\vb G(3, 330^{\circ})$
    \end{tasks}
\end{qstn}

\begin{qstn}
  For each of the following, you are given a trigonometric ratio, solve for $\theta$. 
  Assume that each angle  $\theta$ lies in the \textbf{fourth} quadrant.
    \begin{tasks}(4)
       \task $\cos \theta_1 = \frac{\sqrt{3}}{2}$
       \task $\sin \theta_2 = -\frac{1}{2} $
       \task $\tan \theta_3 = \sqrt{3} $
       \task $\tan \theta_4 = -\frac{1}{4}$
    \end{tasks}
\end{qstn}

\begin{qstn}
  For each of the following, you are given a trigonometric ratio, solve for $\lambda$. 
  Assume that each angle  $\lambda$ lies in the \textbf{second} quadrant.
    \begin{tasks}(4)
       \task $\cos \lambda_1 = -\frac{1}{\sqrt{2}}$
       \task $\sin \lambda_2 = \frac{\sqrt{3}}{2}$
       \task $\tan \lambda_3 = -\frac{1}{\sqrt{3}}$
       \task $\tan \lambda_4 = -\frac{3}{2}$
    \end{tasks}
\end{qstn}

\begin{qstn}
     Convert the following standard coordinates to polar coordinates,
    \begin{tasks}(4)
     \task $\vb P\left(4\sqrt{3}, -4\right)$
     \task $\vb Q(-1,1)$ 
     \task $\vb T\left(-6,-3\right)$
     \task $\vb M\left(3,-\sqrt{27} \right)$ 
    \end{tasks}
\end{qstn}






\end{document}































