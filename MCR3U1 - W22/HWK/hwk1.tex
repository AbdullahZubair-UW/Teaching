% Prepared by Calvin Kent
%
% Assignment Template v19.02
%
%%% 20xx0x/MATHxxx/Crowdmark/Ax
%
\documentclass[12pt]{article} %
\usepackage{amsthm}
\usepackage{CKpreamble}
\usepackage{CKassignment}
\usepackage{mdframed}
\usepackage{import}
\usepackage{pdfpages}
\usepackage{transparent}
\usepackage{xcolor}

\newcommand{\incfig}[2][1]{%
    \def\svgwidth{#1\columnwidth}
    \import{./figures/}{#2.pdf_tex}
}

\pdfsuppresswarningpagegroup=1


%
\begin{document}
	\pagenumbering{arabic}
	% Start of class settings ...
	\renewcommand*{\coursecode}{MATH 235} % renew course code
	\renewcommand*{\assgnnumber}{Assignment 1} % renew assignment number
	\renewcommand*{\submdate}{September 14, 2021} % renew the date
	\renewcommand*{\studentfname}{Abdullah} % Student first name
	\renewcommand*{\studentlname}{Zubair} % Student last name
    \renewcommand*{\proofname}{Proof:}
	% \renewcommand*{\studentnum}{20836288} % Student number

	\renewcommand\qedsymbol{$\blacksquare$}
	\setfigpath
	% End of class settings	
	% \pagestyle{crowdmark}
	\newgeometry{left=18mm, right=18mm, top=22mm, bottom=22mm} % page is set to default values
	\fancyhfoffset[L,O]{0pt} % header orientation fixed
	% End of class settings
	%%% Note to user:
	% CTRL + F <CHANGE ME:> (without the angular brackets) in CKpreamble to specify graphics paths accordingly.
	% The command \circled[]{} accepts one optional and one mandatory argument.
	% Optional argument is for the size of the circle and mandatory argument is for its contents.
	% \circled{A} produces circled A, with size drawn for letter A. \circled[TT]{A} produces circled A with size drawn for TT.
	% https://github.com/CalvinKent/My-LaTeX
	%%%

	%%%%%%%%%%%%%%%%%%%%%%%%%%%%%%%%%%%%%%%%%%%%%%%%%%%%%%%%%%%%%%%%%%%%%%%%%%%%%%%
	%%%                        CUSTOM MACRO VIM-TEX                             %%%
	%%       call IMAP('NOM', '\nomenclature{}', 'tex')               

	%%%%%%%%%%%%%%%%%%%%%%%%%%%%%%%%%%%%%%%%%%%%%%%%%%%%%%%%%%%%%%%%%%%%%%%%%%%%%%%

	% Crowdmark assignment start
	% qnumber, qname, qpoints

\begin{center}
	\textbf{\underline{\Huge{Lecture 1 - Homework}}}
\end{center}
\begin{qstn}
	Come up with three sets and write them down.
\end{qstn}
\begin{qstn}
	Simply the following sets as much as possible.
	\begin{enumerate}[label = (\alph*)]
		\item $ \{3,2,1,1,2,3\} $
		\item $ \{A,B,B,C,D,D\} $
		\item $\{1,2,3,4\} $
	\end{enumerate}
\end{qstn}
\begin{qstn}
	Write down \textbf{three} equivalent representations of the set $S = \{1,2,3\} .$ (\textbf{Hint:} Remark 1.1)
\end{qstn}
\begin{qstn}
	What are dots $(\ldots)$ used for in sets?
\end{qstn}
\begin{qstn}
	Let $S = \{4,5,6,\triangle\} .$ Which of the following choices are \textbf{True?}
	\begin{enumerate}[label = (\alph*)]
		\item $5 \in S$ 
		\item $\circ \not\in S$ 
		\item $\triangle  \not\in  S$ 
		\item $(4 + 6 )\in S$ 
	\end{enumerate}
\end{qstn}
\begin{qstn}
	Is $\sqrt{2} \in \Q$? Explain your answer.
\end{qstn}
\begin{qstn}
	Why do you think the empty set ($\emptyset$) might be important in set theory?
\end{qstn}
\begin{qstn}
	Describe the following sets symbolically.
	\begin{enumerate}[label = (\alph*)]
		\item All integers that are greater than or equal to $-2$.
		\item All rational numbers that are not zero.
		\item All the real numbers that are greater than or equal to $-2$ and less than $6$.
	\end{enumerate}
\end{qstn}

\begin{qstn}
	Write down the elements of the following sets.
	\begin{enumerate}[label=(\alph*)]
		\item $A = \{x \in \Z \mid x \le -5\} $
		\item $B = \{x \in \Z \mid 1 < x \leq 6 \} $
		\item $T = \{y \in \Z \mid y^{2} = 4 \} $
	\end{enumerate}
	
\end{qstn}
\begin{qstn}
	In mathematics, when we add two sets we basically merge all elements into a \textbf{single} set. Therefore, determine the
	following sum,\[
	\{3,4,5\}  + \{3,6,8\} 
	.\] 
\end{qstn}

\begin{qstn}
	\textbf{(Challenge)} Describe the set of all odd integers symbolically.
\end{qstn}






































\end{document}




















