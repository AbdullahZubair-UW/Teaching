
% Prepared by Calvin Kent
%
% Assignment Template v19.02
%
%%% 20xx0x/MATHxxx/Crowdmark/Ax
%
\documentclass[12pt]{article} %
\usepackage{amsthm}
\usepackage{CKpreamble}
\usepackage{CKassignment}
\usepackage{mdframed}
\usepackage{euscript}
\usepackage{tikz}
\usepackage{pgfplots}

%
\begin{document}
	\pagenumbering{arabic}
	% Start of class settings ...
	\renewcommand*{\coursecode}{MATH 235} % renew course code
	\renewcommand*{\assgnnumber}{Assignment 1} % renew assignment number
	\renewcommand*{\submdate}{September 14, 2021} % renew the date
	\renewcommand*{\studentfname}{Abdullah} % Student first name
	\renewcommand*{\studentlname}{Zubair} % Student last name
    \renewcommand*{\proofname}{Proof:}
	% \renewcommand*{\studentnum}{20836288} % Student number

	\renewcommand\qedsymbol{$\blacksquare$}
	\setfigpath
	% End of class settings	
	% \pagestyle{crowdmark}
	\newgeometry{left=18mm, right=18mm, top=22mm, bottom=22mm} % page is set to default values
	\fancyhfoffset[L,O]{0pt} % header orientation fixed
	% End of class settings
	%%% Note to user:
	% CTRL + F <CHANGE ME:> (without the angular brackets) in CKpreamble to specify graphics paths accordingly.
	% The command \circled[]{} accepts one optional and one mandatory argument.
	% Optional argument is for the size of the circle and mandatory argument is for its contents.
	% \circled{A} produces circled A, with size drawn for letter A. \circled[TT]{A} produces circled A with size drawn for TT.
	% https://github.com/CalvinKent/My-LaTeX
	%%%

	%%%%%%%%%%%%%%%%%%%%%%%%%%%%%%%%%%%%%%%%%%%%%%%%%%%%%%%%%%%%%%%%%%%%%%%%%%%%%%%
	%%%                        CUSTOM MACRO VIM-TEX                             %%%
	%%       call IMAP('NOM', '\nomenclature{}', 'tex')               

	%%%%%%%%%%%%%%%%%%%%%%%%%%%%%%%%%%%%%%%%%%%%%%%%%%%%%%%%%%%%%%%%%%%%%%%%%%%%%%%

	% Crowdmark assignment start
	% qnumber, qname, points

\begin{center}
	\textbf{\underline{\Huge{Solutions - Lecture 4 - Homework}}}
\end{center}

\begin{qstn}
  \begin{solution}
    Yes it is true that the inverse function for the identity function $\operatorname{id}_\R$ is $\operatorname{id}_\R$.
    We can provide proof for this claim quite simply by confirming that both conditions from Definition 4.1 hold,
    \begin{enumerate}
      \item For all $x \in \R$, $\operatorname{id}_\R(\operatorname{id}_\R(x)) = \operatorname{id}_\R(x) = x$.
      \item For all $x \in \R$, $\operatorname{id}_\R(\operatorname{id}_\R(x)) = \operatorname{id}_\R(x) = x$.
    \end{enumerate}
  \end{solution}
\end{qstn}

\begin{qstn}
\begin{prf}
  Proceeding with mapping tables,
  \begin{center}
    \begin{tabular}{c|c}
  \text{$\mathcal{V}$} & \text{$\mathcal{L}^{-1}\left( \mathcal{L}(v) \right) $}\\\hline 
        $1$ & $\mathcal{L}^{-1}\left( \mathcal{L}(1) \right) =  \mathcal{L}^{-1}\left( 3 \right) = (3-2)^2 = 1$\\
        $4$ & $\mathcal{L}^{-1}\left( \mathcal{L}(4) \right) =  \mathcal{L}^{-1}\left( 4 \right) = (4-2)^2 = 4$\\
        $9$ & $\mathcal{L}^{-1}\left( \mathcal{L}(9) \right) =  \mathcal{L}^{-1}\left( 5 \right) = (5-2)^2 = 9$\\
        $25$ & $\mathcal{L}^{-1}\left( \mathcal{L}(25) \right) =  \mathcal{L}^{-1}\left( 7 \right) = (7-2)^2 = 25$\\
        $49$ & $\mathcal{L}^{-1}\left( \mathcal{L}(49) \right) =  \mathcal{L}^{-1}\left( 9 \right) = (9-2)^2 = 49$
 	\end{tabular}

  \vspace*{0.5cm}

    \begin{tabular}{c|c}
  \text{$\mathcal{W}$} & \text{$\mathcal{L}\left( \mathcal{L}^{-1}(w) \right) $}\\\hline 
        $5$ & $\mathcal{L}\left( \mathcal{L}^{-1}(5) \right) =  \mathcal{L}\left( 9 \right) = \sqrt{9}+2 = 5$\\
        $4$ & $\mathcal{L}\left( \mathcal{L}^{-1}(4) \right) =  \mathcal{L}\left( 4 \right) = \sqrt{4}+2 = 4$\\
        $7$ & $\mathcal{L}\left( \mathcal{L}^{-1}(7) \right) =  \mathcal{L}\left( 25 \right) = \sqrt{25}+2 = 7$\\
        $9$ & $\mathcal{L}\left( \mathcal{L}^{-1}(9) \right) =  \mathcal{L}\left( 49 \right) = \sqrt{49}+2 = 9$\\
        $3$ & $\mathcal{L}\left( \mathcal{L}^{-1}(3) \right) =  \mathcal{L}\left( 1 \right) = \sqrt{1}+2 = 3$
 	\end{tabular}
\end{center}
By our results from the mapping tables, we conclude that $\mathcal{L}^{-1}(w) = (w-2)^2$ is indeed the inverse function
of $\mathcal{L}$.

\end{prf}
\end{qstn}


\begin{qstn}
\begin{prf}
  Proceeding with mapping tables,
  \begin{center}
    \begin{tabular}{c|c}
  \text{$\mathcal{V}$} & \text{$\mathcal{T}^{-1}\left( \mathcal{T}(v) \right) $}\\\hline 
        $-1$ & $\mathcal{T}^{-1}\left( \mathcal{T}(-1) \right) =  \mathcal{T}^{-1}\left( 1 \right) = 2(1) / (1-3) = -1$\\
        $0$ & $\mathcal{T}^{-1}\left( \mathcal{T}(0) \right) =  \mathcal{T}^{-1}\left( 0 \right) = 2(0) / (0-3) = 0$\\
        $1$ & $\mathcal{T}^{-1}\left( \mathcal{T}(1) \right) =  \mathcal{T}^{-1}\left( -3 \right) = 2(-3) / (-3-3) = 1$\\
        $5$ & $\mathcal{T}^{-1}\left( \mathcal{T}(5) \right) =  \mathcal{T}^{-1}\left( 5 \right) = 2(5) / (5-3) = 5$\\
        $8$ & $\mathcal{T}^{-1}\left( \mathcal{T}(8) \right) =  \mathcal{T}^{-1}\left( 4 \right) = 2(4) / (4-3) = 8$\\
 	\end{tabular}

    \begin{tabular}{c|c}
  \text{$\mathcal{W}$} & \text{$\mathcal{T}\left( \mathcal{T}^{-1}(w) \right) $}\\\hline 
        $-3$ & $\mathcal{T}\left( \mathcal{T}^{-1}(-3) \right) =  \mathcal{T}\left( 1 \right) = 3(1) / (1-2) = -3$\\
        $0$ & $\mathcal{T}\left( \mathcal{T}^{-1}(0) \right) =  \mathcal{T}\left( 0 \right) = 3(0) / (0-2) = 0$\\
        $1$ & $\mathcal{T}\left( \mathcal{T}^{-1}(1) \right) =  \mathcal{T}\left( -1 \right) = 3(-1) / (-1-2) = 1$\\
        $4$ & $\mathcal{T}\left( \mathcal{T}^{-1}(4) \right) =  \mathcal{T}\left( 8 \right) = 3(8) / (8-2) = 4$\\
        $5$ & $\mathcal{T}\left( \mathcal{T}^{-1}(5) \right) =  \mathcal{T}\left( 5 \right) = 3(5) / (5-2) = 5$\\
 	\end{tabular}
\end{center}
By our results from the mapping tables, we conclude that $\mathcal{T}^{-1}(w) = \frac{2w}{w - 4}$ is indeed the inverse function
of $\mathcal{T}$.

\end{prf}
\end{qstn}


\begin{qstn}
 \begin{solution}
   I suspect that the formula is $F(a) = 2a + 4$. In order to check that it this is true, it is sufficient at check
   that the mapping table for  $F(a) = 2a + 4$ matches our mapping diagram.

\begin{center}

    \begin{tabular}{c|c}
  \text{$\mathcal{A}$} & \text{$F(a) = 2a + 4 $}\\\hline 
        $-8$ & $F(-8) = 2(-8) + 4 = -16 + 4 = -12$\\
        $-2$ & $F(-2) = 2(-2) + 4 = -4 + 4 = 0$\\
        $4$ & $F(4) = 2(4) + 4 = 8 + 4 = 12$\\
        $6$ & $F(6) = 2(6) + 4 = 12 + 4 = 16$
 	\end{tabular}
\end{center}
Since each element in $\mathcal{A}$ is mapped to the correct element in $\mathcal{B}$ with respect to the mapping
diagram, we conclude that $F(a) = 2a + 4$ is indeed the correct formula for the function.

I claim that $F^{-1}(b) = -2b$ is the inverse function of $F$.

\begin{prf}
 Its sufficient to check that $F^{-1}(b) = (b - 4) / 2$ satisfies the conditions in Definition 4.1. In order to check that it
 does, we can use mapping tables.

\begin{center}
  \begin{tabular}{c|c}
  \text{$\mathcal{A}$} & \text{$F^{-1}\left( F(a) \right) $}\\\hline 
    $-8$ & $F^{-1}\left( F(-8) \right) =  F^{-1}\left( -12 \right) = (-12-4) / 2 = -8$\\
    $-2$ & $F^{-1}\left( F(-2) \right) =  F^{-1}\left( 0 \right) = (0-4) / 2 = -2$\\
    $4$ & $F^{-1}\left( F(4) \right) =  F^{-1}\left( 12 \right) = (12-4) / 2 = 4$\\
    $6$ & $F^{-1}\left( F(6) \right) =  F^{-1}\left( 16 \right) = (16-4) / 2 = 6$
 	\end{tabular}

  \begin{tabular}{c|c}
  \text{$\mathcal{B}$} & \text{$F\left( F^{-1}(b) \right) $}\\\hline 
    $0$ & $F\left( F^{-1}(0) \right) =  F\left( -2 \right) = 2(-2) + 4 = 0$\\
    $-12$ & $F\left( F^{-1}(-12) \right) =  F\left( -8 \right) = 2(-8) + 4 = -12$\\
    $16$ & $F\left( F^{-1}(16) \right) =  F\left( 6 \right) = 2(6) + 4 = 16$\\
    $12$ & $F\left( F^{-1}(12) \right) =  F\left( 4 \right) = 2(4) + 4 = 12$
 	\end{tabular}
\end{center}
By our results from the mapping tables, we conclude that the claim was correct and that \\$F^{-1}(b) = (b-4) / 2$ is indeed
the inverse function of $F$.

\end{prf}

\end{solution} 
\end{qstn}


\begin{qstn}
  \begin{solution}
    Well first we can talk about how we gain the ability to determine the domain of a function strictly based on 
    knowledge of the inverse function and the range of the function. For a more more applicable example, suppose we have a 
    function that goes from Celsius to Fahrenheit. Then of course at times we would also be interested in going from Fahrenheit to Celsius,
    this would require knowledge of the inverse function. (This would make a great test question\ldots).
  \end{solution}
\end{qstn}

\newpage

\begin{qstn}
  \begin{solution}
  $f$ is \textbf{not} invertible.
  \begin{prf}
    Its easy to show that $f$ fails to be injective, for example note that,
    \begin{align*}
      f(-2) &= \left| - 2\right| = 2\\
      f(2) &= \left|2\right| = 2
    .\end{align*}
  Since $f$ fails to be injective, it fails to be invertible.
  \end{prf}     
  \end{solution}
\end{qstn}

\begin{qstn}
  \begin{solution}
  Since $G$ is invertible, we know that its both surjective and injective. Both of these properties assert that
  each element in $\mathcal{B}$ is mapped to from a \textbf{unique} element in $\mathcal{A}$. Since we are given the
  inverse function $G^{-1}$, we can simply determine the output of each element $b \in \mathcal{B}$ to reconstruct $\mathcal{A}$.
  We can use mapping tables to help with this,

\begin{center}
  \begin{tabular}{c|c}
  \text{$\mathcal{B}$} & \text{$G^{-1}(b)$}\\\hline 
    $-2$ & $G^{-1}(-2) =  -2 - 2 = -4$\\
    $-1$ & $G^{-1}(-1) =  -1 - 2 = -3$\\
    $0$ & $G^{-1}(0) =  0 - 2 = -2$\\
    $3$ & $G^{-1}(3) =  3 - 2 = 1$\\
    $7$ & $G^{-1}(7) =  7 - 2 = 5$
 	\end{tabular}
 \end{center}
 And hence $\mathcal{A} = \{-4,-3,-2,1,5\} $.
    
  \end{solution}

\end{qstn}


\begin{qstn}
  \begin{solution} \texttt{ }
  \begin{enumerate}[label=(\alph*)]
    \item $\mathbf{S} + \mathbf{T} + 0 = 00101 + 1101 + 0 = 0010111010$.
    \item $1 + \mathbf{T} + 1 = 1 + 1101 + 1 = 111011$.
    \item $\epsilon + \mathbf{T} + \mathbf{S} = \epsilon + 1101 + 00101 = 110100101$.
    \item $\mathbf{T} + \mathbf{T} = 1101 + 1101 = 11011101$.
  \end{enumerate}
    
  \end{solution}
\end{qstn}


\begin{qstn}
  This is \textbf{false}. An easy counter example is to let $\mathbf{S} = 1$ and $T = 0$, then
  \begin{align*}
    \mathbf{S} + \mathbf{T} &= 1 + 0 = 10\\
        \mathbf{T} + \mathbf{S} &= 0 + 1 = 01
  .\end{align*} Clearly $\mathbf{S} + \mathbf{T} \neq \mathbf{T} + \mathbf{S}$, hence this is not always true. (Weird right?) 
\end{qstn}

\begin{qstn}
  \begin{solution} \texttt{  }
    \begin{enumerate}[label=(\alph*)]
      \item $\mathbf{R} = \vb a_1 \vb a_3 \vb a_5 = 101$.
      \item $\mathbf{Z} = \vb a_1 \vb a_2 = 11$.
      \item $\mathbf{X} = \vb a_2 \vb a_4 = 11$.
    \end{enumerate}
  \end{solution}
\end{qstn}

\newpage

\begin{qstn}
  \begin{solution}
    I claim that $\phi^{-1}(\mathbf{R}) = \vb r_2 \vb r_3$ is the inverse function of $\phi$.
    \begin{prf}
 Its sufficient to check that $\phi^{-1}(\mathbf{R}) = \vb r_2 \vb r_3$ satisfies the conditions in Definition 4.1. In order to check that it
 does, we can use mapping tables.

\begin{center}
  \begin{tabular}{c|c}
  \text{$\EuScript{S}$} & \text{$\phi^{-1}\left( \phi(\mathbf{S}) \right) $}\\\hline 
    $11$ & $\phi^{-1}\left( \phi(11) \right) =  \phi^{-1}\left( 11101 \right) = 11$\\
    $01$ & $\phi^{-1}\left( \phi(01) \right) =  \phi^{-1}\left( 10101 \right) = 01$\\
    $10$ & $\phi^{-1}\left( \phi(10) \right) =  \phi^{-1}\left( 11001 \right) = 10$\\
    $00$ & $\phi^{-1}\left( \phi(00) \right) =  \phi^{-1}\left( 10001 \right) = 00$
 	\end{tabular}

  \begin{tabular}{c|c}
  \text{$\EuScript{R}$} & \text{$\phi\left( \phi^{-1}(\mathbf{R}) \right) $}\\\hline 
    $10001$ & $\phi\left( \phi^{-1}(10001) \right) =  \phi\left( 00 \right) = 1 + 00 + 0 + 1 = 10001$\\
    $10101$ & $\phi\left( \phi^{-1}(10101) \right) =  \phi\left( 01 \right) = 1 + 01 + 0 + 1 = 10101$\\
    $11001$ & $\phi\left( \phi^{-1}(11001) \right) =  \phi\left( 10 \right) = 1 + 10 + 0 + 1 = 11001$\\
    $11101$ & $\phi\left( \phi^{-1}(11101) \right) =  \phi\left( 11 \right) = 1 + 11 + 0 + 1 = 11101$
 	\end{tabular}
\end{center}
By our results from the mapping tables, we conclude that the claim was correct and that 
\\$\phi^{-1}(\mathbf{R}) = \vb r_2 \vb r_3$ is indeed the inverse function of $\phi$.

    \end{prf}
  \end{solution}
\end{qstn}

\end{document}



























