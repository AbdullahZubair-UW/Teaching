% Prepared by Calvin Kent
%
% Assignment Template v19.02
%
%%% 20xx0x/MATHxxx/Crowdmark/Ax
%
\documentclass[12pt]{article} %
\usepackage{amsthm}
\usepackage{CKpreamble}
\usepackage{CKassignment}
\usepackage{mdframed}
\usepackage{euscript}
\usepackage{tikz}
\usepackage{pgfplots}

%
\begin{document}
	\pagenumbering{arabic}
	% Start of class settings ...
	\renewcommand*{\coursecode}{MATH 235} % renew course code
	\renewcommand*{\assgnnumber}{Assignment 1} % renew assignment number
	\renewcommand*{\submdate}{September 14, 2021} % renew the date
	\renewcommand*{\studentfname}{Abdullah} % Student first name
	\renewcommand*{\studentlname}{Zubair} % Student last name
    \renewcommand*{\proofname}{Proof:}
	% \renewcommand*{\studentnum}{20836288} % Student number

	\renewcommand\qedsymbol{$\blacksquare$}
	\setfigpath
	% End of class settings	
	% \pagestyle{crowdmark}
	\newgeometry{left=18mm, right=18mm, top=22mm, bottom=22mm} % page is set to default values
	\fancyhfoffset[L,O]{0pt} % header orientation fixed
	% End of class settings
	%%% Note to user:
	% CTRL + F <CHANGE ME:> (without the angular brackets) in CKpreamble to specify graphics paths accordingly.
	% The command \circled[]{} accepts one optional and one mandatory argument.
	% Optional argument is for the size of the circle and mandatory argument is for its contents.
	% \circled{A} produces circled A, with size drawn for letter A. \circled[TT]{A} produces circled A with size drawn for TT.
	% https://github.com/CalvinKent/My-LaTeX
	%%%

	%%%%%%%%%%%%%%%%%%%%%%%%%%%%%%%%%%%%%%%%%%%%%%%%%%%%%%%%%%%%%%%%%%%%%%%%%%%%%%%%%%%%%%%%%%%%%%%%%%%%%%%%%%%%%%%%%%
	%%%                        CUSTOM MACRO VIM-TEX                                                      (Word Wrap->
	%%       call IMAP('NOM', '\nomenclature{}', 'tex')               

	%%%%%%%%%%%%%%%%%%%%%%%%%%%%%%%%%%%%%%%%%%%%%%%%%%%%%%%%%%%%%%%%%%%%%%%%%%%%%%%%%%%%%%%%%%%%%%%%%%%%%%%%%%%%%%%%%%

	% Crowdmark assignment start
	% qnumber, qname, points

\begin{center}
	\textbf{\underline{\Huge{Lecture 7 - Homework}}}
\end{center}

\textbf{Note: }In the textbook questions, they will ask you to state the restrictions on the variables. \textbf{You
don't have to do this.}

\begin{qstn}
  \textbf{Textbook, Pg. 84\,\,\, Q6. (Skip f))}
\end{qstn}

\begin{qstn}
  \textbf{Textbook, Pg. 94 \,\,\,Q1. b), d)}
\end{qstn}

\begin{qstn}
  \textbf{Textbook, Pg. 94 \,\,\,Q2. a), c)}
\end{qstn}

\begin{qstn}
  \textbf{Textbook, Pg. 94 \,\,\,Q3.}
\end{qstn}

\begin{qstn}
  \textbf{Textbook, Pg. 94 \,\,\,Q4. a), b), d).}
\end{qstn}

\begin{qstn}
  \textbf{Textbook, Pg. 94 \,\,\,Q5.}
\end{qstn}


\begin{qstn}
  \textbf{Textbook, Pg. 94 \,\,\,Q6.}
\end{qstn}


\begin{qstn}
  \textbf{Textbook, Pg. 94 \,\,\,Q8.}
\end{qstn}

\begin{qstn}
  \textbf{Textbook, Pg. 94 \,\,\,Q9.}
\end{qstn}

\begin{qstn}
  \textbf{Textbook, Pg. 95 \,\,\,Q13.}
\end{qstn}

\section{Solutions}
Refer to the textbook for all solutions. Ask me for clarification if needed.

\end{document}































