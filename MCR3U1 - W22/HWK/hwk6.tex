% Prepared by Calvin Kent
%
% Assignment Template v19.02
%
%%% 20xx0x/MATHxxx/Crowdmark/Ax
%
\documentclass[12pt]{article} %
\usepackage{amsthm}
\usepackage{CKpreamble}
\usepackage{CKassignment}
\usepackage{mdframed}
\usepackage{euscript}
\usepackage{tikz}
\usepackage{pgfplots}

%
\begin{document}
	\pagenumbering{arabic}
	% Start of class settings ...
	\renewcommand*{\coursecode}{MATH 235} % renew course code
	\renewcommand*{\assgnnumber}{Assignment 1} % renew assignment number
	\renewcommand*{\submdate}{September 14, 2021} % renew the date
	\renewcommand*{\studentfname}{Abdullah} % Student first name
	\renewcommand*{\studentlname}{Zubair} % Student last name
    \renewcommand*{\proofname}{Proof:}
	% \renewcommand*{\studentnum}{20836288} % Student number

	\renewcommand\qedsymbol{$\blacksquare$}
	\setfigpath
	% End of class settings	
	% \pagestyle{crowdmark}
	\newgeometry{left=18mm, right=18mm, top=22mm, bottom=22mm} % page is set to default values
	\fancyhfoffset[L,O]{0pt} % header orientation fixed
	% End of class settings
	%%% Note to user:
	% CTRL + F <CHANGE ME:> (without the angular brackets) in CKpreamble to specify graphics paths accordingly.
	% The command \circled[]{} accepts one optional and one mandatory argument.
	% Optional argument is for the size of the circle and mandatory argument is for its contents.
	% \circled{A} produces circled A, with size drawn for letter A. \circled[TT]{A} produces circled A with size drawn for TT.
	% https://github.com/CalvinKent/My-LaTeX
	%%%

	%%%%%%%%%%%%%%%%%%%%%%%%%%%%%%%%%%%%%%%%%%%%%%%%%%%%%%%%%%%%%%%%%%%%%%%%%%%%%%%%%%%%%%%%%%%%%%%%%%%%%%%%%%%%%%%%%%
	%%%                        CUSTOM MACRO VIM-TEX                                                      (Word Wrap->
	%%       call IMAP('NOM', '\nomenclature{}', 'tex')               

	%%%%%%%%%%%%%%%%%%%%%%%%%%%%%%%%%%%%%%%%%%%%%%%%%%%%%%%%%%%%%%%%%%%%%%%%%%%%%%%%%%%%%%%%%%%%%%%%%%%%%%%%%%%%%%%%%%

	% Crowdmark assignment start
	% qnumber, qname, points

\begin{center}
	\textbf{\underline{\Huge{Lecture 6 - Homework}}}
\end{center}
\begin{qstn}
  Let $x \in \Z$ and let $p$ be a prime number. Determine all possible values of $\gcd(x,p)$.
\end{qstn}

\begin{qstn}
  Which of the following numbers are prime?\\
  \textbf{Hint: }Theorem 6.2 should help speeding up the process.
  \begin{itemize}
    \item 231.
    \item 560.
    \item 433.
    \item 79.
  \end{itemize}
\end{qstn}

\begin{qstn}
  Preform a prime factorization of the following natural numbers.
  \begin{enumerate}[label=(\alph*)]
    \item 363.
    \item 237.
    \item 688.
    \item 732.
  \end{enumerate}
\end{qstn}

\begin{qstn}
  Simplify the following exponential expressions.\\ \textbf{(Leave answers with positive exponents)}
  \begin{enumerate}[label=(\alph*)]
    \item $\left( x^{-4} \right) / \left( y^{-5} \right)^{-2}  $.
    \item $\left( z^2y^{-6}x^{-7} \right)^{-3}$.
    \item $\left( 16a^3b^{-4}z^{-12} \right)  / \left( 4a^{-3}b^{-2} \right)^{3}$.
    \item $16^{\frac{3}{2}}$.
  \end{enumerate}
\end{qstn}

\begin{qstn}
  \textbf{Textbook, Pg. 39\,\,\, Q1. b),d),f)}
\end{qstn}

\begin{qstn}
  \textbf{Textbook, Pg. 39\,\,\, Q2.}
\end{qstn}

\begin{qstn}
  \textbf{Textbook, Pg. 39\,\,\, Q3. b),d),f)}
\end{qstn}

\begin{qstn}
  \textbf{Textbook, Pg. 39\,\,\, Q4. b),d),f)}
\end{qstn}

\begin{qstn}
  \textbf{Textbook, Pg. 39\,\,\, Q5. b),d),f)}
\end{qstn}

\begin{qstn}
  \textbf{Textbook, Pg. 39\,\,\, Q6. a),c),e),f)}
\end{qstn}

\begin{qstn}
  \textbf{Textbook, Pg. 39\,\,\, Q7. b),d),f)}
\end{qstn}

\begin{qstn}
  \textbf{Textbook, Pg. 39\,\,\, Q8.}
\end{qstn}

\end{document}































