% Prepared by Calvin Kent
\documentclass[12pt,oneside]{book} %
\usepackage{CKpreamble}
\usepackage{CKlecture}
\usepackage{mdframed}
\usepackage{import}
\usepackage{pdfpages}
\usepackage{euscript}
\usepackage{transparent}
\usepackage{xcolor}
\usepackage{tasks}

\newcommand{\incfig}[2][1]{%
    \def\svgwidth{#1\columnwidth}
    \import{./figures/}{#2.pdf_tex}
}


\newcounter{step}[section]
\newenvironment{step}[1][]
{\refstepcounter{step} \textit{Step #1.}}

\newcounter{Rule}[section]
\newenvironment{Rule}[1][]
{\refstepcounter{Rule} \textit{Rule #1.}}


\pdfsuppresswarningpagegroup=1

%
\renewcommand*{\doctitle}{Class Based Lecture Notes}
\makeatletter\patchcmd{\chapter}{\if@openright\cleardoublepage\else\clearpage\fi}{}{}{}\makeatother % only used in class based
\begin{document}
	% Start of Class settings
	\renewcommand*{\term}{Term 2} % Term
	\renewcommand*{\coursecode}{MCR3U} % Course code
	\renewcommand*{\coursename}{Course Name} % Full course name
	\renewcommand*{\thelecnum}{7} % Lecture number
	\renewcommand*{\profname}{Prof Name} % Prof Name
	\renewcommand*{\colink}{http://www.student.math.uwaterloo.ca/~c2kent} % Course outline link
	% End of Class settings
	\clearpage
	\pagenumbering{arabic}
	\pagestyle{classlecture}
	%%% Note to user: CTRL + F <CHANGE ME:> (without the angular brackets) in CKpreamble to specify graphics paths accordingly.
	%%% If a new chapter was started in the middle of a lecture, \fix chap{Second Chapter} must be used immediately above the next lecture.
	% Course notes start
\setchap{7}{Rational Expressions}
\begin{lec}{January 2022}
	\chapter{\chapname\chaplec}

  \begin{defn}
      Let $a,b,c \in \R$. A \textbf{polynomial} $P(x)$ is an expression that has form,
        \[
            P(x) = ax^2 + bx + c
        .\] 
  \end{defn}

  \begin{rem}
      Really polynomials are not limited to second degree, however they will be for this course.
  \end{rem}

  \begin{ex}
    Examples of polynomials,
    \begin{itemize}
      \item $Q(x) = 2x^2 + 3x + 1$.
      \item $P(x) = 3$.
      \item $R(x) = 5x - 3$.
    \end{itemize}
  \end{ex}

  \begin{defn}
      A \textbf{rational expression} is a ratio of two polynomials, $P(x)$ and $Q(x)$, 
       \[
         \frac{P(x)}{Q(x)}
      .\] 
  \end{defn}

  \section{Multiplication and Division.}
  Suppose we are given a product of two rational expressions,
  \[
        \frac{A(x)}{B(x)} \times \frac{C(x)}{D(x)}
  .\] 
  Follow these steps to simply,

  \begin{step}[1]
    Factor all of $A(x),B(x),C(x),D(x)$.
  \end{step}
  
  \begin{step}[2]
    Rewrite as a single fraction \textbf{in factored form},
    \[
        \frac{A(x)C(x)}{B(x)D(x)}
    .\] 
  \end{step}

  \begin{step}[2]
    Cancel all common factors, reduce any coefficients with common factors.
  \end{step}

  If instead we are faced with a division of two rational expressions, then simply use the reciprocal to change to
  the equivalent product,
  \[
        \frac{A(x)}{B(x)} \div \frac{C(x)}{D(x)} = \frac{A(x)}{B(x)} \times \frac{D(x)}{C(x)}
  .\] 

  \newpage

  \begin{ex} Simply the following,
    \begin{tasks}(2)
      \task \[
            \frac{x^2 + 7x + 12}{x^2 + 5x + 4}
      .\] 

      \task
      \begin{align*}
        \frac{x^2 - 3x - 18}{2x^2 + 5x - 3} \tag{**}
      \end{align*}

    \end{tasks}
  \end{ex}

  \begin{ex}
    Simply the following, 
    \begin{tasks}(2)
      \task\[
            \frac{x^2+10x + 21}{x + 3} \times \frac{x + 3}{x^2 + 9x + 14}
      .\] 

      \task
      \begin{align*}
          \frac{2x^2 - 8x}{x^2 - 3x - 10} \times \frac{4x^2}{x^2 - 9x + 20} \tag{**}
      \end{align*}

    \end{tasks}

  \end{ex}

  \begin{ex}
    Simply the following, 
    \begin{tasks}(2)
      \task\[
            \frac{16x^5}{x^2 - 2x + 1} \div \frac{4x^{3}}{x^2 - 1}
      .\] 

      \task
      \begin{align*}
          \frac{x^2 - x}{x^2 + x - 2} \div \frac{4x}{x^2 + 3x + 2} \tag{**}
      \end{align*}

    \end{tasks}

  \end{ex}

  \section{GCD's of polynomials}
  We can extend the concept of the $\gcd$ to polynomials as  well. Generally speaking, given polynomials $P(x)$ and
  $Q(x)$ in \textbf{factored form}, the value of $\gcd \left( P(x), Q(x) \right) $ can be obtained trivially by
  analysing the greatest common product of factors.

  \begin{ex}
    Let $P(x) = x^2 - 5x + 6$ and  $Q(x) = 2x^2 - 14x + 24$. Determine $\gcd \left( P(x), Q(x) \right) $.
  \end{ex}

  \begin{ex}
    Let $P(x) = 5x-10$ and  $R(x) = 10x$. Determine $\gcd \left( P(x), R(x) \right) $. \hfill (**)
  \end{ex}

  \begin{ex}
    Let $A(x) = x^2 - 4x + 3$ and  $B(x) = 2x^2 - 7x + 3$. Determine $\gcd \left( A(x), B(x) \right) $.\hfill (**)
  \end{ex}

  \section{Addition and Subtraction}
  Suppose we are given a sum or difference of two rational expressions,
  \[
        \frac{A(x)}{B(x)} \pm \frac{C(x)}{D(x)}
  .\] 

  The arithmetic here is actually \textbf{very similar} to addition of regular fractions. Follow these steps to simplify,
  \newpage

  \begin{step}[1]
    Factor both $B(x)$ and $D(x)$.
  \end{step}
  
  \begin{step}[2]
    Determine the LCD, 
    \[
        L(x) = \frac{B(x)\cdot D(x)}{\gcd \left( B(x),D(x) \right) }
    .\] 
  \end{step}
  
  \begin{step}[3]
    Determine the \textit{missing factors},
    \[
          R(x) \hspace*{1cm} Q(x)
    .\] 
  \end{step}

  \begin{step}[4]
    Rewrite the fraction with the LCD and simplify,
    \[
        \frac{A(x)\cdot \textcolor{black}{R(x)} \pm  C(x) \cdot \textcolor{black}{Q(x)}}{\textcolor{black}{L(x)}}
    .\] 
  \end{step}

  \begin{ex}
    Simply the following,
    \begin{enumerate}[label=(\alph*)]
      \item \[
          \frac{1}{5x} + \frac{1}{2x}
      .\] 

      \item \[
          \frac{x}{x - 1} - \frac{x + 1}{x + 2}
      .\] 
      
    \item
    \begin{align*}
        \frac{x}{x^2 + 2x + 1} + \frac{2}{x + 1} \tag{**}
    .\end{align*}

    \item 
      \begin{align*}
      \frac{x + 9}{x^2 + 2x - 48} - \frac{x - 9}{x^2 - x - 30} \tag{**}
      \end{align*}
   

    \end{enumerate}
  \end{ex}





  

	\end{lec}

\end{document}




















































