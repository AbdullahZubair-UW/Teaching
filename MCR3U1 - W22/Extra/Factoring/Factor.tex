% Prepared by Calvin Kent
%
% Assignment Template v19.02
%
%%% 20xx0x/MATHxxx/Crowdmark/Ax
%
\documentclass[12pt]{article} %
\usepackage{amsthm}
\usepackage{CKpreamble}
\usepackage{CKassignment}
\usepackage{mdframed}
\usepackage{import}
\usepackage{pdfpages}
\usepackage{transparent}
\usepackage{xcolor}



\newcommand{\incfig}[2][1]{%
    \def\svgwidth{#1\columnwidth}
    \import{./figures/}{#2.pdf_tex}
}

\pdfsuppresswarningpagegroup=1

\newcounter{step}[section]
\newenvironment{step}[1][]
{\refstepcounter{step} \textbf{Step #1.}}


%
\begin{document}
	\pagenumbering{arabic}
	% Start of class settings ...
	\renewcommand*{\coursecode}{MATH 235} % renew course code
	\renewcommand*{\assgnnumber}{Assignment 1} % renew assignment number
	\renewcommand*{\submdate}{September 14, 2021} % renew the date
	\renewcommand*{\studentfname}{Abdullah} % Student first name
	\renewcommand*{\studentlname}{Zubair} % Student last name
    \renewcommand*{\proofname}{Proof:}
	% \renewcommand*{\studentnum}{20836288} % Student number

	\renewcommand\qedsymbol{$\blacksquare$}
	\setfigpath
	% End of class settings	
	% \pagestyle{crowdmark}
	\newgeometry{left=18mm, right=18mm, top=22mm, bottom=22mm} % page is set to default values
	\fancyhfoffset[L,O]{0pt} % header orientation fixed
	% End of class settings
	%%% Note to user:
	% CTRL + F <CHANGE ME:> (without the angular brackets) in CKpreamble to specify graphics paths accordingly.
	% The command \circled[]{} accepts one optional and one mandatory argument.
	% Optional argument is for the size of the circle and mandatory argument is for its contents.
	% \circled{A} produces circled A, with size drawn for letter A. \circled[TT]{A} produces circled A with size drawn for TT.
	% https://github.com/CalvinKent/My-LaTeX
	%%%

	%%%%%%%%%%%%%%%%%%%%%%%%%%%%%%%%%%%%%%%%%%%%%%%%%%%%%%%%%%%%%%%%%%%%%%%%%%%%%%%
	%%%                        CUSTOM MACRO VIM-TEX                             %%%
	%%       call IMAP('NOM', '\nomenclature{}', 'tex')               

	%%%%%%%%%%%%%%%%%%%%%%%%%%%%%%%%%%%%%%%%%%%%%%%%%%%%%%%%%%%%%%%%%%%%%%%%%%%%%%%

	% Crowdmark assignment start
	% qnumber, qname, qpoints

\begin{center}
		\Huge{\underline{\textbf{Review on Factoring}}}
\end{center}

\section{Difference of Squares factoring $(f(x) = ax^2 + c)$}
\textbf{IF} $c < 0$ then you can factor the quadratic as follows,
\[
    f(x) = \left(\sqrt{a} x + \sqrt{\left|c\right|} \right)\left(\sqrt{a} x - \sqrt{\left|c\right|} \right)
.\] 

\textbf{ELSE IF} $a < 0$ then you can factor the quadratic as follows,
\[
    f(x) = -\left(\sqrt{\left|a\right|} x + \sqrt{c} \right)\left(\sqrt{\left|a\right|} x - \sqrt{c} \right)
.\] 

\section{Simple Factoring $(f(x) = x^2 + bx + c)$}
\textbf{IF} $a = 1$, then the factoring is known as 'simple' because the procedure is quite straight forward. As such we call
quadratics with $a = 1$ simple trinomials. So assume that we are working
with a quadratic function with $a = 1$,
 \[
        f(x) = x^2 + bx + c
.\]
\begin{step}[1] Find the integers $p,q$ such that,
  \begin{align*}
    p + q &= b\\
    p \cdot  q & =c
  .\end{align*}
\end{step}

\begin{step}[2]
  Factor the quadratic as follows,
  \[
        f(x) = (x + q)(x + p)
  .\] 
\end{step}

\section{Non-Simple Factoring (FIXED) $(f(x) = ax^2 + bx + c)$}
\textbf{If }$a \neq 1$ then proceed with the following steps,


\begin{step}[1] \textbf{IF} $b$ \textbf{and} $c$ are divisible by $a$, then factor $a$ out of the polynomial. Then apply simple
  factoring to the polynomial leftover and your done!
\end{step}

\begin{step}[2]
  \textbf{ELSE}, find the integers $p,q$ such that,
  \begin{align*}
    p + q &= b\\
    p\cdot q &= ac
  .\end{align*}
\end{step}


\begin{step}[3]
Then ,
\begin{itemize}
  \item Find the $\gcd \left(\left| a\right| , \left| p\right|\right)$, let that integer be $t$.
  \item Find the $\gcd \left(\left| q\right| , \left| c\right|\right)$, let that integer be $k$.
\end{itemize}

\end{step}

\begin{step}[4]
  \textbf{IF $a \cdot q > 0$}, then complete the factorization by writing,
  \[
        f(x) = \left( tx + k \right)  \left( \frac{a}{t}x + \frac{p}{t} \right) 
  .\] 
\end{step}

\begin{step}[5]
  \textbf{ELSE IF $a \cdot q < 0$}, then complete the factorization by writing,
  \[
        f(x) = \left( tx - k \right)  \left( \frac{a}{t}x + \frac{p}{t} \right) 
  .\] 
\end{step}

\textbf{\underline{\Large{Practice Problems:}}}\\
Completely factor the following quadratics. (I will typeset the solutions to these problems soon, just ask me in class if you feel
uncertain about an answer).
\begin{enumerate}
  \item $f(x) = x^2 + 16x + 64$
  \item $g(x) = -x^2 - 15x - 26$
  \item $f(x) = 9x^2 - 49$
  \item $r(x) = 6x^2 + x - 2$
  \item $g(x) = -9x^2 + 6x - 1$
  \item $Y(x) = 21x^2 + 27x + 6$
  \item $I(x) = 12x^2 - 26x - 16$ \hspace*{1cm}(Make life easy by first factoring out a $4$)
  \item $A(x) = 2x^2 + 7x + 5$
  \item $f(x) = 4x^2 + 16x + 16$
  \item $r(x) = -x^2 + 13x - 22$
  \item $f(x) = 121 - 100x^2$
  \item $f(x) = 7x^2 + 15 - 18$
  \item $L(x) = 4 - 9x^2$
  \item $\Gamma(x) = 12x^2 + 17x + 6$
  \item $\Lambda(x) = -2x^2 + 8x  + 24$
  \item $\Phi(x) = -5x^2 + 2x + 7$
  \item $\Theta(x) = -3x^2 - 9x + 30$
  \item $\xi(x) = x^2 + 4x - 77$
  \item $\alpha(x) = 2x^2 - 3x - 14$
  \item $S(x) = 6x^2 + 5x - 6$
\end{enumerate}



\end{document}






































































