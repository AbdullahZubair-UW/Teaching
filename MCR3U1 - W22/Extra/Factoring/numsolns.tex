% Prepared by Calvin Kent
%
% Assignment Template v19.02
%
%%% 20xx0x/MATHxxx/Crowdmark/Ax
%
\documentclass[12pt]{article} %
\usepackage{amsthm}
\usepackage{CKpreamble}
\usepackage{CKassignment}
\usepackage{mdframed}
\usepackage{import}
\usepackage{pdfpages}
\usepackage{transparent}
\usepackage{xcolor}



\newcommand{\incfig}[2][1]{%
    \def\svgwidth{#1\columnwidth}
    \import{./figures/}{#2.pdf_tex}
}

\pdfsuppresswarningpagegroup=1

\newcounter{step}[section]
\newenvironment{step}[1][]
{\refstepcounter{step} \textbf{Step #1.}}


%
\begin{document}
	\pagenumbering{arabic}
	% Start of class settings ...
	\renewcommand*{\coursecode}{MATH 235} % renew course code
	\renewcommand*{\assgnnumber}{Assignment 1} % renew assignment number
	\renewcommand*{\submdate}{September 14, 2021} % renew the date
	\renewcommand*{\studentfname}{Abdullah} % Student first name
	\renewcommand*{\studentlname}{Zubair} % Student last name
    \renewcommand*{\proofname}{Proof:}
	% \renewcommand*{\studentnum}{20836288} % Student number

	\renewcommand\qedsymbol{$\blacksquare$}
	\setfigpath
	% End of class settings	
	% \pagestyle{crowdmark}
	\newgeometry{left=18mm, right=18mm, top=22mm, bottom=22mm} % page is set to default values
	\fancyhfoffset[L,O]{0pt} % header orientation fixed
	% End of class settings
	%%% Note to user:
	% CTRL + F <CHANGE ME:> (without the angular brackets) in CKpreamble to specify graphics paths accordingly.
	% The command \circled[]{} accepts one optional and one mandatory argument.
	% Optional argument is for the size of the circle and mandatory argument is for its contents.
	% \circled{A} produces circled A, with size drawn for letter A. \circled[TT]{A} produces circled A with size drawn for TT.
	% https://github.com/CalvinKent/My-LaTeX
	%%%

	%%%%%%%%%%%%%%%%%%%%%%%%%%%%%%%%%%%%%%%%%%%%%%%%%%%%%%%%%%%%%%%%%%%%%%%%%%%%%%%
	%%%                        CUSTOM MACRO VIM-TEX                             %%%
	%%       call IMAP('NOM', '\nomenclature{}', 'tex')               

	%%%%%%%%%%%%%%%%%%%%%%%%%%%%%%%%%%%%%%%%%%%%%%%%%%%%%%%%%%%%%%%%%%%%%%%%%%%%%%%

	% Crowdmark assignment start
	% qnumber, qname, qpoints

\begin{center}
		\Huge{\underline{\textbf{How to find the number of solutions}}}
\end{center}
Given a quadratic,
\[
      f(x) = ax^2 + bx + c
\] we are sometimes curious about how many solutions the quadratic will have, we can answer it by determining the discriminant.
The \textbf{discriminant} is the following number,
\[
      d = b^2 - 4ac
\]After finding the discriminant, you can make the following conclusions,
\begin{itemize}
  \item \textbf{IF} $d = 0$, the quadratic will have \textbf{one} solution. (Called a double-root)
  \item \textbf{ELSE IF} $d > 0$, the quadratic will have \textbf{two} different solutions.
  \item \textbf{ELSE IF} $d < 0$, the quadratic will have \textbf{NO} solutions. 
\end{itemize}

\textbf{\underline{\Large{Practice Problems:}}}
\begin{qstn}
  Determine the number of solutions to the following functions,
  \begin{enumerate}[label=(\alph*)]
    \item $f(x) = x^2 + 1$
    \item $g(x) = x^2 + 2x + 1$
    \item $f(x) = -3x^2 + x + 1$
    \item $r(x) = 2x^2 - 12x + 18$
  \end{enumerate}
\end{qstn}

\begin{qstn}
  \textbf{Textbook, Pg 49} Question 6 a),b),c).
\end{qstn}





























\end{document}
















































