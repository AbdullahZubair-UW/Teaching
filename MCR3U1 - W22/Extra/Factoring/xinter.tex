
% Prepared by Calvin Kent
%
% Assignment Template v19.02
%
%%% 20xx0x/MATHxxx/Crowdmark/Ax
%
\documentclass[12pt]{article} %
\usepackage{amsthm}
\usepackage{CKpreamble}
\usepackage{CKassignment}
\usepackage{mdframed}
\usepackage{import}
\usepackage{pdfpages}
\usepackage{transparent}
\usepackage{xcolor}



\newcommand{\incfig}[2][1]{%
    \def\svgwidth{#1\columnwidth}
    \import{./figures/}{#2.pdf_tex}
}

\pdfsuppresswarningpagegroup=1

\newcounter{step}[section]
\newenvironment{step}[1][]
{\refstepcounter{step} \textbf{Step #1.}}


%
\begin{document}
	\pagenumbering{arabic}
	% Start of class settings ...
	\renewcommand*{\coursecode}{MATH 235} % renew course code
	\renewcommand*{\assgnnumber}{Assignment 1} % renew assignment number
	\renewcommand*{\submdate}{September 14, 2021} % renew the date
	\renewcommand*{\studentfname}{Abdullah} % Student first name
	\renewcommand*{\studentlname}{Zubair} % Student last name
    \renewcommand*{\proofname}{Proof:}
	% \renewcommand*{\studentnum}{20836288} % Student number

	\renewcommand\qedsymbol{$\blacksquare$}
	\setfigpath
	% End of class settings	
	% \pagestyle{crowdmark}
	\newgeometry{left=18mm, right=18mm, top=22mm, bottom=22mm} % page is set to default values
	\fancyhfoffset[L,O]{0pt} % header orientation fixed
	% End of class settings
	%%% Note to user:
	% CTRL + F <CHANGE ME:> (without the angular brackets) in CKpreamble to specify graphics paths accordingly.
	% The command \circled[]{} accepts one optional and one mandatory argument.
	% Optional argument is for the size of the circle and mandatory argument is for its contents.
	% \circled{A} produces circled A, with size drawn for letter A. \circled[TT]{A} produces circled A with size drawn for TT.
	% https://github.com/CalvinKent/My-LaTeX
	%%%

	%%%%%%%%%%%%%%%%%%%%%%%%%%%%%%%%%%%%%%%%%%%%%%%%%%%%%%%%%%%%%%%%%%%%%%%%%%%%%%%
	%%%                        CUSTOM MACRO VIM-TEX                             %%%
	%%       call IMAP('NOM', '\nomenclature{}', 'tex')               

	%%%%%%%%%%%%%%%%%%%%%%%%%%%%%%%%%%%%%%%%%%%%%%%%%%%%%%%%%%%%%%%%%%%%%%%%%%%%%%%

	% Crowdmark assignment start
	% qnumber, qname, qpoints

\begin{center}
		\Huge{\underline{\textbf{How to find the x-intercepts (FIXED)}}}
\end{center}
\begin{step}[1]
  Try to factor the quadratic by either,
  \begin{itemize}
    \item Difference of Squares factoring.
    \item Simple factoring.
    \item Non-Simple factoring.
  \end{itemize}
\end{step}

\begin{step}[2]
  \textbf{IF} the Difference of squares factoring worked, then the x-intercepts will be,
  \[
          x_1 = -\frac{\sqrt{\left|c\right|} }{\sqrt{\left|a\right|} }  ,\,\,\,\,\,\, x_2 = \frac{\sqrt{\left|c\right|} }{\sqrt{\left|a\right|} } 
  .\] 

\end{step}

\begin{step}[3]
 \textbf{ELSE IF} Simple factoring worked, then the x-intercepts will be,
  \[
          x_1 = -q  ,\,\,\,\,\,\, x_2 = -p 
  .\] 
\end{step}

\begin{step}[4]
  \textbf{ELSE IF} Non-Simple factoring worked, 
  \begin{itemize}
    \item \textbf{IF $a \cdot q > 0$}, then the x-intercepts will be,
  \[
          x_1 = -\frac{k}{t}  ,\,\,\,\,\,\, x_2 = -\frac{p}{a} 
  .\]

    \item \textbf{ELSE IF $a \cdot q < 0$}, then the x-intercepts will be,
    \[
          x_1 = \frac{k}{t}  ,\,\,\,\,\,\, x_2 = -\frac{p}{a} 
    .\]
  \end{itemize}
\end{step}


\begin{step}[5]
  \textbf{ELSE} You will have to resort to the quadratic formula (Which happens to
  work either way, even if you were just lazy and wanted to use it).\\ 

  Proceed by solving,
  \[
        x = \frac{-b \pm \sqrt{b^2 - 4ac}}{2a} 
  .\] 
  From there you will obtain two solutions as your x-intercepts, and your done!

\end{step}

\section{Notion of Solving Quadratic Equations}
Given a quadratic,
\[
    f(x) = ax^2 + bx + c
.\] When we ask you to \textbf{solve the quadratic}, we mean find the solutions to,
\[
    0 = ax^2 + bx + c
.\] The solutions will be the x-intercepts to the original quadratic  $f(x)$.

\newpage

\textbf{\underline{\Large{Practice Problems:}}}\\
All of these problems are from the textbook so you can check your answers from there.
\begin{qstn}
  \textbf{Textbook, pg 49} Question 1.
\end{qstn}

\begin{qstn}
  \textbf{Textbook, pg 49} Question 3.
\end{qstn}

\begin{qstn}
  \textbf{Textbook, pg 49} Question 5.
\end{qstn}



























\end{document}
















































