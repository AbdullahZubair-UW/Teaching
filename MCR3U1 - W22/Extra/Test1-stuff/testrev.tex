
% Prepared by Calvin Kent
%
% Assignment Template v19.02
%
%%% 20xx0x/MATHxxx/Crowdmark/Ax
%
\documentclass[12pt]{article} %
\usepackage{amsthm}
\usepackage{CKpreamble}
\usepackage{CKassignment}
\usepackage{mdframed}
\usepackage{import}
\usepackage{pdfpages}
\usepackage{transparent}
\usepackage{xcolor}

\newcommand{\incfig}[2][1]{%
    \def\svgwidth{#1\columnwidth}
    \import{./figures/}{#2.pdf_tex}
}

\pdfsuppresswarningpagegroup=1


%
\begin{document}
	\pagenumbering{arabic}
	% Start of class settings ...
	\renewcommand*{\coursecode}{MATH 235} % renew course code
	\renewcommand*{\assgnnumber}{Assignment 1} % renew assignment number
	\renewcommand*{\submdate}{September 14, 2021} % renew the date
	\renewcommand*{\studentfname}{Abdullah} % Student first name
	\renewcommand*{\studentlname}{Zubair} % Student last name
    \renewcommand*{\proofname}{Proof:}
	% \renewcommand*{\studentnum}{20836288} % Student number

	\renewcommand\qedsymbol{$\blacksquare$}
	\setfigpath
	% End of class settings	
	% \pagestyle{crowdmark}
	\newgeometry{left=18mm, right=18mm, top=22mm, bottom=22mm} % page is set to default values
	\fancyhfoffset[L,O]{0pt} % header orientation fixed
	% End of class settings
	%%% Note to user:
	% CTRL + F <CHANGE ME:> (without the angular brackets) in CKpreamble to specify graphics paths accordingly.
	% The command \circled[]{} accepts one optional and one mandatory argument.
	% Optional argument is for the size of the circle and mandatory argument is for its contents.
	% \circled{A} produces circled A, with size drawn for letter A. \circled[TT]{A} produces circled A with size drawn for TT.
	% https://github.com/CalvinKent/My-LaTeX
	%%%

	%%%%%%%%%%%%%%%%%%%%%%%%%%%%%%%%%%%%%%%%%%%%%%%%%%%%%%%%%%%%%%%%%%%%%%%%%%%%%%%
	%%%                        CUSTOM MACRO VIM-TEX                             %%%
	%%       call IMAP('NOM', '\nomenclature{}', 'tex')               

	%%%%%%%%%%%%%%%%%%%%%%%%%%%%%%%%%%%%%%%%%%%%%%%%%%%%%%%%%%%%%%%%%%%%%%%%%%%%%%%

	% Crowdmark assignment start
	% qnumber, qname, qpoints

\begin{center}
	\textbf{\underline{\Huge{Test 1 - Review}}}
\end{center}
\begin{qstn}
  Let the set \textbf{English} be the entire English dictionary, and the set \textbf{Arabic} be the entire Arabic dictionary. Lets
  define the following function,
  \begin{align*}
    f \colon \textbf{English} &\rightarrow \textbf{Arabic}\\
    f(\text{English Word}) &= \text{Arabic Word}
  .\end{align*}

  \begin{enumerate}[label=(\alph*)]
    \item Determine $f(\text{door})$.
    \item Determine $f(\text{tall})$.
    \item Determine $f(\text{books})$.
    \item Determine $f(\text{short})$.
  \end{enumerate}
\end{qstn}

\begin{qstn}
  Determine the elements of the following sets, (Reacll that $\N = \{1,2,3,4,5,\dots\} $ ).
  \begin{enumerate}[label=(\alph*)]
    \item $H = \{n \in \N \mid n \geq 4\} $
    \item $R = \{y \in \Z \mid -2 < y \leq 4\}  $
    \item $A = \{r \in \Z \mid r^2 - 4 = 0\} $
  \end{enumerate}
\end{qstn}


\begin{qstn}
  Let $\mathcal{V} = \{3,4,5,6,8,10\}$, and $\mathcal{W} = \{0,1,2,3,4,5\} $. Define the following function,
  \begin{align*}
    R &\colon \mathcal{V} \rightarrow \mathcal{W}\\
    R(v) &= \gcd \left(2v, \operatorname{rem}(v,3)\right)
  .\end{align*}
  Draw a mapping diagram of the function. (\textbf{Note:} $\gcd(x,0) = x$).
\end{qstn}

\begin{qstn}
  Let $T(x) = 3x^2 + 4x$, and $H(x) = x - 1$.
  \begin{enumerate}[label=(\alph*)]
    \item Determine $T(T(1))$.
    \item Determine $H(H(-2))$.
    \item Determine $T(H(0))$.
    \item Factor $T(x)$. \textbf{(This should take one step)}
    \item Factor $T(H(x))$. ***** \textbf{Test Question}
  \end{enumerate}

\end{qstn}

\begin{qstn}
  Determine the Domain and Range of the following functions,
  \begin{enumerate}[label=(\alph*)]
    \item $\mathcal{T}(x) = -\sqrt{-4x + 8} - 7 $.
    \item $F(x) = -x^2+2x+5$.
    \item $L(x) = -2x + 1$.
    \item $\mathcal{P}(x) = 2|-x + 1| - 5$.
    \item $\mathcal{P}(x) = -\frac{3}{5x - 2} + 4$.
    \item $(x + 1)^2 + y^2 = 4$.
  \end{enumerate}
\end{qstn}

\begin{qstn}
  Let $f(x) = 2x^2 + 5x - 3$.
  \begin{enumerate}[label=(\alph*)]
    \item How many solutions will $f(x)$ have?
    \item Factor $f(x)$.
    \item State the x-intercepts of $f(x)$.
    \item Convert $f(x)$ from factored form to vertex form.
    \item Using your function in vertex form, sketch it \textbf{(Label the y-intercept, x-intercepts and the vertex}).
  \end{enumerate}
\end{qstn}






































\end{document}




















