% Prepared by Calvin Kent
%
% Assignment Template v19.02
%
%%% 20xx0x/MATHxxx/Crowdmark/Ax
%
\documentclass[12pt]{article} %
\usepackage{amsthm}
\usepackage{CKpreamble}
\usepackage{CKassignment}
\usepackage{mdframed}
\usepackage{tikz}
\usepackage{pgfplots}

%
\begin{document}
	\pagenumbering{arabic}
	% Start of class settings ...
	\renewcommand*{\coursecode}{MATH 235} % renew course code
	\renewcommand*{\assgnnumber}{Assignment 1} % renew assignment number
	\renewcommand*{\submdate}{September 14, 2021} % renew the date
	\renewcommand*{\studentfname}{Abdullah} % Student first name
	\renewcommand*{\studentlname}{Zubair} % Student last name
    \renewcommand*{\proofname}{Proof:}
	% \renewcommand*{\studentnum}{20836288} % Student number

	\renewcommand\qedsymbol{$\blacksquare$}
	\setfigpath
	% End of class settings	
	% \pagestyle{crowdmark}
	\newgeometry{left=18mm, right=18mm, top=22mm, bottom=22mm} % page is set to default values
	\fancyhfoffset[L,O]{0pt} % header orientation fixed
	% End of class settings
	%%% Note to user:
	% CTRL + F <CHANGE ME:> (without the angular brackets) in CKpreamble to specify graphics paths accordingly.
	% The command \circled[]{} accepts one optional and one mandatory argument.
	% Optional argument is for the size of the circle and mandatory argument is for its contents.
	% \circled{A} produces circled A, with size drawn for letter A. \circled[TT]{A} produces circled A with size drawn for TT.
	% https://github.com/CalvinKent/My-LaTeX
	%%%

	%%%%%%%%%%%%%%%%%%%%%%%%%%%%%%%%%%%%%%%%%%%%%%%%%%%%%%%%%%%%%%%%%%%%%%%%%%%%%%%
	%%%                        CUSTOM MACRO VIM-TEX                             %%%
	%%       call IMAP('NOM', '\nomenclature{}', 'tex')               

	%%%%%%%%%%%%%%%%%%%%%%%%%%%%%%%%%%%%%%%%%%%%%%%%%%%%%%%%%%%%%%%%%%%%%%%%%%%%%%%

	% Crowdmark assignment start
	% qnumber, qname, qpoints

\begin{center}
	\textbf{\underline{\Huge{Interactive Problem Solving}}}
\end{center}

In this problem set you will solve problems interactively, meaning that you will confirm your solutions to the questions
\textbf{Desmos} and \textbf{Python}. Explanation of \textbf{Python} will be done in class,

\begin{qstn}
  Let $f(x) = -2x^2 + 5x + 3$, and $g(x) = -2\sqrt{8x+1} + 4$.
  \begin{enumerate}[label=(\alph*)]
    \item Compute $f(-1)$. (Confirm with \textbf{Python})
    \item Compute $g(3)$. (Confirm with \textbf{Python})
    \item Compute $f(f(3))$. (Confirm with \textbf{Python})
    \item Compute $f(g(f(0)))$. (Confirm with \textbf{Python})
    \item State the Domain and Range of $f$. (Confirm with \textbf{Desmos})
    \item State the Domain and Range of $g$. (Confirm with \textbf{Desmos})
    \item Factor $f$.
      \begin{itemize}
        \item To confirm this, use \textbf{desmos} and plot both the original $f$ and $f$ in factored form. 
        \item If the graphs \textbf{overlap} then your answer is correct.
      \end{itemize}
    \item State the x-intercepts of $f$. (Confirm with  \textbf{Desmos}).
    \item Convert $f$ to vertex form. (Confirm your answer in \textbf{Desmos} using the overlapping strategy).
  \end{enumerate}
\end{qstn}


\begin{qstn} Let $f(x) = 3x^2 -5x -12$ and $g(x) = x - 3$. In this question we will try to find the intersection point of the two
  graphs.
  \begin{enumerate}[label=(\alph*)]
    \item Factor $f$ (Just for practice). 
    \item Let $h(x) = f(x) - g(x)$, state the Domain and Range of $h(x)$.
    \item Factor $h(x)$.
    \item Determine the x-intercepts of $h(x)$, label them $x_1$ and $x_2$.
    \item Calculate $g(x_1)$ and  $g(x_2)$, afterwards write down the points $(x_1,g(x_1))$ and  $(x_2,g(x_2))$.
    \item Graph both  $f(x)$ and $g(x)$ in Desmos, hover over the intersection points and confirm that they \textbf{both}
      match what you got.
  \end{enumerate}
  
\end{qstn}




\end{document}




























