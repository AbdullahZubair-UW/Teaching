
% Prepared by Calvin Kent
%
% Assignment Template v19.02
%
%%% 20xx0x/MATHxxx/Crowdmark/Ax
%
\documentclass[12pt]{article} %
\usepackage{amsthm}
\usepackage{CKpreamble}
\usepackage{CKassignment}
\usepackage{mdframed}
\usepackage{import}
\usepackage{pdfpages}
\usepackage{transparent}
\usepackage{xcolor}

\newcommand{\incfig}[2][1]{%
    \def\svgwidth{#1\columnwidth}
    \import{./figures/}{#2.pdf_tex}
}

\pdfsuppresswarningpagegroup=1


%
\begin{document}
	\pagenumbering{arabic}
	% Start of class settings ...
	\renewcommand*{\coursecode}{MATH 235} % renew course code
	\renewcommand*{\assgnnumber}{Assignment 1} % renew assignment number
	\renewcommand*{\submdate}{September 14, 2021} % renew the date
	\renewcommand*{\studentfname}{Abdullah} % Student first name
	\renewcommand*{\studentlname}{Zubair} % Student last name
    \renewcommand*{\proofname}{Proof:}
	% \renewcommand*{\studentnum}{20836288} % Student number

	\renewcommand\qedsymbol{$\blacksquare$}
	\setfigpath
	% End of class settings	
	% \pagestyle{crowdmark}
	\newgeometry{left=18mm, right=18mm, top=22mm, bottom=22mm} % page is set to default values
	\fancyhfoffset[L,O]{0pt} % header orientation fixed
	% End of class settings
	%%% Note to user:
	% CTRL + F <CHANGE ME:> (without the angular brackets) in CKpreamble to specify graphics paths accordingly.
	% The command \circled[]{} accepts one optional and one mandatory argument.
	% Optional argument is for the size of the circle and mandatory argument is for its contents.
	% \circled{A} produces circled A, with size drawn for letter A. \circled[TT]{A} produces circled A with size drawn for TT.
	% https://github.com/CalvinKent/My-LaTeX
	%%%

	%%%%%%%%%%%%%%%%%%%%%%%%%%%%%%%%%%%%%%%%%%%%%%%%%%%%%%%%%%%%%%%%%%%%%%%%%%%%%%%
	%%%                        CUSTOM MACRO VIM-TEX                             %%%
	%%       call IMAP('NOM', '\nomenclature{}', 'tex')               

	%%%%%%%%%%%%%%%%%%%%%%%%%%%%%%%%%%%%%%%%%%%%%%%%%%%%%%%%%%%%%%%%%%%%%%%%%%%%%%%

	% Crowdmark assignment start
	% qnumber, qname, qpoints

\begin{center}
		\Huge{\underline{\textbf{How to Complete the Square}}}
\end{center}
	Lets say we are given some quadratic polynomial $y = ax^2 + bx + c$ and we would like to convert this into vertex form, $y =
	a(x - h)^2 + k$, the technique to do so is called completing the square. Remember what we dicussed yesterday, if a polynomial is
	a perfect square, then its discrimnant ($d = b^2 - 4ac$) will be zero. (This idea will be used in the technique)
	\begin{enumerate}
		\item Find the $t$ value such that \[
								0 = b^2 - 4at
		.\] 
	\item Insert $t$ into the equation as follows,
						\[
						y = ax^2 + bx + t - t + c
						.\] 
	\item Seperate the equation into two parts,
		\[
		y = (ax^2 + bx + t) + [c - t]
		.\] 
	\item Factor out $a$ from the round bracktes. \textbf{(IF $a = 1$ then nothing should change)}.
		     \[
		     y = a\left(x^2 + \frac{b}{a}x + \frac{t}{a}\right) + [c - t]
		     .\] 
	\item \textbf{IF $a\cdot b > 0$, then replace the equation with the following},
		 \[
					y = a\left( x + \sqrt{\frac{t}{a}} \right)^2 + [c - t] 
		.\] 
	\item \textbf{ELSE IF $a\cdot b < 0$, then replace the equation with the following},
		 \[
					y = a\left( x - \sqrt{\frac{t}{a}} \right)^2 + [c - t] 
		.\]  
	\item Were done! (The new polynomial is now in vertex form if you look carefully)
	\end{enumerate}


\textbf{\underline{\Large{Practice Problems:}}}

Some of these problems are from the textbook, you can check your answers from there.
\begin{qstn}
	\textbf{Textbook:} Question 1 a),c),e),f)
\end{qstn}

\begin{qstn}
	\textbf{Textbook:} Question 2 b),d),f)
\end{qstn}




\end{document}




