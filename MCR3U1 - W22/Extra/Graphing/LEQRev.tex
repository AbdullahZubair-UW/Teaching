

% Prepared by Calvin Kent
%
% Assignment Template v19.02
%
%%% 20xx0x/MATHxxx/Crowdmark/Ax
%
\documentclass[12pt]{article} %
\usepackage{amsthm}
\usepackage{CKpreamble}
\usepackage{CKassignment}
\usepackage{mdframed}
\usepackage{import}
\usepackage{pdfpages}
\usepackage{transparent}
\usepackage{xcolor}

\newcommand{\incfig}[2][1]{%
    \def\svgwidth{#1\columnwidth}
    \import{./figures/}{#2.pdf_tex}
}

\pdfsuppresswarningpagegroup=1


%
\begin{document}
	\pagenumbering{arabic}
	% Start of class settings ...
	\renewcommand*{\coursecode}{MATH 235} % renew course code
	\renewcommand*{\assgnnumber}{Assignment 1} % renew assignment number
	\renewcommand*{\submdate}{September 14, 2021} % renew the date
	\renewcommand*{\studentfname}{Abdullah} % Student first name
	\renewcommand*{\studentlname}{Zubair} % Student last name
    \renewcommand*{\proofname}{Proof:}
	% \renewcommand*{\studentnum}{20836288} % Student number

	\renewcommand\qedsymbol{$\blacksquare$}
	\setfigpath
	% End of class settings	
	% \pagestyle{crowdmark}
	\newgeometry{left=18mm, right=18mm, top=22mm, bottom=22mm} % page is set to default values
	\fancyhfoffset[L,O]{0pt} % header orientation fixed
	% End of class settings
	%%% Note to user:
	% CTRL + F <CHANGE ME:> (without the angular brackets) in CKpreamble to specify graphics paths accordingly.
	% The command \circled[]{} accepts one optional and one mandatory argument.
	% Optional argument is for the size of the circle and mandatory argument is for its contents.
	% \circled{A} produces circled A, with size drawn for letter A. \circled[TT]{A} produces circled A with size drawn for TT.
	% https://github.com/CalvinKent/My-LaTeX
	%%%

	%%%%%%%%%%%%%%%%%%%%%%%%%%%%%%%%%%%%%%%%%%%%%%%%%%%%%%%%%%%%%%%%%%%%%%%%%%%%%%%
	%%%                        CUSTOM MACRO VIM-TEX                             %%%
	%%       call IMAP('NOM', '\nomenclature{}', 'tex')               

	%%%%%%%%%%%%%%%%%%%%%%%%%%%%%%%%%%%%%%%%%%%%%%%%%%%%%%%%%%%%%%%%%%%%%%%%%%%%%%%

	% Crowdmark assignment start
	% qnumber, qname, qpoints

\begin{center}
		\underline{\textbf{\Huge{How to Graph Linear Equations}}}
\end{center}

	\vspace*{0.5cm}
\begin{enumerate}
			\item Identify the slope and the y-intercept.
			\item Label the y-intercept on the y-axis.
			\item Remeber that, \[
								\text{slope} = \frac{\text{rise}}{\text{run}}
			.\] Starting from the y-intercept go up by 'rise' units and go to the right by 'run' units. (Remeber that a negative 'rise' means go down and a
			negative 'run' means go left).
			\item Label the point where you end up at.
			\item Draw a straight line through the two points you have labelled.
			\item Finish by labelling the graph and drawing the tip arrows.
\end{enumerate}

\textbf{\underline{\Large{Practice Problems:}}}

Double check your answers by using the graphing website \textbf{Desmos}.(Google it)
\begin{qstn}
	Graph the following linear equations.
	\begin{enumerate}[label=(\alph*)]
		\item $y = \frac{2}{3}x + 11$
		\item $y = \frac{-2}{3}x$
		\item $3x - y = 4$
		\item $3x + 7y = 1$
		\item $y = -2x - 1$
		\item $y = -\frac{4}{5}x + 1$
	\end{enumerate}
\end{qstn}

\begin{qstn}
	Determine the point of intersection of the following pairs of linear equations.
	\begin{enumerate}[label=(\alph*)]
		\item \begin{align*}
			3x - y &= 4\\
			2x + 2y &= 18
		.\end{align*}
		\item \begin{align*}
		  x + y &= 4\\
			2x + 7y &= 0
		.\end{align*}
		\item \begin{align*}
			9x + 3y &= 21\\
			y - 2x &= 1
		.\end{align*}
	\end{enumerate}



\end{qstn}



\end{document}





