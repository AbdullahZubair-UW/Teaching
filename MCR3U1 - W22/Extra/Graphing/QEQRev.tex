

% Prepared by Calvin Kent
%
% Assignment Template v19.02
%
%%% 20xx0x/MATHxxx/Crowdmark/Ax
%
\documentclass[12pt]{article} %
\usepackage{amsthm}
\usepackage{CKpreamble}
\usepackage{CKassignment}
\usepackage{mdframed}
\usepackage{import}
\usepackage{pdfpages}
\usepackage{transparent}
\usepackage{xcolor}

\newcommand{\incfig}[2][1]{%
    \def\svgwidth{#1\columnwidth}
    \import{./figures/}{#2.pdf_tex}
}

\pdfsuppresswarningpagegroup=1


%
\begin{document}
	\pagenumbering{arabic}
	% Start of class settings ...
	\renewcommand*{\coursecode}{MATH 235} % renew course code
	\renewcommand*{\assgnnumber}{Assignment 1} % renew assignment number
	\renewcommand*{\submdate}{September 14, 2021} % renew the date
	\renewcommand*{\studentfname}{Abdullah} % Student first name
	\renewcommand*{\studentlname}{Zubair} % Student last name
    \renewcommand*{\proofname}{Proof:}
	% \renewcommand*{\studentnum}{20836288} % Student number

	\renewcommand\qedsymbol{$\blacksquare$}
	\setfigpath
	% End of class settings	
	% \pagestyle{crowdmark}
	\newgeometry{left=18mm, right=18mm, top=22mm, bottom=22mm} % page is set to default values
	\fancyhfoffset[L,O]{0pt} % header orientation fixed
	% End of class settings
	%%% Note to user:
	% CTRL + F <CHANGE ME:> (without the angular brackets) in CKpreamble to specify graphics paths accordingly.
	% The command \circled[]{} accepts one optional and one mandatory argument.
	% Optional argument is for the size of the circle and mandatory argument is for its contents.
	% \circled{A} produces circled A, with size drawn for letter A. \circled[TT]{A} produces circled A with size drawn for TT.
	% https://github.com/CalvinKent/My-LaTeX
	%%%

	%%%%%%%%%%%%%%%%%%%%%%%%%%%%%%%%%%%%%%%%%%%%%%%%%%%%%%%%%%%%%%%%%%%%%%%%%%%%%%%
	%%%                        CUSTOM MACRO VIM-TEX                             %%%
	%%       call IMAP('NOM', '\nomenclature{}', 'tex')               

	%%%%%%%%%%%%%%%%%%%%%%%%%%%%%%%%%%%%%%%%%%%%%%%%%%%%%%%%%%%%%%%%%%%%%%%%%%%%%%%

	% Crowdmark assignment start
	% qnumber, qname, qpoints

\begin{center}
	\textbf{\underline{\Huge{How to Sketch Quadratic Equations}}}
\end{center}

	\vspace*{0.5cm}
	Lets say we have a quadratic equation in standard form $y = ax^2 + bx + c$ and we want to sketch it, then preform the following,
	\begin{enumerate}
		\item Convert $y$ into vertex form by completing the square \\
			(Refer to the \emph{How to complete the square sheet} if you are stuck)
		\item Now you should have a new quadratic in vertex form,
						\[
						y = a(x-h)^2 + k
						.\] 
						Where the vertex is located at $(h,k)$.
		\item Label the vertex point on your graph.
		\item \textbf{IF $a < 0$ ($a$ is negative)} then draw the shape of the parabola in the \emph{downwards} direction starting
			from the vertex.
		\item \textbf{ELSE IF $a > 0$ ($a$ is positive)} then draw the shape of the parabola in the \emph{upwards} direction starting
			from the vertex.
	\end{enumerate}

\section{Maximum's and Minimums}
\begin{itemize}
		\item \textbf{IF $a < 0$ ($a$ is negative)} then we say that the vertex represents a \emph{maximum}.
			\begin{itemize}
				\item This makes sense because if $a < 0$, the parabola points down, and the highest point of the parabola must have been
					the vertex.
			\end{itemize}
		\item \textbf{ELSE IF $a > 0$ ($a$ is positive)} then we say that the vertex represents a \emph{minimum}.
			\begin{itemize}
				\item This makes sense because if $a > 0$, the parabola points up, and the lowest point of the parabola must have been
					the vertex.
			\end{itemize}
\end{itemize}

\textbf{\underline{\Large{Practice Problems:}}}

Double check your answers by using the graphing website \textbf{Desmos}.(Google it)
\begin{qstn}
	Sketch the following Quadratic equations. Label the $y-$intercept of each graph.
	\begin{enumerate}[label=(\alph*)]
		\item $y = x^2 - x - 6$
		\item $y = -4x^2 + 7x - 12$
		\item $y = 2x^2 + 5x - 3$
		\item $y = -x^2 + 4x$
		\item $y = x^2 - 9$ (Isnt this already in vertex form?)
		\item $y = \frac{2}{3}x - \frac{5}{3}x + \frac{1}{3}$
	\end{enumerate}
\end{qstn}
\begin{qstn}
	\textbf{(Challenge)} Determine the intersection point of the following two quadratic equations,
	\begin{align*}
		y &= x^2 - 5x + 6\\
		y &= x^2 + x - 6	
	.\end{align*}
	(The technique is the same as in the linear case).
	
\end{qstn}






\end{document}














