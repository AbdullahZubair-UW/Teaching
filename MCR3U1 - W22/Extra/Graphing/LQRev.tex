% Prepared by Calvin Kent
% Prepared by Calvin Kent
%
% Lecture Template v19.02
%
%%% 201901/MATHxxx/Notes
%
\documentclass[12pt,oneside]{book} %
\usepackage{CKpreamble}
\usepackage{CKlecture}
\usepackage{mdframed}
\usepackage{import}
\usepackage{pdfpages}
\usepackage{transparent}
\usepackage{xcolor}

\newcommand{\incfig}[2][1]{%
    \def\svgwidth{#1\columnwidth}
    \import{./figures/}{#2.pdf_tex}
}

\pdfsuppresswarningpagegroup=1

%
\renewcommand*{\doctitle}{Class Based Lecture Notes}
\makeatletter\patchcmd{\chapter}{\if@openright\cleardoublepage\else\clearpage\fi}{}{}{}\makeatother % only used in class based
\begin{document}
	% Start of Class settings
	\renewcommand*{\term}{Term 2} % Term
	\renewcommand*{\coursecode}{MCR3U} % Course code
	\renewcommand*{\thelecnum}{2} % Course code
	\renewcommand*{\coursename}{Course Name} % Full course name
	\renewcommand*{\profname}{Prof Name} % Prof Name
	\renewcommand*{\colink}{http://www.student.math.uwaterloo.ca/~c2kent} % Course outline link
	% End of Class settings
	\clearpage
	\pagenumbering{arabic}
	%%% Note to user: CTRL + F <CHANGE ME:> (without the angular brackets) in CKpreamble to specify graphics paths accordingly.
	%%% If a new chapter was started in the middle of a lecture, \fixchap{Second Chapter} must be used immediately above the next lecture.
	% Course notes start
\setchap{2}{Graphing Linear equations}
\begin{lec}{November 2021}
	\begin{center}
		\Huge{\underline{\textbf{How to Graph Linear Equations}}}
	\end{center}

	\vspace*{0.5cm}

	\begin{mdframed}
		\begin{defn}
			An equation is \textbf{linear} if it looks like
			\[
						y = mx+ b
			.\] 
			Where $m,b$ are real numbers and $x,y$ are variables.
		\end{defn}
	\end{mdframed}
	\begin{ex}
		Here are some examples of linear equations,
		\begin{enumerate}
			\item $y = 2x + 4$
			\item $y = 3$
			\item $y = \frac{-2}{3}x + 7$
		\end{enumerate}
	\end{ex}
	\begin{ex}
		The following are examples of non-linear graphs.
		\begin{enumerate}
			\item $y = x^2 + 2x + 4$
			\item $y^2 + x^2 = 4$ (What is this?)
		\end{enumerate}
	\end{ex}
	We are often interested about how linear equations look like geometrically on a \emph{Cartesian Plot}, we call these plots the
	\emph{graph} of the linear equation.
	\begin{ex}
		Here are what the graphs of the linear equations from Example 2.1 look like. \textbf{(In class)}
	\end{ex}


	\begin{mdframed}
		\begin{defn}
			The \textbf{slope} of a linear equation is a measure of how 'steep' its graph is. The slope is given by the number $m$ in $y = mx + b$.
		\end{defn}
	\end{mdframed}
	We say that larger values of $m$ correspond to graphs that have very steep slopes.\\
	\textbf{(Class explanation)}
	\begin{ex}
		What were the slopes of the linear equations from Example 2.1? \textbf{(In class)}
	\end{ex}

	\vspace*{0.2cm}


	\begin{mdframed}
		\begin{defn}
			The \textbf{y-intercept} of a liner equation is the point where the graph intersects the y-axis. This point is given by the
			number $b$ in $y = mx + b$.
		\end{defn}
	\end{mdframed}

	\begin{ex}
		What were the y-intercept's of the linear equations from Example 2.1?\\ \textbf{(In class)}
	\end{ex}

	At this point we would like to answer the following question, if I give you a liner equation $y = mx + b$, can you plot its
	graph? We can! by using the following procedure,
	\newpage

	\begin{proc}
		Graphing Linear equations.
		\begin{enumerate}
			\item Identify the slope and the y-intercept.
			\item Label the y-intercept on the y-axis.
			\item Remeber that, \[
								\text{slope} = \frac{\text{rise}}{\text{run}}
			.\] Starting from the y-intercept go up by 'rise' units and go to the right by 'run' units. (Remeber that a negative 'rise' means go down and a
			negative 'run' means go left).
			\item Label the point where you end up at.
			\item Draw a straight line through the two points you have labelled.
			\item Finish by labelling the graph and drawing the tip arrows.
		\end{enumerate}
	\end{proc}

	\begin{ex}
		Graph the following linear equations. \textbf{(In class)}
		\begin{enumerate}
			\item $y = -2x + 6$
			\item $y = \frac{-2}{3}x - 1$
			\item $y = \frac{1}{-2}x$
		\end{enumerate}
	\end{ex}

	\begin{mdframed}
		\begin{defn}
			The \textbf{x-intercept} of a liner equation is the point where the graph intersects the x-axis. This point is given by the
			 solution to $0 = mx + b$. (Solving for $x $)
		\end{defn}
	\end{mdframed}

	\begin{ex}
		Determine the x-intercepts of the following linear equations.\textbf{(In class)}
		\begin{enumerate}
			\item $y = 2x - 3$
			\item $y = \frac{-2}{5}x + 2$
			\item $y = 5x$
			\item $y = 3$
		\end{enumerate}
	\end{ex}

	\end{lec}

\end{document}
%\fixchap{Second Chapter}
%	\begin{figure}[H]
%	\centering
%	\includegraphics[width=0.75\linewidth]{p}
%	\caption{caption.\label{fig:}}
%	\end{figure}
































