


% Prepared by Calvin Kent
%
% Assignment Template v19.02
%
%%% 20xx0x/MATHxxx/Crowdmark/Ax
%
\documentclass[12pt]{article} %
\usepackage{amsthm}
\usepackage{CKpreamble}
\usepackage{CKassignment}
\usepackage{mdframed}
\usepackage{import}
\usepackage{pdfpages}
\usepackage{transparent}
\usepackage{xcolor}

\newcommand{\incfig}[2][1]{%
    \def\svgwidth{#1\columnwidth}
    \import{./figures/}{#2.pdf_tex}
}

\pdfsuppresswarningpagegroup=1


%
\begin{document}
	\pagenumbering{arabic}
	% Start of class settings ...
	\renewcommand*{\coursecode}{MATH 235} % renew course code
	\renewcommand*{\assgnnumber}{Assignment 1} % renew assignment number
	\renewcommand*{\submdate}{September 14, 2021} % renew the date
	\renewcommand*{\studentfname}{Abdullah} % Student first name
	\renewcommand*{\studentlname}{Zubair} % Student last name
    \renewcommand*{\proofname}{Proof:}
	% \renewcommand*{\studentnum}{20836288} % Student number

	\renewcommand\qedsymbol{$\blacksquare$}
	\setfigpath
	% End of class settings	
	% \pagestyle{crowdmark}
	\newgeometry{left=18mm, right=18mm, top=22mm, bottom=22mm} % page is set to default values
	\fancyhfoffset[L,O]{0pt} % header orientation fixed
	% End of class settings
	%%% Note to user:
	% CTRL + F <CHANGE ME:> (without the angular brackets) in CKpreamble to specify graphics paths accordingly.
	% The command \circled[]{} accepts one optional and one mandatory argument.
	% Optional argument is for the size of the circle and mandatory argument is for its contents.
	% \circled{A} produces circled A, with size drawn for letter A. \circled[TT]{A} produces circled A with size drawn for TT.
	% https://github.com/CalvinKent/My-LaTeX
	%%%

	%%%%%%%%%%%%%%%%%%%%%%%%%%%%%%%%%%%%%%%%%%%%%%%%%%%%%%%%%%%%%%%%%%%%%%%%%%%%%%%
	%%%                        CUSTOM MACRO VIM-TEX                             %%%
	%%       call IMAP('NOM', '\nomenclature{}', 'tex')               

	%%%%%%%%%%%%%%%%%%%%%%%%%%%%%%%%%%%%%%%%%%%%%%%%%%%%%%%%%%%%%%%%%%%%%%%%%%%%%%%

	% Crowdmark assignment start
	% qnumber, qname, qpoints

\begin{center}
	\textbf{\underline{\Huge{How to Graph Circles}}}
\end{center}

	\vspace*{0.5cm}
	Lets say we have the equation of a circle $(x-a)^2 + (y-b)^2 = r$ and we want to graph it, then preform the follwing,
	\begin{enumerate}
		\item Label the point $(a,b)$ on your graph. (FWI, this is the centre of the circle)
		\item Starting from $(a,b)$, go to the \textbf{right} by $\sqrt{r} $ units, label the point.
		\item Starting from $(a,b)$, go to the \textbf{left} by $\sqrt{r} $ units, label the point.
		\item Starting from $(a,b)$, go to the \textbf{up} by $\sqrt{r} $ units, label the point.
		\item Starting from $(a,b)$, go to the \textbf{up} by $\sqrt{r} $ units, label the point.
		\item Conect the \textbf{four} points you have labled to form a circle.
\end{enumerate}

A couple of things to note,
\begin{itemize}
	\item $(a,b)$ represents the centre of the circle.
	\item $\sqrt{r} $ represents the radius of the circle (not $r$).
\end{itemize}

\textbf{\underline{\Large{Practice Problems:}}}

Double check your answers by using the graphing website \textbf{Desmos}.(Google it)

\begin{qstn}
	Graph the following cirlces.
	\begin{enumerate}[label=(\alph*)]
		\item $(x - 1)^2 + (y-2)^2 = 4$
		\item $x^2 + (y-1)^2 = 9$
		\item $(x+4)^2 + (y - 4)^2 = 1$
		\item $x^2 + y^2 = 16$
		\item $(x+2)^2 + (x+3)^2 = 25$
	\end{enumerate}
\end{qstn}

\begin{qstn}
	Write the equation of your favourite circle, then graph it.
\end{qstn}

\begin{qstn}
	I am a circle with a radius of $2$. My centre is unknown (yikes\ldots). Here are some points that I lie upon,
	\begin{itemize}
		\item $(4,-4)$
		\item $(2,-2)$
	\end{itemize}
	Help me find my \textbf{centre and my equation}.
\end{qstn}



\begin{qstn}
	\textbf{(SUPER CHALLENGE)} Prove that 
				\[
					2y^2 + x^2 = 1
				\] is not a circle.
\end{qstn}




\end{document}














