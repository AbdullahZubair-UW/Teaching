% Prepared by Calvin Kent
%
% Assignment Template v19.02
%
%%% 20xx0x/MATHxxx/Crowdmark/Ax
%
\documentclass[12pt]{article} %
\usepackage{amsthm}
\usepackage{CKpreamble}
\usepackage{CKassignment}
\usepackage{mdframed}
\usepackage{import}
\usepackage{pdfpages}
\usepackage{transparent}
\usepackage{xcolor}

\newcommand{\incfig}[2][1]{%
    \def\svgwidth{#1\columnwidth}
    \import{./figures/}{#2.pdf_tex}
}

\pdfsuppresswarningpagegroup=1


%
\begin{document}
	\pagenumbering{arabic}
	% Start of class settings ...
	\renewcommand*{\coursecode}{MATH 235} % renew course code
	\renewcommand*{\assgnnumber}{Assignment 1} % renew assignment number
	\renewcommand*{\submdate}{September 14, 2021} % renew the date
	\renewcommand*{\studentfname}{Abdullah} % Student first name
	\renewcommand*{\studentlname}{Zubair} % Student last name
    \renewcommand*{\proofname}{Proof:}
	% \renewcommand*{\studentnum}{20836288} % Student number

	\renewcommand\qedsymbol{$\blacksquare$}
	\setfigpath
	% End of class settings	
	% \pagestyle{crowdmark}
	\newgeometry{left=18mm, right=18mm, top=22mm, bottom=22mm} % page is set to default values
	\fancyhfoffset[L,O]{0pt} % header orientation fixed
	% End of class settings
	%%% Note to user:
	% CTRL + F <CHANGE ME:> (without the angular brackets) in CKpreamble to specify graphics paths accordingly.
	% The command \circled[]{} accepts one optional and one mandatory argument.
	% Optional argument is for the size of the circle and mandatory argument is for its contents.
	% \circled{A} produces circled A, with size drawn for letter A. \circled[TT]{A} produces circled A with size drawn for TT.
	% https://github.com/CalvinKent/My-LaTeX
	%%%

	%%%%%%%%%%%%%%%%%%%%%%%%%%%%%%%%%%%%%%%%%%%%%%%%%%%%%%%%%%%%%%%%%%%%%%%%%%%%%%%
	%%%                        CUSTOM MACRO VIM-TEX                             %%%
	%%       call IMAP('NOM', '\nomenclature{}', 'tex')               

	%%%%%%%%%%%%%%%%%%%%%%%%%%%%%%%%%%%%%%%%%%%%%%%%%%%%%%%%%%%%%%%%%%%%%%%%%%%%%%%

	% Crowdmark assignment start
	% qnumber, qname, qpoints

\begin{center}
  \underline{\textbf{\Huge{How to determine Domain \& Range}}}
\end{center}

	\vspace*{0.5cm}
  We first must identify which type of function we are dealing with before we proceed.
  \section{Linear Functions $(f(x) = mx + b)$}
  \begin{enumerate}
    \item The Domain will always be,
      \[
            \mathcal{D} = \R
      .\] 
    \item The Range will always be,
      \[
            \mathcal{R} = \R
      .\] 
  \end{enumerate}

  \section{Quadratic Functions $(f(x) = ax^2 + bx + c)$}
  \begin{enumerate}
    \item First convert the quadratic into vertex form by completing the square. After your done, you should have something that
      looks like,
      \[
            f(x) = a\left( x - h  \right)^2 +  k
      .\] 
      Where $(h,k)$ is the vertex.
    \item The Domain will always be,
      \[
        \mathcal{D} = \R
      .\] 
    \item \textbf{IF $a > 0$ ($a$ is positive)} then the Range will be,
      \[
            \mathcal{R} = \{y\in \R \mid y \geq k\} 
      .\] 
    \item \textbf{ELSE IF $a < 0$ ($a$ is negative)} then the Range will be,
      \[
            \mathcal{R} = \{y\in \R \mid y \leq k\} 
      .\] 
  \end{enumerate}

  \section{Absolute Value functions $(f(x) = a|mx + b| + k)$}
  \begin{enumerate}
  \item The Domain will always be,
      \[
        \mathcal{D} = \R
      .\] 
    \item \textbf{IF $a > 0$ ($a$ is positive)} then the Range will be,
      \[
            \mathcal{R} = \{y\in \R \mid y \geq k\} 
      .\] 
    \item \textbf{ELSE IF $a < 0$ ($a$ is negative)} then the Range will be,
      \[
            \mathcal{R} = \{y\in \R \mid y \leq k\} 
      .\] 
  \end{enumerate}

  \section{Rational Functions $(f(x) = \frac{a}{mx + b} + k)$}
  \begin{enumerate}
    \item Then the Domain will be,
      \[
        \mathcal{D} = \{x\in \R \mid x\neq -\frac{b}{m}\} 
      .\]
    \item The Range will always be,
      \[
            \mathcal{R} = \{y \in \R \mid y \neq k\} 
      .\]   \end{enumerate}


  \section{Radical Functions $(f(x) = a\sqrt{mx + b} + k)$ }
  \begin{enumerate}
    \item \textbf{IF $m < 0$ ($m$ is negative)} then the Domain will be,
      \[ \mathcal{D} = \{x\in \R \mid x \leq -\frac{b}{m}\} 
      .\] 
    \item \textbf{ELSE IF $m > 0$ ($m$ is positive)} then the Domain will be,
      \[
            \mathcal{D} = \{x \in \R \mid x \geq -\frac{b}{m}\} 
      .\] 
    \item \textbf{IF $a > 0$ ($a$ is positive)} then the Range will be,
      \[
            \mathcal{R} = \{y\in \R \mid y \geq k\} 
      .\] 
    \item \textbf{ELSE IF $a < 0$ ($a$ is negative)} then the Range will be,
      \[
            \mathcal{R} = \{y\in \R \mid y \leq k\} 
      .\] 
  \end{enumerate}
  
  \section{Cirlces $\left((x - a)^2 + (y - b)^2 = r\right)$}
  \textbf{NOTE!: }Circles are \emph{not} functions! \textbf{(Explanation in Class)}. We would still like to know the Domain and
  Range of circles (Just ignore how they are not functions).
  \begin{enumerate}
    \item First write down the centre of the circle $(a,b)$.
    \item The Domain will be,
      \[
        \mathcal{D} = \{x\in \R \mid a - \sqrt{r} \leq x \leq a + \sqrt{r} \} 
      .\] 
    \item The Range will be,
      \[
      \mathcal{R} = \{y \in \R \mid b - \sqrt{r} \leq y \leq b + \sqrt{r} \} 
      .\] 
  \end{enumerate}

  \newpage

\textbf{\underline{\Large{Practice Problems:}}}
\begin{qstn}
  Determine the Domain and Range of the following functions,
  \begin{enumerate}[label=(\alph*)]
    \item $L(x) = x$
    \item $f(x) = 2x - 1$
    \item $T(x) = 2$
    \item $f(x) = 2(x-1)$
    \item $2y - 5x = 11$
    \item $y - 5x = 1$
  \end{enumerate}
\end{qstn}

\begin{qstn}
  Determine the Domain and Range of the following functions,
  \begin{enumerate}[label=(\alph*)]
    \item $L(x) = x^2$
    \item $f(x) = x^2 - x - 6$.
    \item $q(x) = -3x^2 - 2x + 1$.
    \item $p(x) = 4x^2 - 9$. (Isnt this already in vertex form?)
    \item $r(x) = -2x^2 - 8x + 1$.
    \item $V(x) = -x^2 - \frac{9}{5}x + 1$.
  \end{enumerate}
\end{qstn}

\begin{qstn}
  Determine the Domain and Range of the following functions,
  \begin{enumerate}[label=(\alph*)]
    \item $L(x) = \left|x\right|$
    \item $f(x) = -2| 2x + 1| - 1$.
    \item $g(x) = 3|-x - 1| + 2$.
    \item $m(x) = 5|-\frac{x}{2} - \frac{3}{2}| - \frac{5}{2}$.
    \item $\xi(x) = -|-x|$.
  \end{enumerate}
\end{qstn}

\begin{qstn}
  Determine the Domain and Range of the following functions,
  \begin{enumerate}[label=(\alph*)]
    \item $L(x) = \frac{1}{x}$
    \item $f(x) = \frac{-3}{-2x + 1} - 3$ 
    \item $T(x) = \frac{3}{x - 1} - \frac{4}{3}$
    \item $r(x) = \frac{1}{\frac{1}{3}x - \frac{5}{4}} - 16$
    \item $q(x) = \frac{1}{-x}$
  \end{enumerate}
\end{qstn}

\newpage

\begin{qstn}
  Determine the Domain and Range of the following functions,
  \begin{enumerate}[label=(\alph*)]
    \item $L(x) = \sqrt{x} $.
    \item $f(x) = -2\sqrt{-5x + 7} - 1 $
    \item $g(x) = 5\sqrt{-2x-36} + 12$
    \item $H(x) = -\sqrt{-x}$
    \item $V(x) = \sqrt{x + 1} - 1$
  \end{enumerate}
\end{qstn}

\begin{qstn}
  Determine the Domain and Range of the following circles,
  \begin{enumerate}[label=(\alph*)]
		\item $(x + 7)^2 + (y-2)^2 = 4$
		\item $x^2 + (y+1)^2 = 9$
		\item $(x+9)^2 + (y - 4)^2 = 1$
		\item $x^2 + y^2 = 1$
		\item $(x-5)^2 + (x+3)^2 = 25$
  \end{enumerate}

\end{qstn}





























\end{document}
































