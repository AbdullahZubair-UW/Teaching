% Prepared by Calvin Kent
%
% Assignment Template v19.02
%
%%% 20xx0x/MATHxxx/Crowdmark/Ax
%
\documentclass[12pt]{article} %
\usepackage{amsthm}
\usepackage{CKpreamble}
\usepackage{CKassignment}
\usepackage{mdframed}
\usepackage{import}
\usepackage{pdfpages}
\usepackage{transparent}
\usepackage{xcolor}



\newcommand{\incfig}[2][1]{%
    \def\svgwidth{#1\columnwidth}
    \import{./figures/}{#2.pdf_tex}
}

\pdfsuppresswarningpagegroup=1

\newcounter{step}[section]
\newenvironment{step}[1][]
{\refstepcounter{step} \textbf{Step #1.}}


%
\begin{document}
	\pagenumbering{arabic}
	% Start of class settings ...
	\renewcommand*{\coursecode}{MATH 235} % renew course code
	\renewcommand*{\assgnnumber}{Assignment 1} % renew assignment number
	\renewcommand*{\submdate}{September 14, 2021} % renew the date
	\renewcommand*{\studentfname}{Abdullah} % Student first name
	\renewcommand*{\studentlname}{Zubair} % Student last name
    \renewcommand*{\proofname}{Proof:}
	% \renewcommand*{\studentnum}{20836288} % Student number

	\renewcommand\qedsymbol{$\blacksquare$}
	\setfigpath
	% End of class settings	
	% \pagestyle{crowdmark}
	\newgeometry{left=18mm, right=18mm, top=22mm, bottom=22mm} % page is set to default values
	\fancyhfoffset[L,O]{0pt} % header orientation fixed
	% End of class settings
	%%% Note to user:
	% CTRL + F <CHANGE ME:> (without the angular brackets) in CKpreamble to specify graphics paths accordingly.
	% The command \circled[]{} accepts one optional and one mandatory argument.
	% Optional argument is for the size of the circle and mandatory argument is for its contents.
	% \circled{A} produces circled A, with size drawn for letter A. \circled[TT]{A} produces circled A with size drawn for TT.
	% https://github.com/CalvinKent/My-LaTeX
	%%%

	%%%%%%%%%%%%%%%%%%%%%%%%%%%%%%%%%%%%%%%%%%%%%%%%%%%%%%%%%%%%%%%%%%%%%%%%%%%%%%%
	%%%                        CUSTOM MACRO VIM-TEX                             %%%
	%%       call IMAP('NOM', '\nomenclature{}', 'tex')               

	%%%%%%%%%%%%%%%%%%%%%%%%%%%%%%%%%%%%%%%%%%%%%%%%%%%%%%%%%%%%%%%%%%%%%%%%%%%%%%%

	% Crowdmark assignment start
	% qnumber, qname, qpoints

\begin{center}
		\Huge{\underline{\textbf{How to determine the inverse of a function}}}
\end{center}

So far we have been dealing with invertible functions equipped with relatively small domains and co-domains. In such a 
scenario, it was easy to have intuition for the inverse function, or to perhaps guess it and confirm with mapping tables,
However for domains like $\R,\Z,\N$, its not so obvious. Hence we would like to formulate an algorithm to assist
in determine the inverse function. Given an \textbf{invertible} function $f(x)$, to determine the inverse 
function  $f^{-1}(x)$, follow the procedure given below, 

\begin{step}[1]
  Replace $f(x)$ with the variable  $y$.
\end{step}

\begin{step}[2]
  Isolate for $x$.
\end{step}

\begin{step}[3]
  Replace $x$ with $f^{-1}(x)$, and replace $y$ with $x$. And your done!
\end{step}
\begin{ex} Determine the inverse function for the following function, \textbf{(In class)}
  
\begin{enumerate}[label=(\alph*)]
  \item $f(x) = 3x + 7$
  \item $g(x) = -2\sqrt{4x + 16} + 1$
  \item $h(x) = -\frac{4}{2x - 1} - 3$
\end{enumerate}
  
\end{ex}

\section*{Horizontal Line Test}
The \textbf{Horizontal Line Test} is a quick way to check weather or not a function is invertible (This is very
similar to the vertical line test). Preform the following, 
\begin{enumerate}
  \item Graph or sketch the function then draw a Horizontal through \textbf{the ENTIRE }the plot.
  \item \textbf{IF} the Horizontal line hits the graph \textbf{once}, then the function is invertible.
  \item \textbf{ELSE}, the function is \textbf{not} invertible.
\end{enumerate}

\begin{ex}
  Determine weather or not the following functions are invertible or not using the Horizontal line test,
  \textbf{(In class)},
  \begin{enumerate}[label=(\alph*)]
      \item $g(x) = \sqrt{x}$.
      \item $f(x) = 2x - 3$.
      \item $G(x) = x^2 - 3$.
  \end{enumerate}
\end{ex}

\newpage


\textbf{\underline{\Large{Practice Problems:}}}
\begin{qstn}
  Determine the inverse of the following functions,
  \begin{enumerate}[label=(\alph*)]
    \item $T(x) = \frac{1}{2x - 1} + 4$
    \item $H(x) = -2x + 22$.
    \item $F(x) = \sqrt{4x + 4}$.
    \item $\mathcal{L}(x) = \sqrt{4x - 1} + 7$
    \item $\mathcal{H}(x) = \frac{3}{4}x - 1$
    \item $\mathcal{F}(x) = -\frac{3}{x + 1} + 6$
    \item $\mathcal{P}(x) = -4\sqrt{2x + 8}$
  \end{enumerate}
\end{qstn}

\begin{qstn}
  A linear function passes through the points $(1,3)$ and $(2,5)$. Determine its inverse function.
\end{qstn}

\begin{qstn}
  Determine weather or not the following functions are invertible or not using the Horizontal line test,
  \begin{enumerate}[label=(\alph*)]
    \item $f(x) = -x + 3$.
    \item $g(x) = -x^2 + 6x - 8$.
    \item $h(x) = \left|x\right|$.
  \end{enumerate}
\end{qstn}

\begin{qstn}
  Textbook, Pg 138 Q4.
\end{qstn}

\begin{qstn}
  Textbook, Pg 138 Q12.
\end{qstn}

\begin{qstn}
  Textbook, Pg 138 Q18. \textbf{(Important Question *)}
\end{qstn}

\begin{qstn}
  Textbook, Pg 138 Q21.\textbf{(Important Question *)}
\end{qstn}


\newpage




\textbf{\underline{\Large{Solutions to Practice Problems:}}}\\
\textbf{Question 1.} \texttt{  }
  \begin{enumerate}[label=(\alph*)]
    \item $T^{-1}(x) = \frac{1}{2(y - 4)} + \frac{1}{2}$.
    \item $H^{-1}(x) = -\frac{1}{2}(x - 22)$
    \item $F^{-1}(x) = \frac{1}{4}(x^2 - 4)$
    \item $\mathcal{L}^{-1}(x) = \frac{1}{4}(x - 7)^2 + \frac{1}{4}$.
    \item $\mathcal{H}^{-1}(x) = \frac{4}{3}x + \frac{4}{2}$.
    \item $\mathcal{F}^{-1}(x) = -\frac{3}{x - 6} - 1$.
    \item $\mathcal{P}^{-1}(x) = \frac{x^2}{8} - 4$.
  \end{enumerate}


\textbf{Question 2.} $f^{-1}(x) = \frac{1}{2}(x - 1)$.


\textbf{Question 3.} \texttt{ }
\begin{enumerate}[label=(\alph*)]
  \item Invertible.
  \item Not Invertible.
  \item Not Invertible.
\end{enumerate}

\textbf{Question 4 - 7.} \textbf{(Refer to Textbook, ASK me if your confused by their answers)}


\end{document}

















