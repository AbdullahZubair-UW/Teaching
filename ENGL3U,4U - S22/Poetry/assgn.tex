% Prepared by Calvin Kent
%
% Assignment Template v19.02
%
%%% 20xx0x/MATHxxx/Crowdmark/Ax
%
\documentclass[12pt]{article} %
\usepackage{amsthm}
\usepackage{CKpreamble}
\usepackage{CKassignment}
\usepackage{dirtytalk}
\usepackage{csquotes}
\usepackage{mdframed}
\usepackage{euscript}
\usepackage{tikz}
\usepackage{pgfplots}

%
\begin{document}
	\pagenumbering{arabic}
	% Start of class settings ...
	\renewcommand*{\coursecode}{MATH 235} % renew course code
	\renewcommand*{\assgnnumber}{Assignment 1} % renew assignment number
	\renewcommand*{\submdate}{September 14, 2021} % renew the date
	\renewcommand*{\studentfname}{Abdullah} % Student first name
	\renewcommand*{\studentlname}{Zubair} % Student last name
    \renewcommand*{\proofname}{Proof:}
	% \renewcommand*{\studentnum}{20836288} % Student number

	\renewcommand\qedsymbol{$\blacksquare$}
	\setfigpath
	% End of class settings	
	% \pagestyle{crowdmark}
	\newgeometry{left=18mm, right=18mm, top=22mm, bottom=22mm} % page is set to default values
	\fancyhfoffset[L,O]{0pt} % header orientation fixed
	% End of class settings
	%%% Note to user:
	% CTRL + F <CHANGE ME:> (without the angular brackets) in CKpreamble to specify graphics paths accordingly.
	% The command \circled[]{} accepts one optional and one mandatory argument.
	% Optional argument is for the size of the circle and mandatory argument is for its contents.
	% \circled{A} produces circled A, with size drawn for letter A. \circled[TT]{A} produces circled A with size drawn for TT.
	% https://github.com/CalvinKent/My-LaTeX
	%%%

	%%%%%%%%%%%%%%%%%%%%%%%%%%%%%%%%%%%%%%%%%%%%%%%%%%%%%%%%%%%%%%%%%%%%%%%%%%%%%%%%%%%%%%%%%%%%%%%%%%%%%%%%%%%%%%%%%%
	%%%                        CUSTOM MACRO VIM-TEX                                                      (Word Wrap->
	%%       call IMAP('NOM', '\nomenclature{}', 'TeX')               

	%%%%%%%%%%%%%%%%%%%%%%%%%%%%%%%%%%%%%%%%%%%%%%%%%%%%%%%%%%%%%%%%%%%%%%%%%%%%%%%%%%%%%%%%%%%%%%%%%%%%%%%%%%%%%%%%%%

	% Crowdmark assignment start
	% qnumber, qname, points

\begin{center}
	\textbf{\underline{\Huge{\textit{Funeral Blues} Poem - Assignment}}}
\end{center}

\section*{Due Date: Thursday, March 31}

\section*{Preamble}

Answer the following question thoroughly with detail. Make sure to correctly cite your evidences.

\section*{Questions}
\begin{qstn}
  Research and provide contextual information related to this poem. (Author, time written, etc) \textbf{Be sure} to use
  proper ML citation.
\end{qstn}
\begin{qstn}
  Explain how and where the author uses \textit{imagery} throughout the poem. (\textbf{Be sure} to provide the corresponding verses)
\end{qstn}
\begin{qstn}
  Explain how and where the author uses \textit{metaphor's} throughout the poem. (\textbf{Be sure} to provide the corresponding verses)
\end{qstn}
\begin{qstn}
  Explain how and where the author uses \textit{personification} throughout the poem. (\textbf{Be sure} to provide the corresponding verses)
\end{qstn}
\begin{qstn}
  Describe three or four themes you have noticed throughout the poem.
\end{qstn}
\begin{qstn}
  Describe the tone/mood in first stanza.
\end{qstn}
\begin{qstn}
  Describe the tone/mood in second stanza.
\end{qstn}
\begin{qstn}
  Describe the tone/mood in third stanza.
\end{qstn}
\begin{qstn}
  Describe the tone/mood in fourth stanza.
\end{qstn}
\begin{qstn}
  Describe the significance of the title of the story and how its related to the poem.
\end{qstn}







\end{document}































