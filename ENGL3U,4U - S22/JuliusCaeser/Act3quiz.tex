% Prepared by Calvin Kent
%
% Assignment Template v19.02
%
%%% 20xx0x/MATHxxx/Crowdmark/Ax
%
\documentclass[12pt]{article} %
\usepackage{CKpreamble}
\usepackage{CKassignment}
\usepackage{tkz-euclide}
\usepackage{physunits}
\usepackage{physics}
\usepackage{lmodern}
\usepackage{microtype}
\usepackage{upgreek}
\usepackage{xcolor}
\usepackage{euscript}
\usepackage{tasks}
\usepackage{tkz-euclide}


\usepackage[misc]{ifsym}


%%Title
\title{Julius Caeser - Act 3 Quiz}
\date{April, 2022}

%%% Maths and science packages

\usepackage{amsmath,amsthm,amssymb}
\usepackage{pgfplots}
	\usetikzlibrary{
		calc,
		patterns,
		positioning
	}
	\pgfplotsset{
		compat=1.16,
		samples=200,
		clip=false,
		my axis style/.style={
			axis x line=middle,
			axis y line=middle,
			legend pos=outer north east,
			axis line style={
				->,
			},
			legend style={
				font=\footnotesize
			},
			label style={
				font=\footnotesize
			},
			tick label style={
				font=\footnotesize
			},
			xlabel style={
				at={
					(ticklabel* cs:1)
				},
				anchor=west,
				font=\footnotesize,
			},
			ylabel style={
				at={
					(ticklabel* cs:1)
				},
				anchor=west,
				font=\footnotesize,
			},
			xlabel= $x$,
			ylabel=$\vec d (\m \tx{[East]})$
		},
	}
	\tikzset{
		>=stealth
	}

%%% Tables and figures packages

\usepackage{float}
\usepackage{caption}
	\captionsetup{
		format=plain,
		labelfont=bf,
		font=small,
		justification=centering
	}

%%% Numbers and sets

\newcommand{\E}{\mathrm{e}}

\newcommand{\tx}[1]{\text{#1}}

\begin{document}
    \pagenumbering{arabic}
    % Start of class settings ...
    \renewcommand*{\coursecode}{MCR3U Quiz} % Quiz Title
    \renewcommand*{\assgnnumber}{2} % Quiz number
    \renewcommand*{\submdate}{January, 2022} % renew the date
    \renewcommand*{\studentfname}{\textbf{Name:}} % Student first name
    \renewcommand*{\studentlname}{} % Student last name
    %\renewcommand*{\studentnum}{SNumber} % Student number

    \renewcommand\qedsymbol{$\blacksquare$}
    \setfigpath
    % End of class settings
    \pagestyle{crowdmark}
    \newgeometry{left=18mm, right=18mm, top=22mm, bottom=22mm} % page is set to default values
    \fancyhfoffset[L,O]{0pt} % header orientation fixed
    % End of class settings
    %%% Note to user:
    % CTRL + F <CHANGE ME:> (without the angular brackets) in CKpreamble to specify graphics paths accordingly.
    % The command \circled[]{} accepts one optional and one mandatory argument.
    % Optional argument is for the size of the circle and mandatory argument is for its contents.
    % \circled{A} produces circled A, with size drawn for letter A. \circled[TT]{A} produces circled A with size drawn for TT.
    % https://github.com/CalvinKent/My-LaTeX
    %%%
    % Crowdmark assignment start


    %%%%%%%%%%%%%%%%%%%%%%%%%%%%%%%%%%%%%%%%%%%%%%%%%%%%%%%%%%%%%%%%%%%%%%%%%%%%%%%%%%%%%%%%%%%%%%%%%%%%%%%%
    %%%%%%%%%%%%%%%%%                  PROBLEM IDEAS                  %%%%%%%%%%%%%%%%%%%%%%%%%%%%%%%%%%%%%%
    %%%%%                   ----------------------------------------                                %%%%%%%%

    % --> Do a hard tangent line problem

	\maketitle
	\section{Preamble}
    This is a quiz covering all of Act 3 from \textit{Julius Caeser}.
	\section{Allowed Aids}
  This is a closed book quiz, nothing is allowed.
	\section{Name and Date:}
	Print your name and todays date below;

  \vspace*{0.4cm}

	\begin{center}
	\noindent\begin{tabular}{ll}
		\makebox[3in]{\hrulefill} & \makebox[3in]{\hrulefill}\\
		Name & Date\\[8ex]% adds space between the two sets of signatures
	\end{tabular}
	\end{center}
	\newpage

    \begin{qstn}
      Define the following terms;
      \begin{enumerate}[label=(\alph*)]
        \item \textbf{Puissant:} 
          \vspace*{1cm}
        \item \textbf{Repealing:}
          \vspace*{1cm}
        \item \textbf{Corse:}
          \vspace*{1cm}
        \item \textbf{Havoc:}
          \vspace*{1cm}
        \item \textbf{Couchings:}
          \vspace*{1cm}
      \end{enumerate}
    \end{qstn}

    \begin{qstn}
        What does Metellus Cimber ask Caeser?
    \end{qstn}
         \vspace*{2cm}
    \begin{qstn}
      Why do Cassius and Brutus say they did Caeser a favor a favor by killing him?
    \end{qstn}
         \vspace*{4cm}
    \begin{qstn}
      After the assassination, Mark Antony sent his servant to talk to Brutus. Why?
    \end{qstn}
         \vspace*{5cm}
    \begin{qstn}
        Who stabs Caeser first and last?
    \end{qstn}
         \vspace*{2cm}
    \begin{qstn}
       What does Antony predict will happen because of Caeser's death?
    \end{qstn}
         \vspace*{5cm}
    \begin{qstn}
       Who is Octavius Ceasear?
    \end{qstn}
         \vspace*{5cm}
    \begin{qstn}
       What does Antony say to the crowd of people to change their interpretation of Caeser?
    \end{qstn}
    \newpage
    \begin{qstn}
      Who arrives in Rome at the end of the scene, after Antony's speech?
    \end{qstn}
         \vspace*{2cm}
    \begin{qstn}
      Antony says "The evil that men do lives after them, The good is oft interred with their bones." What does this mean? Do you agree?
    \end{qstn}
         \vspace*{10cm}
    \begin{qstn}
      In scene III, who is Cinna? What happens to him?
    \end{qstn}



































\end{document}
