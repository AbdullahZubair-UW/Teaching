% Prepared by Calvin Kent
%
% Assignment Template v19.02
%
%%% 20xx0x/MATHxxx/Crowdmark/Ax
%
\documentclass[12pt]{article} %
\usepackage{CKpreamble}
\usepackage{CKassignment}
\usepackage{tkz-euclide}
\usepackage{physics}
\usepackage{lmodern}
\usepackage{microtype}
\usepackage{tasks}
\usepackage{upgreek}
\usepackage{xcolor}
\usepackage{euscript}
\usepackage{tasks}
\usepackage{upgreek}
\usepackage[misc]{ifsym}



\usepackage{pgfplots}
\usepgfplotslibrary{polar}
\usepgflibrary{shapes.geometric}
\usetikzlibrary{calc}


\usepackage{euscript}
\usepackage{microtype}
\usepackage{upgreek}
\usepackage[misc]{ifsym}

%%Title
\title{ENG4U/ENG3U Final Exam}
\date{April 22, 2022}

%%% Maths and science packages




%%% Tables and figures packages

\usepackage{float}
\usepackage{caption}
	\captionsetup{
		format=plain,
		labelfont=bf,
		font=small,
		justification=centering
	}

%%% Numbers and sets

\newcommand{\E}{\mathrm{e}}

\newcommand{\tx}[1]{\text{#1}}

\begin{document}
    \pagenumbering{arabic}
    % Start of class settings ...
    \renewcommand*{\coursecode}{MCR3U Quiz} % Quiz Title
    \renewcommand*{\assgnnumber}{1} % Quiz number
    \renewcommand*{\submdate}{November, 2021} % renew the date
    \renewcommand*{\studentfname}{\textbf{Name:}} % Student first name
    \renewcommand*{\studentlname}{} % Student last name
    %\renewcommand*{\studentnum}{SNumber} % Student number

    \renewcommand\qedsymbol{$\blacksquare$}
    \setfigpath
    % End of class settings
    \newgeometry{left=18mm, right=18mm, top=22mm, bottom=22mm} % page is set to default values
    \fancyhfoffset[L,O]{0pt} % header orientation fixed
    % End of class settings
    %%% Note to user:
    % CTRL + F <CHANGE ME:> (without the angular brackets) in CKpreamble to specify graphics paths accordingly.
    % The command \circled[]{} accepts one optional and one mandatory argument.
    % Optional argument is for the size of the circle and mandatory argument is for its contents.
    % \circled{A} produces circled A, with size drawn for letter A. \circled[TT]{A} produces circled A with size drawn for TT.
    % https://github.com/CalvinKent/My-LaTeX
    %%%
    % Crowdmark assignment start


    %%%%%%%%%%%%%%%%%%%%%%%%%%%%%%%%%%%%%%%%%%%%%%%%%%%%%%%%%%%%%%%%%%%%%%%%%%%%%%%%%%%%%%%%%%%%%%%%%%%%%%%%
    %%%%%%%%%%%%%%%%%                  PROBLEM IDEAS                  %%%%%%%%%%%%%%%%%%%%%%%%%%%%%%%%%%%%%%
    %%%%%                   ----------------------------------------                                %%%%%%%%

    % --> Do a hard tangent line problem

	\maketitle
	\section{Total Marks : 50}
	\section{Preamble}
	This final exam covers everything we have learned throughout the course.
	\section{Allowed Aids}
	The following aids are allowed on the Test
	\begin{itemize}
		\item Pencil, Pen, Eraser, Highlighter, Ruler, Protractor, Spare sheets of \textbf{blank} paper.
	\end{itemize}
  \section{Remarks:}
  \begin{itemize}
    \item When referencing quotations, \textbf{use} proper MLA citation.
    \item You \textbf{do not} have to reference exact line/page numbers.
    \item All answers are to be composed in a coherent, unified manner using a multi-paragraph structure.
    \item Write legible answers and use Standard English! Implement the Grammar techniques we have learnt throughout the term to the best of your ability.
    \item PROOFREAD all your work before submitting.
  \end{itemize}

	\section{Name and Date:}

  \vspace*{0.1cm}

	\begin{center}
	\noindent\begin{tabular}{ll}
		\makebox[3in]{\hrulefill} & \makebox[3in]{\hrulefill}\\
		Name & Date\\[8ex]% adds space between the two sets of signatures
	\end{tabular}
	\end{center}
	\newpage


\section*{Part A - Knowledge (19 marks)}

\begin{qstn}
    \textbf{(7 marks)} Define the following literary devices;
    \begin{enumerate}[label=(\alph*)]
      \item \textbf{Personification:}
            \vspace*{1cm}
      \item \textbf{Irony:}
            \vspace*{1cm}
      \item \textbf{Foreshadowing:}
            \vspace*{1cm}
      \item \textbf{Metaphor:}
            \vspace*{1cm}
      \item \textbf{Oxymoron:}
            \vspace*{1cm}

      \item \textbf{Imagery:}
            \vspace*{1cm}

      \item \textbf{Symbolism:}
            \vspace*{1cm}
    \end{enumerate}
\end{qstn}


\begin{qstn}
  \textbf{(4 marks)} We studied that, generally speaking,  there are four different types of Essays. Describe the \textbf{Narrative} type, in
  particular, describe the authors purpose, style and approach found in a narrative essay.
  \newpage

\end{qstn}


\begin{qstn}
  \textbf{(4 marks)} Describe the difference between a formal and informal essay. In particular, comment on the differences in the audience, the purpose/goal, how the thesis is stated and the overall tone.
\end{qstn}

\vspace*{9cm}

\begin{qstn}
  \textbf{(4 marks)} We studied various methods that authors use to develop information within their essays. Describe the \textbf{Cause \& Effect} method of development. In particular, comment on the manner in which arguments are developed as well as in which situation's this method would
  be useful.
\end{qstn}

\newpage

\section*{Part B - Poetry (7 Marks)}
\begin{qstn}
  Read the poem \textbf{``Night Clouds''} and answer the questions below;
  \begin{enumerate}[label=(\alph*)]
    \item \textbf{(1 mark)} Describe a theme you have observed throughout the poem.
            \vspace*{2cm}
    \item \textbf{(2 mark)} Describe the overall tone/mood of the poem
            \vspace*{2cm}
    \item \textbf{(2 mark)} Identify where the author uses \textit{Imagery} and describe the affect. (Provide at least two verses)
            \vspace*{7cm}
    \item \textbf{(2 mark)} Identify where the author uses \textit{Personification} and describe the affect. (Provide at least two verses)
          \newpage
  \end{enumerate}


\section*{Part C - Interpretation of Literature (12 Marks)}
Write a multi-paragraph ($\sim$ 300 words) response to \textbf{ONE OF THE FIVE} topics below;
  \begin{enumerate}
    \item Compare and contrast either of the following:
          \begin{align*}
            \text{Antony } \& \text{ Brutus}\\
            \text{Cassius } \& \text{ Brutus}
          \end{align*}
    \item Compare and contrast Ellen from \textit{The Lamp at Noon} and Portia from \textit{Julius Caeser}.
    \item Compare and contrast the funeral speeches for Caeser given by Brutus and Caeser. Describe the effect that both speeches had.
    \item Explain how Brutus's inability to make rational decisions lead to his destruction. (Provide at least two examples)
    \item Flattery is a tool used a number of times by characters in Julius Caeser. Gives examples of scenarios where flattery is used and
          describe their effect.
  \end{enumerate}

            \vspace*{1cm}

  \textbf{Response to Question : $\underline{ \hspace*{1cm}}$}

  \newpage
  \textbf{(Part C Continued)}
    \vspace*{19cm}

    \newpage

\section*{Part D - Analysis and Reflection (12 marks)}
Write a multi-paragraph ($\sim$ 300 words) response to \textbf{ONE OF THE FOUR} topics below. In your composition, you may apply
any effective and appropriate method of development which includes any combination of cause \& effect, compare \& contrast,
persuasion, description or narration. Your response may draw upon any aspect of your life : previous readings, your own
experiences, the experience of others, etc.

\begin{enumerate}
  \item The significance and dramatic irony of the following passage from Brutus;
        \begin{align*}
           \text{``I found no man but he was true to me''} \tag{V : V 35-40}
        \end{align*}
  \item ``The media is a power tool of persuasion''. Do you agree or disagree? Explain.
  \item Do you feel as if Santiago was a good role model for Manolin? Explain.
  \item Justify who you feel was the protagonist \& antagonist of \textit{Julius Caeser}.
\end{enumerate}

\vspace*{1cm}

\textbf{Response to Question : $\underline{ \hspace*{1cm}}$}

\newpage
\textbf{(Part D Continued)}
\vspace*{19cm}

\newpage





\end{qstn}
























































\end{document}
