% Prepared by Calvin Kent
%
% Assignment Template v19.02
%
%%% 20xx0x/MATHxxx/Crowdmark/Ax
%
\documentclass[12pt]{article} %
\usepackage{amsthm}
\usepackage{CKpreamble}
\usepackage{CKassignment}
\usepackage{dirtytalk}
\usepackage{csquotes}
\usepackage{dirtytalk}
\usepackage{mdframed}
\usepackage{euscript}
\usepackage{tikz}
\usepackage{pgfplots}

%
\begin{document}
	\pagenumbering{arabic}
	% Start of class settings ...
	\renewcommand*{\coursecode}{MATH 235} % renew course code
	\renewcommand*{\assgnnumber}{Assignment 1} % renew assignment number
	\renewcommand*{\submdate}{September 14, 2021} % renew the date
	\renewcommand*{\studentfname}{Abdullah} % Student first name
	\renewcommand*{\studentlname}{Zubair} % Student last name
    \renewcommand*{\proofname}{Proof:}
	% \renewcommand*{\studentnum}{20836288} % Student number

	\renewcommand\qedsymbol{$\blacksquare$}
	\setfigpath
	% End of class settings	
	% \pagestyle{crowdmark} 
	\newgeometry{left=18mm, right=18mm, top=22mm, bottom=22mm} % page is set to default values
	\fancyhfoffset[L,O]{0pt} % header orientation fixed
	% End of class settings
	%%% Note to user:
	% CTRL + F <CHANGE ME:> (without the angular brackets) in CKpreamble to specify graphics paths accordingly.
	% The command \circled[]{} accepts one optional and one mandatory argument.
	% Optional argument is for the size of the circle and mandatory argument is for its contents.
	% \circled{A} produces circled A, with size drawn for letter A. \circled[TT]{A} produces circled A with size drawn for TT.
	% https://github.com/CalvinKent/My-LaTeX
	%%%

	%%%%%%%%%%%%%%%%%%%%%%%%%%%%%%%%%%%%%%%%%%%%%%%%%%%%%%%%%%%%%%%%%%%%%%%%%%%%%%%%%%%%%%%%%%%%%%%%%%%%%%%%%%%%%%%%%%
	%%%                        CUSTOM MACRO VIM-TEX                                                      (Word Wrap->
	%%       call IMAP('NOM', '\nomenclature{}', 'tex')               

	%%%%%%%%%%%%%%%%%%%%%%%%%%%%%%%%%%%%%%%%%%%%%%%%%%%%%%%%%%%%%%%%%%%%%%%%%%%%%%%%%%%%%%%%%%%%%%%%%%%%%%%%%%%%%%%%%%

	% Crowdmark assignment start
	% qnumber, qname, points


\begin{center}
  \textbf{\underline{\Huge{Grammar Lesson 8 - Quotation}}}
\end{center}

\section*{General Rules}
\begin{enumerate}
  \item Within your essay, if there exists speech from outside the narrative perspective, enclose it in a set of\
  \textbf{double quotes}. (``'').\\
  \underline{Example:} ``How did you do on your final?'', my roommate asked.

  \item If there exists a quote within a quote, nest the second quote with \textbf{single quotes}. \\
  \underline{Example:} ``He told me 'Don't go there''', my brother said.
\end{enumerate}

\section*{Long Direct Quotation:}
\begin{enumerate}
  \item When there exists a dialogue, enclose each excerpt in a set of quotations, insert a newline each time the speaker
  changes.
  \item \textbf{If} any piece of a dialogue or quotation exceeds \textbf{four} lines, separate it from the essay into a
  separate block of text \textbf{with} quotations.
\end{enumerate}

\section*{Direct Quotes:}
\begin{enumerate}
  \item A direct quotation needs to start with a capital letter.
  \item \textbf{If} it is divided into two or more parts then the second part should start with the lower case letter.
  \item Separate main clause from quotes with \textbf{commas}.
  \item \textbf{Include} all punctuation/periods within the quotes.
  \item Semicolons and colons are placed \textbf{outside} of the quotes.
\end{enumerate}

\section*{Indirect Quotes:}
Indirect quotes indicate/paraphrase what the speaker was saying, \textbf{not} exactly what he/she is saying.
\begin{enumerate}
  \item \textbf{Do not} enclose indirect quotes in quotation marks.
\end{enumerate}

\newpage

\section*{General Usage of Double Quotes:}
Other uses of double quotes include; 
  \begin{enumerate}
    \item Titles of short stories, short poems, one-act plays, short films, songs,
    television episodes, essays, articles, and other short works.\\
    \underline{Examples:} "Hills Like White Elephants" is a story by Hemingway.

    \item \textbf{Word Definitions}, whenever you define a word.\\
    \underline{Example:} The word accept means "to agree or receive favourably."
  \end{enumerate}



  

  


\vspace*{2cm}




\textit{Quotation}, Effective Writing Practice Tutorial, Northern Illinois University.\\

\end{document}
































































