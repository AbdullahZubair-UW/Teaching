% Prepared by Calvin Kent
%
% Assignment Template v19.02
%
%%% 20xx0x/MATHxxx/Crowdmark/Ax
%
\documentclass[12pt]{article} %
\usepackage{amsthm}
\usepackage{CKpreamble}
\usepackage{CKassignment}
\usepackage{dirtytalk}
\usepackage{csquotes}
\usepackage{dirtytalk}
\usepackage{mdframed}
\usepackage{euscript}
\usepackage{tikz}
\usepackage{pgfplots}

%
\begin{document}
	\pagenumbering{arabic}
	% Start of class settings ...
	\renewcommand*{\coursecode}{MATH 235} % renew course code
	\renewcommand*{\assgnnumber}{Assignment 1} % renew assignment number
	\renewcommand*{\submdate}{September 14, 2021} % renew the date
	\renewcommand*{\studentfname}{Abdullah} % Student first name
	\renewcommand*{\studentlname}{Zubair} % Student last name
    \renewcommand*{\proofname}{Proof:}
	% \renewcommand*{\studentnum}{20836288} % Student number

	\renewcommand\qedsymbol{$\blacksquare$}
	\setfigpath
	% End of class settings	
	% \pagestyle{crowdmark}
	\newgeometry{left=18mm, right=18mm, top=22mm, bottom=22mm} % page is set to default values
	\fancyhfoffset[L,O]{0pt} % header orientation fixed
	% End of class settings
	%%% Note to user:
	% CTRL + F <CHANGE ME:> (without the angular brackets) in CKpreamble to specify graphics paths accordingly.
	% The command \circled[]{} accepts one optional and one mandatory argument.
	% Optional argument is for the size of the circle and mandatory argument is for its contents.
	% \circled{A} produces circled A, with size drawn for letter A. \circled[TT]{A} produces circled A with size drawn for TT.
	% https://github.com/CalvinKent/My-LaTeX
	%%%

	%%%%%%%%%%%%%%%%%%%%%%%%%%%%%%%%%%%%%%%%%%%%%%%%%%%%%%%%%%%%%%%%%%%%%%%%%%%%%%%%%%%%%%%%%%%%%%%%%%%%%%%%%%%%%%%%%%
	%%%                        CUSTOM MACRO VIM-TEX                                                      (Word Wrap->
	%%       call IMAP('NOM', '\nomenclature{}', 'tex')               

	%%%%%%%%%%%%%%%%%%%%%%%%%%%%%%%%%%%%%%%%%%%%%%%%%%%%%%%%%%%%%%%%%%%%%%%%%%%%%%%%%%%%%%%%%%%%%%%%%%%%%%%%%%%%%%%%%%

	% Crowdmark assignment start
	% qnumber, qname, points

\begin{center}
  \textbf{\underline{\Huge{Grammar Lesson 4 - Pronoun Agreement}}}
\end{center}

In this grammar lesson, we will analyze the importance of pronoun agreement and a few mistakes that are common for students.\\

\textbf{Pronoun:} A word that substitutes a noun or a noun phrase for reference.\\
\textit{Examples:} I, he, she, you, me, we, them, they,\ldots

In a sentence, pronouns \textbf{must agree in gender, person and plural or singular tenses}. For example,

\section{Agreement in Number}

\say{Neither of my classmates are taking the trip this summer.}

\textbf{Question: }What is wrong about the above quote?\\
\textbf{Correct: }"Neither of my classmates \underline{is} taking the trip this summer."

The following indefinite pronouns are always \textbf{singular},

\begin{multicols}{4}
  \begin{itemize}
    \item anybody
    \item anything
    \item either
    \item nobody
      \columnbreak
    \item everybody
    \item everything
    \item neither
    \item someone
      \columnbreak
    \item anyone
    \item another
    \item one
    \item somebody
      \columnbreak
    \item everyone
    \item each
    \item no one.
    \end{itemize}
\end{multicols}

\say{Everybody has been bringing in their own lunch lately.}

\textbf{Question: }What is wrong about the above quote?\\
\textbf{Correct: }"Everybody has been bringing in \underline{his or her} own lunch lately."


\say{Neither of the best players in the last game were injured.}


\textbf{Question:} What is wrong with the above quote?\\
\textbf{Correct: }"Neither of the best players in the last game was injured."


The following pronouns are always \textbf{plural},
\begin{multicols}{4}
  \begin{itemize}
    \item both
      \columnbreak
    \item few
      \columnbreak
    \item several
      \columnbreak
    \item many
  \end{itemize}
\end{multicols}

\textbf{Correct:} "Several of the participants shared their personal experiences."

\newpage

\subsection{Note on Has vs Have}
The participle \textbf{have} is used when,
\begin{itemize}
  \item Speaking in the first person (I, we)
  \item Speaking in the second person (you)
  \item Speaking in the third person \textit{plural}(they)
\end{itemize}

The participle \textbf{has} is used when,

\begin{itemize}
  \item Speaking in the third person singular (he, she, it)
\end{itemize}

\vspace*{1cm}

The following pronouns \textit{can} be both \textbf{singular and plural} depending on the kind of noun they refer to,
\begin{multicols}{5}
  \begin{itemize}
    \item all
      \columnbreak
    \item any
      \columnbreak
    \item most
      \columnbreak
    \item none
      \columnbreak
    \item some
  \end{itemize}
\end{multicols}


\textbf{Correct: }"None of the food has been left after the party."
\textbf{Correct: }"None of the players have quit the team after a difficult season."


\section{Agreement in Gender}
To avoid conflict between assumption of gender, try your best to generalize it by \textit{he or she},

\textbf{Correct: }"Each faculty member of the department is encouraged to share \underline{his or her} grant proposals with the committee."


\section{Agreement in Person}
Whenever a sentence is composed of a multiple instances of pronoun references that refer to the same noun, they should \textbf{all} agree in person. For example, since 
\textit{one, everyone, everybody} are third person \textbf{singular} pronouns, they should be followed up by \textit{he, his, him or she, her, hers}.


\say{One should carefully consider your choice of major.}

\textbf{Question: }What is wrong about the above quote?\\
\textbf{Correct: }"One should carefully consider \underline{his or her} choice of major."



\textit{Pronoun Agreement}, Effective Writing Practice Tutorial, Northern Illinois University.\\
\textit{“Have” vs. “Has”: When To Use Each One}, Thesaurus, September 11, 2020. 


  











\end{document}































