% Prepared by Calvin Kent
%
% Assignment Template v19.02
%
%%% 20xx0x/MATHxxx/Crowdmark/Ax
%
\documentclass[12pt]{article} %
\usepackage{amsthm}
\usepackage{CKpreamble}
\usepackage{CKassignment}
\usepackage{dirtytalk}
\usepackage{csquotes}
\usepackage{dirtytalk}
\usepackage{mdframed}
\usepackage{euscript}
\usepackage{tikz}
\usepackage{pgfplots}

%
\begin{document}
	\pagenumbering{arabic}
	% Start of class settings ...
	\renewcommand*{\coursecode}{MATH 235} % renew course code
	\renewcommand*{\assgnnumber}{Assignment 1} % renew assignment number
	\renewcommand*{\submdate}{September 14, 2021} % renew the date
	\renewcommand*{\studentfname}{Abdullah} % Student first name
	\renewcommand*{\studentlname}{Zubair} % Student last name
    \renewcommand*{\proofname}{Proof:}
	% \renewcommand*{\studentnum}{20836288} % Student number

	\renewcommand\qedsymbol{$\blacksquare$}
	\setfigpath
	% End of class settings	
	% \pagestyle{crowdmark}
	\newgeometry{left=18mm, right=18mm, top=22mm, bottom=22mm} % page is set to default values
	\fancyhfoffset[L,O]{0pt} % header orientation fixed
	% End of class settings
	%%% Note to user:
	% CTRL + F <CHANGE ME:> (without the angular brackets) in CKpreamble to specify graphics paths accordingly.
	% The command \circled[]{} accepts one optional and one mandatory argument.
	% Optional argument is for the size of the circle and mandatory argument is for its contents.
	% \circled{A} produces circled A, with size drawn for letter A. \circled[TT]{A} produces circled A with size drawn for TT.
	% https://github.com/CalvinKent/My-LaTeX
	%%%

	%%%%%%%%%%%%%%%%%%%%%%%%%%%%%%%%%%%%%%%%%%%%%%%%%%%%%%%%%%%%%%%%%%%%%%%%%%%%%%%%%%%%%%%%%%%%%%%%%%%%%%%%%%%%%%%%%%
	%%%                        CUSTOM MACRO VIM-TEX                                                      (Word Wrap->
	%%       call IMAP('NOM', '\nomenclature{}', 'tex')               

	%%%%%%%%%%%%%%%%%%%%%%%%%%%%%%%%%%%%%%%%%%%%%%%%%%%%%%%%%%%%%%%%%%%%%%%%%%%%%%%%%%%%%%%%%%%%%%%%%%%%%%%%%%%%%%%%%%

	% Crowdmark assignment start
	% qnumber, qname, points

\begin{center}
  \textbf{\underline{\Huge{Grammar Lesson 2 - Semicolon}}}
\end{center}

In this grammar lesson, we will learn how to correctly use the semicolon.

\begin{enumerate}
  \item \textbf{Semicolons with independent clauses}:\\
    Semicolons can replace a sentence containing a \underline{comma} as well as a \underline{coordinating conjunction}. 
    For example, in the following sentence we can replace the comma and the coordinating conjunction with, 

    \say{John finished all of his homework, but Ron did not finish his.}

    \say{John finished all of his homework; Ron did not finish his.}

    \vspace*{2cm}

    \textbf{Note:} Do not use a semicolon between a dependent an independent clause.

    \textit{Correct} : "Although Nate is a good worker, that new guy is not" \\
    \textit{Incorrect} : "Although Nate is a good worker; that new guy is not" 

    \textbf{What is wrong:} Because a semicolon is supposed to assume the role of a coordinating conjunction, the semicolon makes it
    sound like "Although Nate is a good worker, and that new guy is not" (or any other coordinating conjunction), the semicolon in this case 
    makes the "new guy" independent from Nate, while we are trying to judge the new guy relative to Nate.


  

  \item \textbf{Semicolons with conjunctive adverbs} : \\
    \textit{conjunctive adverbs:} For example, for instance, that is, besides, furthermore, moreover, otherwise, however, thus, therefore.

    Instead of ending an idea with a period and starting a new sentence with a conjunctive adverb related to the previous sentence, you can simply suffix
    the first sentence with a semicolon. For example,

    \say{Mary worked tirelessly on all her homework; nevertheless, she was unable to finish it.}

    \say{Harvey is a good driver; moreover, he is a friendly one.}

    \vspace*{3cm}


\end{enumerate}

\textbf{Citation}: 

Semicolon, Effective Writing Practice Tutorial, Northern Illinois University.


  











\end{document}































