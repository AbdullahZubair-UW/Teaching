% Prepared by Calvin Kent
%
% Assignment Template v19.02
%
%%% 20xx0x/MATHxxx/Crowdmark/Ax
%
\documentclass[12pt]{article} %
\usepackage{amsthm}
\usepackage{CKpreamble}
\usepackage{CKassignment}
\usepackage{dirtytalk}
\usepackage{csquotes}
\usepackage{dirtytalk}
\usepackage{mdframed}
\usepackage{euscript}
\usepackage{tikz}
\usepackage{pgfplots}

%
\begin{document}
	\pagenumbering{arabic}
	% Start of class settings ...
	\renewcommand*{\coursecode}{MATH 235} % renew course code
	\renewcommand*{\assgnnumber}{Assignment 1} % renew assignment number
	\renewcommand*{\submdate}{September 14, 2021} % renew the date
	\renewcommand*{\studentfname}{Abdullah} % Student first name
	\renewcommand*{\studentlname}{Zubair} % Student last name
    \renewcommand*{\proofname}{Proof:}
	% \renewcommand*{\studentnum}{20836288} % Student number

	\renewcommand\qedsymbol{$\blacksquare$}
	\setfigpath
	% End of class settings	
	% \pagestyle{crowdmark}
	\newgeometry{left=18mm, right=18mm, top=22mm, bottom=22mm} % page is set to default values
	\fancyhfoffset[L,O]{0pt} % header orientation fixed
	% End of class settings
	%%% Note to user:
	% CTRL + F <CHANGE ME:> (without the angular brackets) in CKpreamble to specify graphics paths accordingly.
	% The command \circled[]{} accepts one optional and one mandatory argument.
	% Optional argument is for the size of the circle and mandatory argument is for its contents.
	% \circled{A} produces circled A, with size drawn for letter A. \circled[TT]{A} produces circled A with size drawn for TT.
	% https://github.com/CalvinKent/My-LaTeX
	%%%

	%%%%%%%%%%%%%%%%%%%%%%%%%%%%%%%%%%%%%%%%%%%%%%%%%%%%%%%%%%%%%%%%%%%%%%%%%%%%%%%%%%%%%%%%%%%%%%%%%%%%%%%%%%%%%%%%%%
	%%%                        CUSTOM MACRO VIM-TEX                                                      (Word Wrap->
	%%       call IMAP('NOM', '\nomenclature{}', 'tex')               

	%%%%%%%%%%%%%%%%%%%%%%%%%%%%%%%%%%%%%%%%%%%%%%%%%%%%%%%%%%%%%%%%%%%%%%%%%%%%%%%%%%%%%%%%%%%%%%%%%%%%%%%%%%%%%%%%%%

	% Crowdmark assignment start
	% qnumber, qname, points


\begin{center}
  \textbf{\underline{\Huge{Grammar Lesson 7 - Split Infinitives}}}
\end{center}

\textbf{Infinitive:} A verb preceded by the word \textbf{to}. \\
\underline{Examples:} to write, to read, to take, to examine.

\textbf{Split Infinitive:} A situation where given an infinitive, an \textit{adverb} comes between the word to and the verb
itself.

\textbf{**AVOID splitting infinitives in formal writing.}, in everyday speech, we commonly split infinitives abundantly and
this is generally speaking acceptable. 

\begin{enumerate}
  \item Example:\\
  \textit{Incorrect} : She decided \textbf{to instantly quit} her job.\\
  \textit{Correct} : She decided to quit her job \textbf{instantly}.

 \item Don't split infinitives for ``degree adverbs'' such as \textbf{completely, entirely, unduly}.\\
 \textit{Incorrect} : It's hard \textbf{to completely follow} his reasoning.\\
 \textit{Correct} : It's hard to follow his reasoning \textbf{completely}.

  
  \item At times, \textbf{not} splitting infinitives creates ambiguity.
  \begin{itemize}
    \item The patient was told \textbf{to occasionally monitor} her blood sugar level.
    \item The patient was told \textbf{occasionally to monitor} her blood sugar level.\\
      (This slightly changes the meaning as the word \textbf{occasionally} is modifying the verb \textbf{told}.)
    \item The patient was told to monitor her blood sugar level \textbf{occasionally}.\\
     (Here both views may be correct, one could argue that \textbf{occasionally} is modifying either the verb\textbf{told}
     \textbf{monitor}, hence it is ambiguous).
    \item (\textbf{**}) The patient was told \textbf{to monitor occasionally} her blood sugar level.
  \end{itemize}
  
  \item Avoid sentences with more than one infinitive.\\
  \textit{Avoid} : Our company decided to legally and rightfully seek damages for fraudulent use of the company documents.\\
  If necessary, then reword as follows,\\
  \textit{Correct} : Legally and rightfully, our company decided to seek damages for fraudulent use of the company documents.\\
  \textit{Correct} : Our company decided to seek damages in a legal and rightful way for fraudulent use of the company documents.

  

  
\end{enumerate}


\vspace*{2cm}




\textit{Split Infinitives}, Effective Writing Practice Tutorial, Northern Illinois University.\\

\end{document}































































