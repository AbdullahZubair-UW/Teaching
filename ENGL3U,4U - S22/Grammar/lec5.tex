% Prepared by Calvin Kent
%
% Assignment Template v19.02
%
%%% 20xx0x/MATHxxx/Crowdmark/Ax
%
\documentclass[12pt]{article} %
\usepackage{amsthm}
\usepackage{CKpreamble}
\usepackage{CKassignment}
\usepackage{dirtytalk}
\usepackage{csquotes}
\usepackage{dirtytalk}
\usepackage{mdframed}
\usepackage{euscript}
\usepackage{tikz}
\usepackage{pgfplots}

%
\begin{document}
	\pagenumbering{arabic}
	% Start of class settings ...
	\renewcommand*{\coursecode}{MATH 235} % renew course code
	\renewcommand*{\assgnnumber}{Assignment 1} % renew assignment number
	\renewcommand*{\submdate}{September 14, 2021} % renew the date
	\renewcommand*{\studentfname}{Abdullah} % Student first name
	\renewcommand*{\studentlname}{Zubair} % Student last name
    \renewcommand*{\proofname}{Proof:}
	% \renewcommand*{\studentnum}{20836288} % Student number

	\renewcommand\qedsymbol{$\blacksquare$}
	\setfigpath
	% End of class settings	
	% \pagestyle{crowdmark}
	\newgeometry{left=18mm, right=18mm, top=22mm, bottom=22mm} % page is set to default values
	\fancyhfoffset[L,O]{0pt} % header orientation fixed
	% End of class settings
	%%% Note to user:
	% CTRL + F <CHANGE ME:> (without the angular brackets) in CKpreamble to specify graphics paths accordingly.
	% The command \circled[]{} accepts one optional and one mandatory argument.
	% Optional argument is for the size of the circle and mandatory argument is for its contents.
	% \circled{A} produces circled A, with size drawn for letter A. \circled[TT]{A} produces circled A with size drawn for TT.
	% https://github.com/CalvinKent/My-LaTeX
	%%%

	%%%%%%%%%%%%%%%%%%%%%%%%%%%%%%%%%%%%%%%%%%%%%%%%%%%%%%%%%%%%%%%%%%%%%%%%%%%%%%%%%%%%%%%%%%%%%%%%%%%%%%%%%%%%%%%%%%
	%%%                        CUSTOM MACRO VIM-TEX                                                      (Word Wrap->
	%%       call IMAP('NOM', '\nomenclature{}', 'tex')               

	%%%%%%%%%%%%%%%%%%%%%%%%%%%%%%%%%%%%%%%%%%%%%%%%%%%%%%%%%%%%%%%%%%%%%%%%%%%%%%%%%%%%%%%%%%%%%%%%%%%%%%%%%%%%%%%%%%

	% Crowdmark assignment start
	% qnumber, qname, points

\begin{center}
  \textbf{\underline{\Huge{Grammar Lesson 5 - Subject-Verb Agreement}}}
\end{center}

\section{Agreement in Number}

Any sentence composed of a subject and verb composition must ensure that the subject and verb agree in \textbf{person and number}.

\say{My friend, with his parents, are flying in today to visit me and my family.}

\textbf{Question:} What is wrong with this quote?
\begin{itemize}
  \item \textit{friend} is a singular subject.
  \item \textit{his parents} is a \textbf{prepositional phrase}.
  \item \textit{are flying} is in plural from.
\end{itemize}
Hence the subject and verb fail to agree on the number clause.

\textbf{Correct:} "My friend, with his parents, \underline{is} flying in today to visit me and my family".

\subsection*{Note on Prepositional phrases}

Prepositional phrases such as \textbf{with, together, along with, as well as} are NOT part of the subject and have NO
effect on the form of the verb. \textit{Only} focus on the original subject of the sentence. 

\say{Laura, together with a friend, practice yoga every day.}

\textbf{Question:} What is wrong with this quote?
\begin{itemize}
  \item Laura is singular subject.
  \item \textit{together with} is a prepositional phrase. 
  \item \textit{practice} is used in the plural sense.
\end{itemize}

\textbf{Correct:} \say{Laura, together with a friend, \underline{practices} yoga every day.}

\newpage

\section{Compound Subjects}

Compound sentences joined by \textbf{and} need a plural verb.

\textbf{Correct:} "Healthy diet and regular exercise are a necessity for a longer, happier life."

When the compound subject is joined by \textbf{or, nor, neither... nor, either... or}
and one part of the compound subject is singular and the other part is plural,
the verb needs to agree with the part closest to it.


\say{Neither students nor their teacher are participating in this play.}

\textbf{Question:} What is wrong with this quote?
\begin{itemize}
  \item \textit{Teacher} is a singular subject.
  \item \textit{are} is used in the plural sense
\end{itemize}

\textbf{Correct:} \say{Neither students nor their teacher \underline{is} participating in this play.}

\section{Subjects following verbs}
If the subject is following the verb in the sentence, rather than preceding it,
it still has to agree with it in number

\say{Here is my test scores.}

\textbf{Question:} What is wrong with this quote?
\begin{itemize}
  \item \textit{my test scores} are the plural subject
  \item \textit{is} is used in the singular sense
\end{itemize}

\textbf{Correct:} \say{Here \underline{are} my test scores.}

\section{Collective Nouns}
When referring to group nouns, if you are referring to the group as an entire unit, then the group takes the singular 
case, if you are referring to individual members of the group, then the plural case is used.

\textbf{Correct:} \say{The diversity committee was well represented at the last board meeting.}\\
\textbf{Correct:} \say{The committee were voting on the representative to the board of directors' meeting.}\\
\textbf{Correct:} \say{The audience was cheering the performer.}

\newpage

\section{Nouns with Plural Form}
There are a handful of nouns that only exist in the plural form like; \textbf{politics, news, ethics, measles}. As a subject,
we actually treat these nouns as singular.

\textbf{Correct:} \say{Ethics is an important component of human study.}  \\
\textbf{Correct:} \say{Politics was a leading cause for much of global conflict}  


\textit{Subject-Verb Agreement}, Effective Writing Practice Tutorial, Northern Illinois University.\\


  











\end{document}































