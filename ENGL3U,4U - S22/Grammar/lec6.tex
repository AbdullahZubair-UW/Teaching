% Prepared by Calvin Kent
%
% Assignment Template v19.02
%
%%% 20xx0x/MATHxxx/Crowdmark/Ax
%
\documentclass[12pt]{article} %
\usepackage{amsthm}
\usepackage{CKpreamble}
\usepackage{CKassignment}
\usepackage{dirtytalk}
\usepackage{csquotes}
\usepackage{dirtytalk}
\usepackage{mdframed}
\usepackage{euscript}
\usepackage{tikz}
\usepackage{pgfplots}

%
\begin{document}
	\pagenumbering{arabic}
	% Start of class settings ...
	\renewcommand*{\coursecode}{MATH 235} % renew course code
	\renewcommand*{\assgnnumber}{Assignment 1} % renew assignment number
	\renewcommand*{\submdate}{September 14, 2021} % renew the date
	\renewcommand*{\studentfname}{Abdullah} % Student first name
	\renewcommand*{\studentlname}{Zubair} % Student last name
    \renewcommand*{\proofname}{Proof:}
	% \renewcommand*{\studentnum}{20836288} % Student number

	\renewcommand\qedsymbol{$\blacksquare$}
	\setfigpath
	% End of class settings	
	% \pagestyle{crowdmark}
	\newgeometry{left=18mm, right=18mm, top=22mm, bottom=22mm} % page is set to default values
	\fancyhfoffset[L,O]{0pt} % header orientation fixed
	% End of class settings
	%%% Note to user:
	% CTRL + F <CHANGE ME:> (without the angular brackets) in CKpreamble to specify graphics paths accordingly.
	% The command \circled[]{} accepts one optional and one mandatory argument.
	% Optional argument is for the size of the circle and mandatory argument is for its contents.
	% \circled{A} produces circled A, with size drawn for letter A. \circled[TT]{A} produces circled A with size drawn for TT.
	% https://github.com/CalvinKent/My-LaTeX
	%%%

	%%%%%%%%%%%%%%%%%%%%%%%%%%%%%%%%%%%%%%%%%%%%%%%%%%%%%%%%%%%%%%%%%%%%%%%%%%%%%%%%%%%%%%%%%%%%%%%%%%%%%%%%%%%%%%%%%%
	%%%                        CUSTOM MACRO VIM-TEX                                                      (Word Wrap->
	%%       call IMAP('NOM', '\nomenclature{}', 'tex')               

	%%%%%%%%%%%%%%%%%%%%%%%%%%%%%%%%%%%%%%%%%%%%%%%%%%%%%%%%%%%%%%%%%%%%%%%%%%%%%%%%%%%%%%%%%%%%%%%%%%%%%%%%%%%%%%%%%%

	% Crowdmark assignment start
	% qnumber, qname, points


\begin{center}
  \textbf{\underline{\Huge{Grammar Lesson 6 - Relative Clauses}}}
\end{center}

\textbf{Relative Clause:} A subordinate clause that modifies a noun.\\
\underline{Examples:}
\begin{itemize}
  \item The memo was sent to all instructors who were teaching Spanish 101.
  \item The book which I borrowed from the library last semester is due this week.
\end{itemize}

There are two types of relative clauses; \textbf{non-restrictive and restrictive}.

\section{Restrictive}
\begin{enumerate}
  \item When the noun being modified is a \textbf{person and subject} of the clause, use \textbf{who}.
    \begin{itemize}
      \item The man who robbed the bank was arrested the next day.
    \end{itemize}
  \item If the preceding word is \textit{all, everyone, no one, nobody, those} use \textbf{that}.
    \begin{itemize}
      \item Those who/that took a class online said they would take an online class again.
    \end{itemize}
    \item When the noun being modified is a \textbf{person and object} of the clause, use \textbf{who, that,
    whom} or the relative pronoun may be omitted, \textbf{whom is most formal}.
  \begin{itemize}
    \item The interns whom this company employs come from a number of Canadian schools.
  \end{itemize}
    
  \item When the noun modified is a \textbf{thing and subject} of that clause, which or that can be used. The use of which is more formal.
  \begin{itemize}
    \item This is the package that/which I have been waiting for.
  \end{itemize}

  \item When the noun being modified is a \textbf{thing and object} of the clause, use \textbf{which, that,
    whom} or the relative pronoun may be omitted.
  \begin{itemize}
    \item The car which I rented last week broke down.
    \item This is the best book \textbf{that} I have ever read.
    \item This is the best book I have ever read.
  \end{itemize}
  
\end{enumerate}

\newpage

\section{Non-Restrictive}
\begin{enumerate}
  \item When the noun being modified is a \textbf{person and subject} of the clause, use \textbf{who}.
  \begin{itemize}
    \item Anne, who is an outstanding researcher, presented her paper on the health effects of air pollution at the Conference on Global Warming.
  \end{itemize}

  \item When the noun modified is a \textbf{person and object} of the clause, use \textbf{whom}.\\ (Whom is more formal, however
   in spoken English, who is frequently possible.)
   \begin{itemize}
     \item Anne, whom everyone respects, was invited to speak at a conference on the health effects of air pollution.
   \end{itemize}

   \item When the noun modified is a \textbf{thing and either an subject, object or objects} , use \textbf{which}.
   \begin{itemize}
     \item The book, which took years to write, was an instant hit.
     \item For my birthday, she gave me a book, which she picked out herself.
   \end{itemize}
\end{enumerate}


\vspace*{2cm}




\textit{Subject-Verb Agreement}, Effective Writing Practice Tutorial, Northern Illinois University.\\

\end{document}































