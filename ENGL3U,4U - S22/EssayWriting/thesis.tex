% Prepared by Calvin Kent
%
% Assignment Template v19.02
%
%%% 20xx0x/MATHxxx/Crowdmark/Ax
%
\documentclass[12pt]{article} %
\usepackage{amsthm}
\usepackage{CKpreamble}
\usepackage{CKassignment}
\usepackage{dirtytalk}
\usepackage{csquotes}
\usepackage{mdframed}
\usepackage{euscript}
\usepackage{tikz}
\usepackage{pgfplots}

%
\begin{document}
	\pagenumbering{arabic}
	% Start of class settings ...
	\renewcommand*{\coursecode}{MATH 235} % renew course code
	\renewcommand*{\assgnnumber}{Assignment 1} % renew assignment number
	\renewcommand*{\submdate}{September 14, 2021} % renew the date
	\renewcommand*{\studentfname}{Abdullah} % Student first name
	\renewcommand*{\studentlname}{Zubair} % Student last name
    \renewcommand*{\proofname}{Proof:}
	% \renewcommand*{\studentnum}{20836288} % Student number

	\renewcommand\qedsymbol{$\blacksquare$}
	\setfigpath
	% End of class settings	
	% \pagestyle{crowdmark}
	\newgeometry{left=18mm, right=18mm, top=22mm, bottom=22mm} % page is set to default values
	\fancyhfoffset[L,O]{0pt} % header orientation fixed
	% End of class settings
	%%% Note to user:
	% CTRL + F <CHANGE ME:> (without the angular brackets) in CKpreamble to specify graphics paths accordingly.
	% The command \circled[]{} accepts one optional and one mandatory argument.
	% Optional argument is for the size of the circle and mandatory argument is for its contents.
	% \circled{A} produces circled A, with size drawn for letter A. \circled[TT]{A} produces circled A with size drawn for TT.
	% https://github.com/CalvinKent/My-LaTeX
	%%%

	%%%%%%%%%%%%%%%%%%%%%%%%%%%%%%%%%%%%%%%%%%%%%%%%%%%%%%%%%%%%%%%%%%%%%%%%%%%%%%%%%%%%%%%%%%%%%%%%%%%%%%%%%%%%%%%%%%
	%%%                        CUSTOM MACRO VIM-TEX                                                      (Word Wrap->
	%%       call IMAP('NOM', '\nomenclature{}', 'tex')               

	%%%%%%%%%%%%%%%%%%%%%%%%%%%%%%%%%%%%%%%%%%%%%%%%%%%%%%%%%%%%%%%%%%%%%%%%%%%%%%%%%%%%%%%%%%%%%%%%%%%%%%%%%%%%%%%%%%

	% Crowdmark assignment start
	% qnumber, qname, points

\begin{center}
	\textbf{\underline{\Huge{Crafting a Thesis}}}
\end{center}

A focal point of Grade 11 and 12 English is to develop your ability to craft a thesis relative to a given topic. This lecture will brief you on a few technique's
you can use to aid in the process. Here are a few tips;

\begin{enumerate}
  \item \textbf{A thesis is never a questions} : Questions in academic essays are expected to be explored by the reader of the essay, but are not
    expected to be invoked by the author of the essay, rather it is that author that makes a statement, or a proposition, and then argues why its true
    \textit{within} the essay.  

  \item \textbf{A thesis does not contain a list of your argument} : For example, "For political, economic, social and cultural reasons, 
    communism collapsed in Eastern Europe" does a good job of "telegraphing" is a thesis that makes an assertion towards what the author believes is the reason behind
    the communist collapse in Eastern Europe. However, the thesis should \textbf{should not} list the primary arguments that the author will use throughout the essay, rather
    he should concise the \textit{modifier} by instead stating; "The governmental infrastructure within Easter Europe lead to the demise of its communist constitution". 

  \item \textbf{A thesis should not be ambiguous, subjective or confrontational} : For example, "Communism collapsed in Eastern Europe because communism is evil."
    is a weak thesis since it contains a subjective argument that could potentially conflict with a reader that happens to disagree with your opinion and turn them
    away. This also paints you as judgmental rather than rational and critical. 

  \item \textbf{A thesis should have an arguable claim or proposition} : "While cultural forces contributed to the collapse of communism in Eastern Europe, 
    the disintegration of economies played the key role in driving its decline" This thesis is precise, rational and arguable. It integrates the central proposition the author
    will attempt to prove within the essay as well as a few of the arguments that the author will choose as a focal point. It makes it crystal clear to the reader that the author
    will talk about both aspects of culture an economics, while perhaps emphasizing the role that economics played in diminishing the communist state. 

  \item \textbf{A thesis should be clear and specific} : Avoid using ambiguous, general term, modifiers and verbs when crafting your thesis. 
    "Communism collapsed in Eastern Europe because of the ruling elite's inability to address the economic concerns of the people" is much more precise and clear than 
    "Communism collapsed due to societal discontent."

\end{enumerate}

\textbf{Citation}: 

Copyright 1999, Maxine Rodburg and The Tutors of the Writing Center at Harvard University

  











\end{document}































