
% Prepared by Calvin Kent
%
% Assignment Template v19.02
%
%%% 20xx0x/MATHxxx/Crowdmark/Ax
%
\documentclass[12pt]{article} %
\usepackage{CKpreamble}
\usepackage{CKassignment}
\usepackage{tkz-euclide}
\usepackage{physunits}
\usepackage{physics}
\usepackage{mathtools}
\usepackage{multicol}
\usepackage{lmodern}
\usepackage{tikz}
\usepackage{pgfplots}
\usepackage{pdfpages}
\usepackage{euscript}
\usepackage{transparent}
\usepackage{xcolor}
\usepackage{tasks}
\usepackage{tkz-euclide}

\usepackage{pgfplots}
\usepgfplotslibrary{polar}
\usepgflibrary{shapes.geometric}
\usetikzlibrary{calc}


\usepackage{euscript}
\usepackage{microtype}
\usepackage{upgreek}
\usepackage[misc]{ifsym}

%%Title
\title{\textbf{Diagnostic Essay} \\ \textbf{Due Date: } Thursday, Feburary 10}
\date{}

%%% Maths and science packages

\usepackage{amsmath,amsthm,amssymb}
\usepackage{pgfplots}
	\usetikzlibrary{
		calc,
		patterns,
		positioning
	}
	\pgfplotsset{
		compat=1.16,
		samples=200,
		clip=false,
		my axis style/.style={
			axis x line=middle,
			axis y line=middle,
			legend pos=outer north east,
			axis line style={
				->,
			},
			legend style={
				font=\footnotesize
			},
			label style={
				font=\footnotesize
			},
			tick label style={
				font=\footnotesize
			},
			xlabel style={
				at={
					(ticklabel* cs:1)
				},
				anchor=west,
				font=\footnotesize,
			},
			ylabel style={
				at={
					(ticklabel* cs:1)
				},
				anchor=west,
				font=\footnotesize,
			},
			xlabel= $x$,
			ylabel=$\vec d (\m \tx{[East]})$
		},
	}
	\tikzset{
		>=stealth
	}

\pgfplotsset{my style/.append style={axis x line=middle, axis y line=
middle, xlabel={$t$}, ylabel={$y[\text{m}]$}, axis equal }}

%%% Tables and figures packages

\usepackage{float}
\usepackage{caption}
	\captionsetup{
		format=plain,
		labelfont=bf,
		font=small,
		justification=centering
	}
	
%%% Numbers and sets

\newcommand{\E}{\mathrm{e}}

\newcommand{\tx}[1]{\text{#1}}
\newcommand{\rem}[1]{\operatorname{rem}{(#1)}}


%
\begin{document}
	\pagenumbering{arabic}
	% Start of class settings ...
	\renewcommand*{\coursecode}{MATH 235} % renew course code
	\renewcommand*{\assgnnumber}{Assignment 1} % renew assignment number
	\renewcommand*{\submdate}{September 14, 2021} % renew the date
	\renewcommand*{\studentfname}{Abdullah} % Student first name
	\renewcommand*{\studentlname}{Zubair} % Student last name
    \renewcommand*{\proofname}{Proof:}
	% \renewcommand*{\studentnum}{20836288} % Student number

	\renewcommand\qedsymbol{$\blacksquare$}
	\setfigpath
	% End of class settings	
	% \pagestyle{crowdmark}
	\newgeometry{left=18mm, right=18mm, top=22mm, bottom=22mm} % page is set to default values
	\fancyhfoffset[L,O]{0pt} % header orientation fixed
	% End of class settings
	%%% Note to user:
	% CTRL + F <CHANGE ME:> (without the angular brackets) in CKpreamble to specify graphics paths accordingly.
	% The command \circled[]{} accepts one optional and one mandatory argument.
	% Optional argument is for the size of the circle and mandatory argument is for its contents.
	% \circled{A} produces circled A, with size drawn for letter A. \circled[TT]{A} produces circled A with size drawn for TT.
	% https://github.com/CalvinKent/My-LaTeX
	%%%

	%%%%%%%%%%%%%%%%%%%%%%%%%%%%%%%%%%%%%%%%%%%%%%%%%%%%%%%%%%%%%%%%%%%%%%%%%%%%%%%
	%%%                        CUSTOM MACRO VIM-TEX                             %%%
	%%       call IMAP('NOM', '\nomenclature{}', 'tex')               

	%%%%%%%%%%%%%%%%%%%%%%%%%%%%%%%%%%%%%%%%%%%%%%%%%%%%%%%%%%%%%%%%%%%%%%%%%%%%%%%

	% Crowdmark assignment start
	% qnumber, qname, points
\maketitle
	\section{Topic}
  \begin{align*}
    \text{\it{What does it mean to be a native speaker of a particular language?}}
  \end{align*}

  \section{Part A - Brainstorm}
  To assist in your brainstorming process, you can think about the following questions, however, what you choose to
  articulate on is not necessarily confined to these questions. To be more precise in our questions, suppose
  that we specifically analyse the English language, and ask the question of what it means to be a 
  native English speaker, 
  \begin{itemize}
    \item Do you \textit{have} to be born in an English speaking country?
    \item Is it age sensitive?
    \item Is it time sensitive? Meaning is there a certain prescribed period for which one can develop a native
      ability to retain English.
    \item Does it depend on whether or not your parents are native English speakers?
    \item Does it depend on whether or not your siblings are native English speakers?
    \item Does it depend on whether or not English was the official language throughout your educational upbringing?
    \item Does it depend on whether or not your friends spoke English?
    \item Must we judge native speakers based on their accent?
    \item Must we judge native speakers based on their style of speech? (Meaning, certain phrases that they use
      that are usually only understood amongst a certain set of people sharing that language)
    \item Must we judge native speakers based on their literacy skills?
  \end{itemize}

  \newpage

  \section{Part B - Rough Draft \& Peer Edit}
  After having brainstormed on a handful of the aforementioned questions, or perhaps questions you have come up with on
  your own, construct a rough draft of your essay and have it peer edited by a classmate.

  \section{Part C - Final Draft}
  When you feel like you are ready to formalize your rough draft, \textbf{type} out your final draft. The editor
  you choose to use is up to your preference, you can use \LaTeX if you like (of course there would be no
  bonus points for having done so) or you can use Microsoft Word. In either case, you are required to meet
  following criteria,
  \begin{itemize}
    \item Font size of 13.
    \item Double Spaced.
  \end{itemize}

  \section{Part D - Final Submission}
  Your final submission must include \textbf{both} the \textit{edited} rough draft as well as the final draft. The
  edited rough draft \textbf{must} include a signature of the person who edited the draft.







\end{document}



































