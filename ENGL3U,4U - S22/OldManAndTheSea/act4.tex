% Prepared by Calvin Kent
%
% Assignment Template v19.02
%
%%% 20xx0x/MATHxxx/Crowdmark/Ax
%
\documentclass[12pt]{article} %
\usepackage{amsthm}
\usepackage{CKpreamble}
\usepackage{CKassignment}
\usepackage{dirtytalk}
\usepackage{csquotes}
\usepackage{mdframed}
\usepackage{euscript}
\usepackage{tikz}
\usepackage{pgfplots}

%
\begin{document}
	\pagenumbering{arabic}
	% Start of class settings ...
	\renewcommand*{\coursecode}{MATH 235} % renew course code
	\renewcommand*{\assgnnumber}{Assignment 1} % renew assignment number
	\renewcommand*{\submdate}{September 14, 2021} % renew the date
	\renewcommand*{\studentfname}{Abdullah} % Student first name
	\renewcommand*{\studentlname}{Zubair} % Student last name
    \renewcommand*{\proofname}{Proof:}
	% \renewcommand*{\studentnum}{20836288} % Student number

	\renewcommand\qedsymbol{$\blacksquare$}
	\setfigpath
	% End of class settings	
	% \pagestyle{crowdmark}
	\newgeometry{left=18mm, right=18mm, top=22mm, bottom=22mm} % page is set to default values
	\fancyhfoffset[L,O]{0pt} % header orientation fixed
	% End of class settings
	%%% Note to user:
	% CTRL + F <CHANGE ME:> (without the angular brackets) in CKpreamble to specify graphics paths accordingly.
	% The command \circled[]{} accepts one optional and one mandatory argument.
	% Optional argument is for the size of the circle and mandatory argument is for its contents.
	% \circled{A} produces circled A, with size drawn for letter A. \circled[TT]{A} produces circled A with size drawn for TT.
	% https://github.com/CalvinKent/My-LaTeX
	%%%

	%%%%%%%%%%%%%%%%%%%%%%%%%%%%%%%%%%%%%%%%%%%%%%%%%%%%%%%%%%%%%%%%%%%%%%%%%%%%%%%%%%%%%%%%%%%%%%%%%%%%%%%%%%%%%%%%%%
	%%%                        CUSTOM MACRO VIM-TEX                                                      (Word Wrap->
	%%       call IMAP('NOM', '\nomenclature{}', 'tex')               

	%%%%%%%%%%%%%%%%%%%%%%%%%%%%%%%%%%%%%%%%%%%%%%%%%%%%%%%%%%%%%%%%%%%%%%%%%%%%%%%%%%%%%%%%%%%%%%%%%%%%%%%%%%%%%%%%%%

	% Crowdmark assignment start
	% qnumber, qname, points

\begin{center}
	\textbf{\underline{\Huge{Post Reading Discussion - Section 4}}}
\end{center}

\section*{Due Date: Monday, March 21}
\large{\textbf{Page Range :} 90 - END}



\section*{Discussion questions}
\begin{qstn}
  If Santiago's estimates are correct, how much is the fish worth? Is that a lot of money to Santiago?
\end{qstn}

\begin{qstn}
  Consider the image the author depicted (On page 99 $\sim$) of Santiago sailing side by side with the fish. What does this symbolize?
\end{qstn}

\begin{qstn}
  On page 101 ($\sim$), the author says "He was full of resolution but he had little hope." What does this mean?
\end{qstn}

\begin{qstn}
  Contemplate and reflect on Santiago's statement on page 103 ($\sim$), "A man can be destroyed but not defeated". Give examples of this from your own experience
  or knowledge. (This can be applied to females as well)
\end{qstn}

\begin{qstn}
  On page 119 ($\sim$), Santiago knows finally that he is beaten. Describe his reaction.
\end{qstn}

\begin{qstn}
  Describe Manolin's reaction when he finds Santiago asleep in the morning, how does he comfort him afterwards?
\end{qstn}

\section*{Written Assignment}
(*) The following written response questions are expected to be detailed, thorough and extensive (2 - 3 paragraphs).
\begin{enumerate}
  \item Most critics feel that Santiago is a hero. Define "hero" and explain whether or not you feel Santiago is one.
  \item Santiago was very fond of Joe DiMaggio. Explain why you think he identified with him
  \item Did you learn anything about yourself in your reflections throughout reading the novel?
\end{enumerate}







\end{document}































